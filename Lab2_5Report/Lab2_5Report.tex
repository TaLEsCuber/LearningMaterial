%!TEX program = xelatex
\documentclass[dvipsnames, svgnames,a4paper,11pt]{article}
% ----------------------------------------------------
%   中山大学物理与天文学院本科实验报告模板
%   作者:Huanyu Shi,2019级
%   知乎:https://www.zhihu.com/people/za-ran-zhu-fu-liu-xing
%   Github:https://github.com/huanyushi/SYSU-SPA-Labreport-Template
%   Last update : 2023.4.10
% ----------------------------------------------------

\input{Settings} % 导入模板的相关设置
\usepackage{lipsum}
\usepackage{enumitem}
\usepackage{tabularray}  %绘制表格时可以更加方便添加框线
\setlist[enumerate]{label=\textup{(\arabic*)}}
% \usepackage{subcaption}
% \usepackage{booktabs}


%---------------------------------------------------------------------
%	正文
%---------------------------------------------------------------------

\begin{document}


\begin{table}
	\renewcommand\arraystretch{1.7}
	\begin{tabularx}{\textwidth}{
		|X|X|X|X
		|X|X|X|X|}
	\hline
	\multicolumn{2}{|c|}{预习报告}&\multicolumn{2}{|c|}{实验记录}&\multicolumn{2}{|c|}{分析讨论}&\multicolumn{2}{|c|}{总成绩}\\
	\hline
	\LARGE25 & & \LARGE30 & & \LARGE25 & & \LARGE80 & \\
	\hline
	\end{tabularx}
\end{table}


\begin{table}
	\renewcommand\arraystretch{1.7}
	\begin{tabularx}{\textwidth}{|X|X|X|X|}
	\hline
	专业:& 物理学 &年级:& 2022级\\
	\hline
	姓名:& 戴鹏辉  & 学号: & 2344016 \\
	\hline
	日期:& 2024/4/9 & 教师签名:& \\
	\hline
	\end{tabularx}
\end{table}

\begin{center}
	\LARGE 实验CC3 \quad 双光栅测量微弱振动位移量实验
\end{center}

\textbf{【实验报告注意事项】}
\begin{enumerate}
	\item 实验报告由三部分组成:
	\begin{enumerate}
		\item 预习报告:(提前一周)认真研读\underline{\textbf{实验讲义}},弄清实验原理;实验所需的仪器设备、用具及其使用(强烈建议到实验室预习),完成课前预习思考题;了解实验需要测量的物理量,并根据要求提前准备实验记录表格(第一循环实验已由教师提供模板,可以打印)。预习成绩低于10分(共20分)者不能做实验。
	    \item 实验记录:认真、客观记录实验条件、实验过程中的现象以及数据。实验记录请用珠笔或者钢笔书写并签名(\textcolor{red}{\textbf{用铅笔记录的被认为无效}})。\textcolor{red}{\textbf{保持原始记录,包括写错删除部分,如因误记需要修改记录,必须按规范修改。}}(不得输入电脑打印,但可扫描手记后打印扫描件);离开前请实验教师检查记录并签名。
	    \item 分析讨论:处理实验原始数据(学习仪器使用类型的实验除外),对数据的可靠性和合理性进行分析;按规范呈现数据和结果(图、表),包括数据、图表按顺序编号及其引用;分析物理现象(含回答实验思考题,写出问题思考过程,必要时按规范引用数据);最后得出结论。
	\end{enumerate}
	\textbf{实验报告就是将预习报告、实验记录、和数据处理与分析合起来,加上本页封面。}
	\item 每次完成实验后的一周内交\textbf{实验报告}(特殊情况不能超过两周)。
	\item 除实验记录外,实验报告其他部分建议双面打印。
\end{enumerate}


\clearpage
\tableofcontents
\clearpage

\setcounter{section}{0}
\section{实验CC3 \quad 双光栅测量微弱振动位移量实验 \quad\heiti 预习报告}
	
\subsection{实验目的}
\begin{enumerate}
	\item 了解利用光的多普勒频移形成光拍的原理并用于测量光拍拍频。
	\item 学会使用精确测量微弱振动位移的一种方法。
	\item 应用双光栅微弱振动测量仪测量音叉振动的微振幅。
	
\end{enumerate}

\subsection{仪器用具}
\begin{table}[htbp]
	\centering
	\renewcommand\arraystretch{1.6}
	% \setlength{\tabcolsep}{10mm}
	\begin{tabular}{p{0.05\textwidth}|p{0.20\textwidth}|p{0.05\textwidth}|p{0.5\textwidth}}
	\hline
	编号& 仪器用具名称 & 数量 &  主要参数(型号,测量范围,测量精度等) \\
	\hline
	1&双光栅微弱振动测量仪 	&1 	& DHGS-1 型, 半导体激光器: $\lambda=650nm$,功率 2-5mW;音叉谐振频率: 500Hz 左右。\\

	2&数字示波器 	&1 	& DS1000E(D) \\
	
	3&信号发生器 &1 & MFG-2000\\
	
	\hline
\end{tabular}
\end{table}

\subsection{原理概述}

	\subsubsection{位移光栅的多普勒频移}

		多普勒频移是一个物理现象,它描述了由于波源和观察者之间的相对运动导致接收到的波频率与发射频率不同的情况。在位相光栅的多普勒频移中当光栅移动时,由于光栅上每个缝隙的衍射作用和缝隙之间的干涉,通过光栅后的光强度会出现周期性变化。这种变化可以用光栅衍射方程来描述。
		当光栅以速度$ v $在$ y $方向移动时,从光栅出射的光的波阵面也会以速度$ v $在$ y $方向移动。这导致了位移量$ vt $和相应的位相变化量$  \Delta\phi(t) $。位相变化量可以表示为:
		\[
			\Delta \phi(t)=2\pi\lambda v t \sin \theta
		\]
		% Δϕ(t)=2πλvt​sinθ
		将光栅衍射方程$ ( d \sin\theta = \pm k\lambda )$ 代入上述方程,我们得到:
		\[
			\Delta \phi(t)=k 2 \pi d v t
		\]
		% Δϕ(t)=k2πdvt​
		因此,从移动光栅出射的光波的电矢量方程可以表示为:
		\[
			E=E_0\cos(\omega_0t+\Delta \phi(t))=E_0\cos(\omega_0+k\omega_d)t
		\]
		% E=E0​cos[(ω0​t+Δϕ(t)]=E0​cos[(ω0​+kωd​)t]
		其中$  \omega_d = \frac{2\pi v}{d} $。这表明,从移动的位相光栅出射的$k$ 级衍射光波相对于静止的位相光栅有一个多普勒频移 $ \omega_a = \omega_0 + k\omega_d $。
		这个多普勒频移是由于光栅的移动导致的位相变化,这种变化反映在衍射光波的频率上。这就是位移光栅的多普勒频移现象。

		\begin{figure}[htbp]
			\centering
			\includegraphics[width=0.4\textwidth]{graph1-1.png}
			\caption{出射的折曲波阵面}
			\label{fig:graph1-1}
		\end{figure}

	\subsubsection{光拍的获得与检测}

		由于光波的频率非常高,直接测量多普勒频移是非常困难的。因此,通常采用“拍频”方法来检测频率的变化。拍频是两束频率相近的光波叠加时产生的低频率变化,这种变化比原始光波的频率要低得多,因此更容易检测。
		本实验使用了两片完全相同的光栅(A和B)来形成光拍。光栅A以速度$  v_A  $移动,起到频移作用,而光栅B保持静止,只起到衍射作用。当激光通过这两个光栅时,由于它们紧贴在一起,激光束的宽度使得两束光能够平行迭加,从而形成光拍。
		光拍信号的形成可以通过以下电矢量方程来描述:
		
		光束1(未频移): $( E_1 = E_{10} \cos(\omega_0 t + \phi_1) )$

		光束2(频移): $ ( E_2 = E_{20} \cos((\omega_0 + \omega_d) t + \phi_2) )$,其中$ k = 1$ 
		
		光电流 ( I ) 可以表示为两个电矢量的平方和:
		\[
			I=\xi [E_{1}^2+E_{2}^2+2E_1E_2\cos(\omega_d t+(\phi_2-\phi_1))]
		\]
		% I=ξ[E12​+E22​+2E1​E2​cos(ωd​t+(ϕ2​−ϕ1​))]
		其中$  \xi  $是光电转换常数。由于光波频率$  \omega_0 $ 非常高,光电检测器无法响应方程中的第一、二、四项,只能响应第三项,即拍频信号。因此,光电流 $ i_S $ 可以简化为:
		\[
			i_S=\xi[E_{10}E_{20}\cos(\omega_d t +(\phi_2-\phi_1))]
		\]
		% iS​=ξ[E10​E20​cos(ωd​t+(ϕ2​−ϕ1​))]
		拍频 $ F $ 可以通过以下公式计算:
		\[
			F=\frac{ω_d}{2\pi}​​=v_A​n_\theta​​
		\]
		% F=\frac{ω_d}{2\pi}​​=v_A​n_\theta​​
		其中 $ n_\theta $ 是光栅密度,本实验中 $ n_\theta = \frac{1}{d} $,( d = 100 ) 条/mm。
		通过测量拍频$  F $,我们可以间接地检测出多普勒频移量。


		\begin{figure}[htbp]
			\centering
			\subfloat[衍射光线在 y 方向上的位移量]
			{\includegraphics[width=0.31\textwidth]{graph1-2.png}\label{fig:graph1-2}}
			\quad
			\subfloat[动光栅的多普勒平移]
			{\includegraphics[width=0.31\textwidth]{graph1-3.png}\label{fig:graph1-3}}
			\quad
			
			% \caption{慧差实验结果图}
			% \label{fig:graph10}
		\end{figure}

	\subsubsection{微弱振动位移量的检测}

		微弱振动位移量的检测是通过分析光拍频率来实现的。这个过程不依赖于光源的频率,而是与光栅的移动速度成正比。当光栅粘贴在振动的音叉上时,光栅的移动速度会随着音叉的振动而周期性变化,从而导致光拍信号频率也随时间变化。以下是检测微弱振动位移量的步骤:
			\begin{enumerate}
				\item 拍频波的个数测量:首先,需要测量在一定时间内(例如音叉振动周期的一半)的拍频波的个数。这可以通过示波器来完成。
				\item 波形数的计算:波形数由完整波形数、波的首数和波的尾数三部分组成。对于不是完整波形的首数和尾数,可以使用反正弦函数将其转换为波形的分数部分。
				\item 位移振幅的计算:使用积分公式 $ A =\frac{1}{2n_\theta} \int_0^{T/2} F_{拍}(t)  dt $,可以计算出位移振幅,其中 $ F_{拍}(t)  $是拍频波的频率,$\int_0^{T/2} F_{拍}(t)  dt$就表示T/2时间内的拍频波的数量,$ A  $是位移振幅。
				\item 示波器读数:通过示波器读取波形数,包括整数波形数和分数波形数。
				\item 波形数的最终计算:将整数波形数和分数波形数相加,得到总波形数。如果波群的首尾幅度与完整波形的振幅之比为 ( a ) 和 ( b ),则波形数可以表示为:
				波形数=整数波形数+波的首数和尾数中满 1/2 或 1/4 或 3/4 个波形分数部份+$\frac{\sin^{-1}a}{360^\circ}$+$\frac{\sin^{-1}b}{360^\circ}$

				式中 a、 b 为波群的首、尾幅度和该处完整波形的振幅之比。
			\end{enumerate}

		微弱振动位移量的确定:通过上述步骤,可以确定微弱振动的位移振幅。


		\begin{figure}[htbp]
			\centering
			\subfloat[激光通过双光栅后所形成的衍射光]
			{\includegraphics[width=0.35\textwidth]{graph1-4.png}\label{fig:graph1-4}}
			\quad
			\subfloat[频差较小的两列光波叠加形成“拍”]
			{\includegraphics[width=0.35\textwidth]{graph1-5.png}\label{fig:graph1-5}}
			\quad
			
			% \caption{慧差实验结果图}
			% \label{fig:graph10}
		\end{figure}




\subsection{实验前思考题}
\begin{question}
	简述位移光栅的多普勒频移原理。
\end{question}



\begin{question}
	简述光拍的获得与检测方法和原理。
\end{question}
	

\begin{question}
	简述微弱振动位移量的检测方法和原理。
\end{question}

\textbf{三道思考题详见前面原理概述部分}

\clearpage
\begin{table}
	\renewcommand\arraystretch{1.7}
	\centering
	\begin{tabularx}{\textwidth}{|X|X|X|X|}
	\hline
	专业:& 物理学 &年级:& 2022级 \\
	\hline
	姓名:& 戴鹏辉 & 学号:& 22344016 \\
	\hline
	室温:& 26℃ & 实验地点: & A511 \\
	\hline
	学生签名:& & 评分: &\\
	\hline
	实验时间:& 2024/4/11 & 教师签名:&\\
	\hline
	\end{tabularx}
\end{table}

\section{实验CC3 \quad 双光栅测量微弱振动位移量实验 \quad\heiti 实验记录}
\subsection{实验内容和步骤}

	\begin{figure}[htbp]
		\centering
		% \subfloat[放置软管前]
		{\includegraphics[width=0.6\textwidth]{graph2-0.png}\label{fig:graph2-0}}
		\quad

		\caption{双光栅实验平台}
		% \label{fig:graph10}
	\end{figure}



	\begin{enumerate}
		% \item 熟悉双踪示波器的使用方法。
		% \item 熟悉固纬信号发生器的使用方法。(使用前看仪器说明书)
		\item 将固纬信号发生器的 CH1 通道连接至光学实验平台音叉驱动器(动光栅), 输出频率约为 500 Hz、 $V_{pp} $约为 6 V 的正弦波驱动信号。
		\item 将光学实验平台光电传感器输出信号连接至示波器的 CH1 通道。因为光电传感器输出信号功率较低,示波器 CH1 探头选择放大约 500 倍。
		\item 将固纬信号发生器同步信号输出(后面板 SYNC)连接至示波器 CH2 作为触发源。
		% \item 检查上述 3、 4、 5 步骤,注意不要接错。
		\item 几何光路调整。 光学实验平台上的“激光器”接测试仪“半导体激光电源”,将激光器、静光栅、动光栅摆在一条直线上。打开半导体激光电源,让激光穿越静、动光栅后形成一竖排衍射光斑,使中间最亮光斑进入光电传感器里面,调节静光栅和动光栅的相对位置,使两光栅尽可能平行。
		\item 音叉谐振调节。先调整好实验平台上音叉和激振换能器的间距,一般 0.3mm 为宜,可使用塞尺辅助调节。将固纬信号发生器的 CH1 通道连接至光学实验平台音叉驱动器(动光栅),输出频率约为 500 Hz、 $V_{pp} $ 约为 6 V 的正弦波驱动信号。 在 500Hz 附近连续微调(间隔 0.1Hz或更小) 输出频率, 直到使得音叉谐振(音叉振幅最大,光拍波数最多),调节时可用手轻轻地按音叉顶部感受振动强弱,或听振动声音,找出调节方向。若音叉谐振太强烈,可调小驱动信号幅度,使振动减弱,在示波器上看到的 T/2 内光拍的波数为 15 个左右。记录此时音叉振动频率、屏上完整波的个数、不足一个完整波形的首数和尾数值以及对应该处完整波形的振幅值。
		\item 测出外力驱动音叉时的谐振曲线。 在音叉谐振点附近,调节驱动信号频率,测出音叉的振动频率与对应的音叉振幅大小,频率间隔可以取 0.1HZ,选 8 个点,分别测出对应的波的个数,计算出各自的振幅 A。
		\item 保持驱动信号输出幅度不变,将软管放入音叉上的小孔从而改变音叉的有效质量,调节驱动信号频率,研究谐振曲线的变化趋势。
	\end{enumerate}

	由公式$ A =\frac{1}{2n_\theta} \int_0^{T/2} F_{拍}(t)  dt $可计算出音叉振动的位移振幅,其中实验用的动光栅有$n_\theta=100条/mm$


	\textbf{得到的测量数据如下表所示:}
		

		\begin{table}[htbp]
			\centering
			\begin{minipage}{.5\linewidth}
			  \centering
			%   \caption{数据:直流稳压电源的伏安特性}
			  	\begin{tblr}{
					cells = {c},
					vlines,
					hline{1-2,10} = {-}{},
					}
					f/Hz   & T/2内的波数 & 振动幅度/mm \\
					503.68 & 1.75    & 0.00875 \\
					503.78 & 3.75    & 0.01875 \\
					503.88 & 5.75    & 0.02875 \\
					\textbf{503.98} & \textbf{14 }     &\textbf{ 0.07000} \\
					504.08 & 6.25    & 0.03125 \\
					504.18 & 3.5     & 0.01750 \\
					504.28 & 2.5     & 0.01250 \\
					504.38 & 2.25    & 0.01125 
					\end{tblr}
					\label{tab:tab1}	
				\end{minipage}%
			\begin{minipage}{.5\linewidth}
			  \centering
			%   \caption{数据:数控恒流源的伏安特性}
			\begin{tblr}{
				cells = {c},
				vlines,
				hline{1-2,10} = {-}{},
				}
				f/Hz   & T/2内的波数 & 振动幅度/mm  \\
				502.48 & 1.875   & 0.009375 \\
				502.58 & 2       & 0.010000 \\
				502.68 & 3.125   & 0.015625 \\
				502.78 & 5.875   & 0.029375 \\
				\textbf{502.88} & \textbf{6.5 }    & \textbf{0.032500} \\
				502.98 & 4       & 0.020000 \\
				503.08 & 3.125   & 0.015625 \\
				503.18 & 2.125   & 0.010625 
			\end{tblr}
			  \label{tab:tab2}
			\end{minipage}
			
			
		\end{table}
		
%\subsection{实验数据记录}

	\begin{figure}[htbp]
		\centering
		\subfloat[加入软管前谐振点图像,$f=503.98Hz$,T/2波数为14]
		{\includegraphics[width=0.45\textwidth]{graph2-1.jpg}\label{fig:graph2-1}}
		\quad
		\subfloat[加入软管后谐振点图像,$f=502.88Hz$,T/2波数为6.5]
		{\includegraphics[width=0.45\textwidth]{graph2-2.jpg}\label{fig:graph2-2}}
		\quad
		% \caption{慧差实验结果图}
		\label{fig:graph10}
	\end{figure}

\subsection{原始数据记录}

	\begin{figure}[htbp]
		\centering
		\includegraphics[width=9.2cm]{data.jpg}
		\caption{实验原始数据记录}
	\end{figure}



\subsection{实验过程中遇到的问题记录}

\begin{enumerate}
	\item 音叉和激振换能器的间距较难调整,间距过大的话会因为未与音叉接触而导致音叉不震动,过小的话会因为直接接触导致激振换能器本身无法振动。
	\item 在调整光路时,一定要保证激光光路未被遮挡,并且使最亮的光斑进入传感器。
	
\end{enumerate}
	

\clearpage
\begin{table}
	\renewcommand\arraystretch{1.7}
	\begin{tabularx}{\textwidth}{|X|X|X|X|}
	\hline
	专业:& 物理学 &年级:& 2022级\\
	\hline
	姓名: & 戴鹏辉 & 学号:& 22344016\\
	\hline
    日期:& 2024/4/16 & 评分: &\\
	\hline
	\end{tabularx}
\end{table}

\section{实验CC3 \quad 双光栅测量微弱振动位移量实验 \quad\heiti 分析与讨论}

\subsection{实验数据分析}

	\subsubsection{求出音叉谐振时光拍信号的平均频率}

	音叉谐振时,即T/2时间内拍波数量最多时,为$f=503.98Hz$,T/2波数为14。

	则由此可计算出出光拍信号的平均频率为:
	\[
		\bar{F}=2\times14\times f=14111.44Hz
	\]





	\subsubsection{求出音叉在谐振点时作微弱振动的位移振幅。}


	由公式$ A =\frac{1}{2n_\theta} \int_0^{T/2} F_{拍}(t)  dt $可计算出音叉振动的位移振幅,其中实验用的动光栅有$n_\theta=100条/mm$

	则在谐振点时,微振动振幅为:
	\[
		A_{max}=\frac{14}{2\times100}mm=0.07mm	
	\]




	\subsubsection{画出音叉的频率—振幅曲线;分析讨论其特点。}




	做出频幅响应曲线如\cref{fig:graph3-2a}所示:


	\begin{figure}[htbp]
		\centering
		\subfloat[放置软管前]
		{\includegraphics[width=0.45\textwidth]{graph3-2a.png}\label{fig:graph3-2a}}
		\quad
		\subfloat[放置软管后]
		{\includegraphics[width=0.45\textwidth]{graph3-2b.png}\label{fig:graph3-2b}}
		\quad
		\caption{频幅响应数据连线图}
		\label{fig:graph10}
	\end{figure}

	% 由拟合数据可知,谐振点频率计算值为$\omega_0=503.985Hz$,与实验值$$


	由图像可知,幅度相应在谐振点附近最大,稍微偏离谐振点后便迅速降低并趋于0。



	在该实验的条件下,可认为音叉做受迫振动,运动方程为:
	\[
	\frac{d^2x}{dt^2}+2\beta\frac{dx}{dt}+\omega_0^2x=\frac{F}{m}\cos(\omega t)	
	\]

	可解得稳态解的振幅为:
	\[
	A=\frac{F}{m\sqrt{(\omega_0^2-\omega^2)^2+4\beta^2\omega^2}}	
	\]

	则以该公式为基础,做出数据点的拟合图像,如\cref{fig:graph3-3a}所示:

	由拟合数据可知,放置软管前的音叉谐振频率为$f_0=\frac{\omega_0}{2\pi}=503.985Hz$,与实验值$f=503.98Hz$接近,相对误差$\eta=\frac{f_0-f}{f_0}=0.001\%$。

	\begin{figure}[htbp]
		\centering
		\includegraphics[width=0.8\textwidth]{graph3-3a.png}\label{fig:graph3-3a}
		\caption{放置软管前实验拟合图像}
	\end{figure}
		
	\begin{figure}[htbp]
		\centering
		\includegraphics[width=0.8\textwidth]{graph3-3b.png}\label{fig:graph3-3b}
		\caption{放置软管后实验拟合图像}
	\end{figure}

	% \begin{figure}[htbp]
	% 	\centering
	% 	\subfloat[放置软管前]
	% 	{\includegraphics[width=0.9\textwidth]{graph3-3a.png}\label{fig:graph3-3a}}
	% 	\quad
	% 	\subfloat[放置软管后]
	% 	{\includegraphics[width=0.9\textwidth]{graph3-3b.png}\label{fig:graph3-3b}}
	% 	\quad
	% 	% \caption{慧差实验结果图}
	% 	\label{fig:graph3-3ab}
	% \end{figure}




	\subsubsection{作出音叉不同有效质量时的谐振曲线,分析讨论其变化趋势。}
		

	% \begin{figure}[htbp]
	% 	\centering
	% 	% \subfloat[放置软管前]
	% 	{\includegraphics[width=0.45\textwidth]{graph3-3.png}\label{fig:graph3-3}}
	% 	\quad

	% 	\caption{不同有效质量对比图}
	% 	% \label{fig:graph10}
	% \end{figure}
	


	% \begin{figure}[htbp]
	% 	\centering
	% 	\includegraphics[width=0.6\textwidth]{graph3-3.png}
	% 	\caption{不同有效质量时的谐振曲线}
	% 	\label{fig:graph3-3}
	% \end{figure}	


	\begin{wrapfigure}{r}{0.5\textwidth}
		\centering
		\includegraphics[width=0.37\textwidth]{graph3-3.png}
		\label{fig:graph3-3}
		\caption{不同有效质量对比图}
	  \end{wrapfigure}

	由\cref{fig:graph3-3}可知,在增加软管后,音叉的等效质量增加,导致谐振频率左移,同时谐振点振幅明显减小。

	% 由公式$A=\frac{F}{m\sqrt{(\omega_0^2-\omega^2)^2+4\beta^2\omega^2}}$可知,有效质量$m$增大,振幅$A$减小,与实验数据相符。			
			
	同样将得到的数据做拟合,结果如\cref{fig:graph3-3b}所示。

	


	对比两组拟合数据可发现,增加软管后,音叉的有效质量增加,导致振幅减小,且阻尼系数$\beta$增大,导致谐振频率增大。
			
			
% \subsection{实验后思考题}

% \begin{question}
% 	检索文献,列举三种测量光波波长的方法,给出参考文献列表。%\lipsum[20]
% \end{question}
	
\subsection{实验后实验桌整理}

如\cref{fig:table}所示。

\begin{figure}[htbp]
	\centering
	\includegraphics[width=9.41cm]{table.jpg}
	
	\caption{整理后实验桌}
	\label{fig:table}
\end{figure}
	

\end{document}
