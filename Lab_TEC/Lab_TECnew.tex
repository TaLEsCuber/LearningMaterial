%!TEX program = xelatex
\documentclass[dvipsnames, svgnames,a4paper,11pt]{article}
% ----------------------------------------------------
%   中山大学物理与天文学院本科实验报告模板
%   作者:Huanyu Shi,2019级
%   知乎:https://www.zhihu.com/people/za-ran-zhu-fu-liu-xing
%   Github:https://github.com/huanyushi/SYSU-SPA-Labreport-Template
%   Last update : 2023.4.10
% ----------------------------------------------------

\input{Settings} % 导入模板的相关设置
\usepackage{lipsum}
\usepackage{enumitem}
\usepackage{tabularray}  %绘制表格时可以更加方便添加框线
\setlist[enumerate]{label=\textup{(\arabic*)}}



%---------------------------------------------------------------------
%	正文
%---------------------------------------------------------------------

\begin{document}


\begin{table}
	\renewcommand\arraystretch{1.7}
	\begin{tabularx}{\textwidth}{
		|X|X|X|X
		|X|X|X|X|}
	\hline
	\multicolumn{2}{|c|}{预习报告}&\multicolumn{2}{|c|}{实验记录}&\multicolumn{2}{|c|}{分析讨论}&\multicolumn{2}{|c|}{总成绩}\\
	\hline
	\LARGE25 & & \LARGE30 & & \LARGE25 & & \LARGE80 & \\
	\hline
	\end{tabularx}
\end{table}


\begin{table}
	\renewcommand\arraystretch{1.7}
	\begin{tabularx}{\textwidth}{|X|X|X|X|}
	\hline
	专业:& 物理学 &年级:& 2022级\\
	\hline
	姓名:& 戴鹏辉\&喻欣宸  & 学号: & 2344016\&22344014 \\
	\hline
	日期:& 2024/5/20 & 教师签名:& \\
	\hline
	\end{tabularx}
\end{table}

\begin{center}
	\LARGE BD4  \quad TEC 半导体控温实验
\end{center}

\textbf{【实验报告注意事项】}
\begin{enumerate}
	\item 实验报告由三部分组成:
	\begin{enumerate}
		\item 预习报告:(提前一周)认真研读\underline{\textbf{实验讲义}},弄清实验原理;实验所需的仪器设备、用具及其使用(强烈建议到实验室预习),完成课前预习思考题;了解实验需要测量的物理量,并根据要求提前准备实验记录表格(第一循环实验已由教师提供模板,可以打印)。预习成绩低于10分(共20分)者不能做实验。
	    \item 实验记录:认真、客观记录实验条件、实验过程中的现象以及数据。实验记录请用珠笔或者钢笔书写并签名(\textcolor{red}{\textbf{用铅笔记录的被认为无效}})。\textcolor{red}{\textbf{保持原始记录,包括写错删除部分,如因误记需要修改记录,必须按规范修改。}}(不得输入电脑打印,但可扫描手记后打印扫描件);离开前请实验教师检查记录并签名。
	    \item 分析讨论:处理实验原始数据(学习仪器使用类型的实验除外),对数据的可靠性和合理性进行分析;按规范呈现数据和结果(图、表),包括数据、图表按顺序编号及其引用;分析物理现象(含回答实验思考题,写出问题思考过程,必要时按规范引用数据);最后得出结论。
	\end{enumerate}
	\textbf{实验报告就是将预习报告、实验记录、和数据处理与分析合起来,加上本页封面。}
	\item 每次完成实验后的一周内交\textbf{实验报告}(特殊情况不能超过两周)。
	\item 实验报告注意事项
		\begin{enumerate}[label=\roman*.]
			\item 本实验电路比较复杂, 需要耐心连接, 认真检查。
			\item 程序编写和调试过程中, 遇到问题及时查看 LabVIEW 帮助, 留意各个器件时刻处于可控状态。
			\item 直流稳压电源的输出,要根据实验过程及时地手动打开或关闭,不要在程序没有正常运行的情况下,让直流稳压电源有持续的输出。
			\item 注意根据信号电流大小选取合适直接的导线,杜邦线一般只用于小电流和信号传输,电流大时则需要选用直径大的导线。
		\end{enumerate}
\end{enumerate}


\clearpage
\tableofcontents
\clearpage

\setcounter{section}{0}
\section{BD4 \quad TEC 半导体控温实验 \quad\heiti 预习报告}
	
\subsection{实验目的}
\begin{enumerate}
	\item 了解TEC半导体制冷的原理
	\item PID反馈控制的参数调节。
	\item 基于LabVIEW和myDAQ设计和搭建温度采集和控制系统。
	
	
\end{enumerate}

\subsection{仪器用具}
\begin{table}[htbp]
	\centering
	\renewcommand\arraystretch{1.6}
	% \setlength{\tabcolsep}{10mm}
	\begin{tabular}{p{0.05\textwidth}|p{0.20\textwidth}|p{0.05\textwidth}|p{0.5\textwidth}}
		\hline
		编号& 仪器用具名称 			& 数量 		&  主要参数(型号,测量范围,测量精度等) \\
		\hline
		1	&	计算机 				&1 			& Windows 系统, 安装 LabVIEW 程序\\

		2	&	myDAQ 				&1 			& 		 \\
		
		3	&	直流稳压电源 		& 1 		&	DP832 \\
		
		4	&	控温实验平台		&1 			&   \\
		
		5	&	$IBT\_2$电机驱动器	&	1 		& 12V供电,最大电流2A \\

		6	&	温度变送器			&	1 		&		24V 供电, 测温范围 0-100℃, 输出电压 0-10V \\

		7	&	电路元器件			&	1套 	& 面包板, 连接线等 \\
		\hline
	\end{tabular}
\end{table}

\subsection{原理概述}

    \subsubsection{PID反馈控制原理}

        PID控制器是一种广泛应用于工业控制系统的反馈控制器,PID代表比例(Proportional),积分(Integral)和微分(Derivative)。PID控制器通过调节这三个参数,使系统输出跟随设定值,达到最佳控制效果。

        \begin{figure}[htbp]
            \centering
            \includegraphics[width=0.6\textwidth]{graph1-1.png}
            \caption{PID控制原理图}
            \label{fig:fig1-1}
        \end{figure}

        \paragraph{比例控制(P)}

            比例控制器根据当前误差(设定值与实际值之间的差值)进行调节。比例控制器的输出与误差成正比,即
            \[ u(t) = K_p e(t) \]
            其中,$u(t)$为控制输出,$K_p$为比例增益,$e(t)$为误差。

        \paragraph{积分控制(I)}

        积分控制器根据误差的累积进行调节,主要用于消除稳态误差。积分控制器的输出与误差的积分成正比,即
        \[ u(t) = K_i \int_0^t e(\tau) d\tau \]
        其中,$K_i$为积分增益。

        \paragraph{微分控制(D)}

        微分控制器根据误差的变化率进行调节,主要用于预见误差的变化趋势。微分控制器的输出与误差的导数成正比,即
        \[ u(t) = K_d \frac{de(t)}{dt} \]
        其中,$K_d$为微分增益。

        \paragraph{PID控制器的总输出}

        综合比例、积分和微分控制的作用,PID控制器的总输出为
        \[ u(t) = K_p e(t) + K_i \int_0^t e(\tau) d\tau + K_d \frac{de(t)}{dt} \]


        \paragraph*{调节 PID 参数的一般方法}

            \begin{enumerate}
                \item \textbf{初始设置:}
                \begin{itemize}
                    \item 将积分增益 $I$ 和微分增益 $D$ 设置为零。
                    \item 增大比例增益 $P$,直到输出信号(或被控物理量)出现振荡。然后将比例增益值减小至目前数值的一半左右。
                \end{itemize}
                
                \item \textbf{调整积分增益:}
                \begin{itemize}
                    \item 设置好比例增益后,增大积分增益 $I$,直到任何漂移都根据用户系统所适用的时间尺度进行校正。如果积分增益 $I$ 过大,将会观察到设定值的严重超调和系统不稳定现象。
                \end{itemize}
                
                \item \textbf{调整微分增益:}
                \begin{itemize}
                    \item 积分增益 $I$ 设定好后,开始增大微分增益 $D$。微分增益 $D$ 可以减小超调,并阻碍系统振荡,使其快速稳定达到目标值。如果微分增益 $D$ 过大,会观察到大幅度的超调(因为系统响应过于缓慢)。
                \end{itemize}
            \end{enumerate}
        
            通过调节 PID 增益,可以使系统达到最佳工作状态,实现快速响应,并有效抑制目标值附近的振荡。
        
            专业的 PID 调节方法中,齐格勒-尼柯尔斯 (Ziegler-Nichols) 方法应用较广。具体步骤如下:
        
            \begin{enumerate}
                \item 将积分增益 $I$ 和微分增益 $D$ 设置为零。
                \item 增大比例增益 $P$,直到系统开始出现振荡,测出此时的增益 $K_u$ 和振荡周期 $T_u$。
                \item 根据不同控制电路类型的增益,参照下表设置 $P$、$I$ 和 $D$ 值。
            \end{enumerate}
        
            PID 控制器的输出可以表示为:
            \[
            u(t) = K_p \left( e(t) + \frac{1}{T_i} \int_0^t e(\tau) \, d\tau + T_d \frac{d e(t)}{d t} \right)
            \]
        
            齐格勒-尼柯尔斯方法的一些典型设置如下:
            
            \[
            \begin{array}{|c|c|c|c|}
            \hline
            \text{控制类型} & K_p & T_i & T_d \\
            \hline
            \text{P} & 0.5 K_u & - & - \\
            \text{PI} & 0.45 K_u & \frac{T_u}{1.2} & - \\
            \text{PID} & 0.6 K_u & \frac{T_u}{2} & \frac{T_u}{8} \\
            \hline
            \end{array}
            \]
        
            这套方法可以作为设定 PID 值的指南,用以帮助实现系统的最佳控制效果。
        


    \subsubsection{半导体控温原理}

        半导体控温基于热电效应,主要包括帕尔帖效应(Peltier Effect)和塞贝克效应(Seebeck Effect)。

        \paragraph*{帕尔帖效应}

        帕尔帖效应描述了当电流通过两个不同类型的半导体材料的接合处时,会吸收或释放热量。具体来说,当电流从p型半导体流向n型半导体时,接合处会吸收热量,形成制冷效果;反之,接合处会释放热量,形成加热效果。

        \paragraph*{塞贝克效应}

        塞贝克效应描述了当两个不同类型的半导体材料的接合处存在温差时,会产生电动势。这种效应用于温度测量和热电发电。

        \paragraph*{两者之间的关系}

        帕尔帖效应和塞贝克效应是热电效应的两个方面。帕尔帖效应用于温度控制,通过电流驱动实现制冷或加热;塞贝克效应用于温度测量或能量回收,通过温差产生电动势。这两种效应都依赖于材料的热电性能,是热电技术的基础。

    \subsubsection{傅里叶变换基本原理}

        傅里叶变换是一种将时间域信号转换为频率域信号的数学变换。它基于任何周期信号都可以表示为一组正弦波和余弦波的叠加。

        \paragraph*{傅里叶变换公式}

        傅里叶变换的数学表达式为
        \[ X(f) = \int_{-\infty}^{\infty} x(t) e^{-j 2 \pi f t} dt \]
        其中,$X(f)$为频率域表示,$x(t)$为时间域信号,$j$为虚数单位,$f$为频率。

        \paragraph*{逆傅里叶变换}

        逆傅里叶变换用于将频率域信号转换回时间域信号,其表达式为
        \[ x(t) = \int_{-\infty}^{\infty} X(f) e^{j 2 \pi f t} df \]

    \subsubsection*{功率谱密度或幅度谱密度基本原理}

    功率谱密度(Power Spectral Density, PSD)和幅度谱密度(Amplitude Spectral Density, ASD)用于描述信号在频率域的能量分布。

        \paragraph*{功率谱密度}

        功率谱密度表示每单位频率区间内的功率分布,其定义为
        \[ S_{xx}(f) = \lim_{T \to \infty} \frac{1}{T} \mathbb{E} \left[ \left| X_T(f) \right|^2 \right] \]
        其中,$X_T(f)$为信号$x(t)$在区间$[-T/2, T/2]$上的傅里叶变换,$\mathbb{E}$为期望值。

        \paragraph*{幅度谱密度}

        幅度谱密度表示每单位频率区间内的振幅分布,其定义为
        \[ A(f) = \sqrt{S_{xx}(f)} \]

        \paragraph*{应用}

        功率谱密度用于分析信号的频率成分和噪声特性,广泛应用于信号处理、通信系统和振动分析。幅度谱密度用于表征信号的振幅分布,常用于振动分析和频谱测量。







\clearpage

\subsection{实验前思考题}



% 思考题1
\begin{question}
	热敏电阻测温点位置不同, 是否会影响PID 参数的设定, 为什么?
\end{question}


热敏电阻测温点的位置确实会影响PID参数的设定,原因如下:

\paragraph*{系统响应时间的变化}

热敏电阻的位置会影响系统的响应时间。具体来说,测温点越靠近热源,响应时间越快,测温点越远离热源,响应时间越慢。不同的响应时间会直接影响PID控制器的调节效果:

\begin{itemize}
    \item \textbf{靠近热源的测温点}:由于温度变化快,系统响应时间短,需要设置较小的比例增益($K_p$)和微分增益($K_d$)以避免超调和振荡。同时,可能需要较小的积分增益($K_i$)以防止积分饱和。
    \item \textbf{远离热源的测温点}:温度变化慢,系统响应时间长,需要设置较大的比例增益($K_p$)和微分增益($K_d$)以提高系统响应速度,同时可能需要较大的积分增益($K_i$)以减少稳态误差。
\end{itemize}

\paragraph*{温度梯度的影响}

在实际应用中,测温点位置不同会导致测得的温度值有所差异,尤其是在存在明显温度梯度的系统中。温度梯度会影响PID控制的精确度和稳定性:

\begin{itemize}
    \item \textbf{温度梯度较大的系统}:测温点位置差异较大,可能会导致测得的温度不准确,从而影响PID控制器的输出。需要通过调整PID参数来补偿这一影响。
    \item \textbf{温度梯度较小的系统}:测温点位置对测温精度影响较小,PID参数的调整幅度较小。
\end{itemize}

\paragraph*{热传导路径的影响}

测温点的位置决定了热传导路径的长短,进而影响系统的热惯性和响应速度:

\begin{itemize}
    \item \textbf{热传导路径短}:热惯性小,响应速度快,PID参数需要更加灵敏的调整。
    \item \textbf{热传导路径长}:热惯性大,响应速度慢,PID参数需要更平缓的调整以防止振荡。
\end{itemize}

\paragraph*{总结}

综上所述,热敏电阻测温点位置不同会显著影响PID参数的设定。测温点位置靠近热源或远离热源、系统的温度梯度大小和热传导路径的长短都会改变系统的响应特性,从而需要通过调整PID参数来优化控制效果。




% 思考题2
\begin{question}
	被控物体材料不同, 是否会影响PID 参数的设定, 为什么?
\end{question}



被控物体的材料不同会影响PID参数的设定,原因如下:

\paragraph*{热物性参数的差异}

不同材料的热物性参数(如热导率、比热容和密度)不同,会导致系统的热响应特性差异:

\begin{itemize}
    \item \textbf{热导率}:热导率高的材料传热快,系统的响应时间短,可能需要较小的比例增益($K_p$)和微分增益($K_d$)以防止过调和振荡。相反,热导率低的材料传热慢,响应时间长,可能需要较大的比例增益和微分增益以提高响应速度。
    \item \textbf{比热容}:比热容大的材料吸收热量多,温度变化慢,需要较大的积分增益($K_i$)以消除稳态误差。而比热容小的材料温度变化快,可能需要较小的积分增益以防止积分饱和。
    \item \textbf{密度}:材料的密度影响其热惯性。密度大的材料热惯性大,响应慢,PID参数需要更平缓的调整。密度小的材料热惯性小,响应快,PID参数需要更加灵敏的调整。
\end{itemize}

\paragraph*{热惯性的影响}

材料的热惯性决定了系统对温度变化的响应速度。热惯性大的材料(如金属)响应慢,可能需要较大的$K_p$和$K_d$以提高响应速度,同时需要较大的$K_i$以减少稳态误差。而热惯性小的材料(如塑料)响应快,需要较小的$K_p$和$K_d$以防止过调和振荡,$K_i$也需要相应调整以防止积分饱和。

\paragraph*{热传导路径的差异}

不同材料的热传导路径也会影响PID参数的设定。例如,在热传导路径较短的材料中,系统的热响应速度快,需要更加灵敏的PID参数调整。而在热传导路径较长的材料中,系统的热响应速度慢,需要更平缓的PID参数调整。

\paragraph*{热容的差异}

材料的热容影响其温度变化的幅度和速度。热容大的材料吸收热量多,温度变化慢,可能需要较大的$K_i$以减少稳态误差。热容小的材料温度变化快,需要较小的$K_i$以防止积分饱和。

\paragraph*{总结}

综上所述,被控物体的材料不同会显著影响PID参数的设定。不同材料的热物性参数(热导率、比热容、密度)、热惯性和热传导路径会导致系统的热响应特性差异,从而需要通过调整PID参数来优化控制效果。




% 思考题3
\begin{question}
	除了脉冲宽度调制控温, 可以使用连续信号控温吗? 如何实现?
\end{question}



连续信号控温通过调节一个连续变化的控制信号(如电压或电流)来控制加热或制冷元件的输出功率,从而实现温度调节。以下是几种常见的连续信号控温方法及其实现方式:

\paragraph*{模拟电压控制}

通过调节模拟电压来控制加热器或制冷器的输出功率。这种方法通常使用模拟控制器,如运算放大器和线性调节器。

\begin{itemize}
    \item \textbf{电压源}:提供一个可调节的电压信号。
    \item \textbf{线性调节器}:根据输入的模拟电压信号调整输出功率。
    \item \textbf{温度传感器}:测量当前温度,并将温度信号反馈给控制器。
    \item \textbf{PID控制器}:根据温度反馈信号调整输出电压,实现温度控制。
\end{itemize}

实现步骤:

\begin{enumerate}
    \item 将温度传感器连接到PID控制器,以测量当前温度。
    \item PID控制器根据设定的目标温度计算误差,并输出一个相应的控制电压信号。
    \item 线性调节器根据PID控制器输出的控制电压信号调整加热器或制冷器的功率。
    \item 温度传感器实时测量温度,并将反馈信号传递给PID控制器,形成闭环控制。
\end{enumerate}



\paragraph*{模拟电流控制}

类似于模拟电压控制,通过调节模拟电流来控制加热或制冷元件的输出功率。这种方法通常使用电流源和电流控制器。

\begin{itemize}
    \item \textbf{电流源}:提供一个可调节的电流信号。
    \item \textbf{电流控制器}:根据输入的模拟电流信号调整输出功率。
    \item \textbf{温度传感器}:测量当前温度,并将温度信号反馈给控制器。
    \item \textbf{PID控制器}:根据温度反馈信号调整输出电流,实现温度控制。
\end{itemize}

实现步骤:

\begin{enumerate}
    \item 将温度传感器连接到PID控制器,以测量当前温度。
    \item PID控制器根据设定的目标温度计算误差,并输出一个相应的控制电流信号。
    \item 电流控制器根据PID控制器输出的控制电流信号调整加热器或制冷器的功率。
    \item 温度传感器实时测量温度,并将反馈信号传递给PID控制器,形成闭环控制。
\end{enumerate}



\paragraph*{线性放大器控制}

使用线性放大器(如晶体管或运算放大器)来调节加热或制冷元件的功率输出。这种方法可以提供更平滑的控制信号,减少功率损耗。

\begin{itemize}
    \item \textbf{线性放大器}:放大控制信号,并驱动加热器或制冷器。
    \item \textbf{温度传感器}:测量当前温度,并将温度信号反馈给控制器。
    \item \textbf{PID控制器}:根据温度反馈信号调整输出信号,实现温度控制。
\end{itemize}

实现步骤:

\begin{enumerate}
    \item 将温度传感器连接到PID控制器,以测量当前温度。
    \item PID控制器根据设定的目标温度计算误差,并输出一个相应的控制信号。
    \item 线性放大器根据PID控制器输出的控制信号调整加热器或制冷器的功率。
    \item 温度传感器实时测量温度,并将反馈信号传递给PID控制器,形成闭环控制。
\end{enumerate}


\paragraph*{总结}

使用连续信号控温是一种有效的温度控制方法,可以通过模拟电压控制、模拟电流控制和线性放大器控制等方式实现。连续信号控温具有平滑的输出信号和较低的功率损耗,适用于各种精密温度控制应用。通过合理设置PID参数和反馈机制,可以实现高效稳定的温度控制。







\clearpage
\begin{table}
	\renewcommand\arraystretch{1.7}
	\centering
	\begin{tabularx}{\textwidth}{|X|X|X|X|}
	\hline
	专业:& 物理学 &年级:& 2022级 \\
	\hline
	姓名:& 戴鹏辉\&喻欣宸 & 学号:& 22344016\&22344014 \\
	\hline
	室温:& 26℃ & 实验地点: & A515 \\
	\hline
	学生签名:& & 评分: &\\
	\hline
	实验时间:& 2024/5/24 & 教师签名:&\\
	\hline
	\end{tabularx}
\end{table}

\section{BD4 \quad TEC 半导体控温实验 \quad\heiti 实验记录}
\subsection{实验内容和步骤}

	\subsubsection{连接电路和测试}

        \begin{figure}[htbp]
            \centering
            \includegraphics[width=0.6\textwidth]{graph2-1.png}
            \caption{控温实验接线示意图}
            \label{fig:fig2-1}
        \end{figure}

        \begin{figure}[htbp]
            \centering
            \includegraphics[width=0.6\textwidth, angle=90]{graph2-1a.jpg} % 将图片逆时针旋转90度
            \caption{控温实验接线实物图}
            \label{fig:fig2-1a}
        \end{figure}

        \begin{enumerate}
            \item \textbf{测量和确认热敏电阻工作状态}
            \begin{enumerate}
                \item 使用万用表测量实验装置的PT100热敏电阻的阻值,确认温度传感器正常工作。
                \item 使用线缆(香蕉插头和鳄鱼夹)将PT100连接到myDAQ的电阻测量端口,利用myDAQ的数字万用表功能进行测量。
            \end{enumerate}
            
            \item \textbf{完成IBT-2电机驱动器与控温实验装置、myDAQ及直流稳压电源的连接}
            \begin{enumerate}
                \item \textbf{TEC连接}
                \begin{enumerate}
                    \item 将TEC的红色线接到IBT-2电机驱动器的M+端,黑色线接到M-端。
                \end{enumerate}
                
                \item \textbf{直流稳压电源连接}
                \begin{enumerate}
                    \item 使用粗导线连接直流稳压电源和IBT-2电机驱动器,注意正负极。选择直流电源通道1,设置电压为12V,电流为2A(运行控制程序后再开启电源输出)。
                \end{enumerate}
                
                \item \textbf{数字端口连接}
                \begin{enumerate}
                    \item 使用杜邦线将IBT-2电机驱动器的数字端口与myDAQ按图示连接。可以用线缆颜色区分信号,例如红色表示加热,蓝色表示制冷等。
                    \item 数字端3和4可以先接到面包板上,短接后再通过杜邦线连接到myDAQ。
                \end{enumerate}
            \end{enumerate}
            
            \item \textbf{检查电脑与myDAQ的通信}
            \begin{enumerate}
                \item 在电脑上双击“NI MAX”图标,检查“设备和接口”栏下分配给myDAQ的设备名称,确保该名称与控温程序中调用的名称一致。如果不一致,需要重命名。
            \end{enumerate}
        \end{enumerate}


	\subsubsection{使用 Labview 程序进行控温实验}

        使用老师所提供的Labview控温程序,后面板如\cref{fig:fig-code}:  

        \begin{figure}[htbp]
            \centering
            \includegraphics[width=0.6\textwidth]{code.jpg} 
            \caption{Labview程序后面板}
            \label{fig:fig-code}
        \end{figure}
    


	\subsubsection{实验三 \quad 自行设计控制程序,利用提供的温度变送器进行温度采集,通过 myDAQ的模拟电压输入端进行电压采集,开展控温实验,并附前面板和程序框图的截图。将控温的结果和实验内容二(利用 myDAQ 数字万用表进行温度采集)的结果进行对比。}

    选。








\subsection{实验数据记录}

	见\cref{fig:data}

	\begin{figure}[htbp]
		\centering
		% \subfloat[原始数据1]
		{\includegraphics[width=0.35\textwidth]{Data1.jpg}\label{fig:data1}}
		\quad

		\caption{原始数据}
		\label{fig:data}
	\end{figure}




\subsection{实验过程中遇到的问题记录}

\begin{enumerate}
	\item 
\end{enumerate}
	

\clearpage
\begin{table}
	\renewcommand\arraystretch{1.7}
	\begin{tabularx}{\textwidth}{|X|X|X|X|}
	\hline
	专业:& 物理学 &年级:& 2022级\\
	\hline
	姓名: & 戴鹏辉\&喻欣宸 & 学号:& 22344016\&22344014\\
	\hline
    日期:& 2024/5/12 & 评分: &\\
	\hline
	\end{tabularx}
\end{table}

\section{BD4 \quad TEC 半导体控温实验 \quad\heiti 分析与讨论}

\subsection{实验数据分析}


\subsubsection{控温结果作图展示}
    
    \begin{enumerate}
        \item 控温时域图:
        
            图像如\cref{fig:PID_1a}、\cref{fig:PID_2a}所示。
        
            \begin{figure}[htbp]
                \centering
                \subfloat[$K_u=600$]
                {\includegraphics[width=1\textwidth]{TimeP_1.png}\label{fig:TimeP_1}}
                \quad
                \subfloat[$K_u=300 \quad T_i=0.198min$]
                {\includegraphics[width=1\textwidth]{TimePI_1.png}\label{fig:TimePI_1}}
                \quad
                \subfloat[$K_u=300 \quad T_i=0.25min \quad T_d=0.025min$]
                {\includegraphics[width=1\textwidth]{TimePID_1.png}\label{fig:TimePID_1}}
                \quad
        
                \caption{第一组PID参数控温时域图像,设置控温温度20℃}
                \label{fig:PID_1a}
            \end{figure}

            \begin{figure}[htbp]
                \centering
                \subfloat[$K_u=300$]
                {\includegraphics[width=1\textwidth]{TimeP_2.png}\label{fig:TimeP_2}}
                \quad
                \subfloat[$K_u=300 \quad T_i=0.067min$]
                {\includegraphics[width=1\textwidth]{TimePI_2.png}\label{fig:TimePI_2}}
                \quad
                \subfloat[$K_u=300 \quad T_i=0.067min \quad T_d=0.025min$]
                {\includegraphics[width=1\textwidth]{TimePID_2.png}\label{fig:TimePID_2}}
                \quad
        
                \caption{第二组PID参数控温时域图像,设置控温温度35℃}
                \label{fig:PID_2a}
            \end{figure}





        \item 控温频域图:
        
            使用对数坐标,图像如\cref{fig:PID_1b}、\cref{fig:PID_2b}所示。

            \begin{figure}[htbp]
                \centering
                \subfloat[$K_u=600$]
                {\includegraphics[width=1\textwidth]{PSD_P_1.png}\label{fig:PSD_P_1}}
                \quad
                \subfloat[$K_u=300 \quad T_i=0.198min$]
                {\includegraphics[width=1\textwidth]{PSD_PI_1.png}\label{fig:PSD_PI_1}}
                \quad
                \subfloat[$K_u=300 \quad T_i=0.25min \quad T_d=0.025min$]
                {\includegraphics[width=1\textwidth]{PSD_PID_1.png}\label{fig:PSD_PID_1}}
                \quad
        
                \caption{第一组PID参数控温频域图像,设置控温温度20℃}
                \label{fig:PID_1b}
            \end{figure}


            \begin{figure}[htbp]
                \centering
                \subfloat[$K_u=300$]
                {\includegraphics[width=1\textwidth]{PSD_P_2.png}\label{fig:PSD_P_2}}
                \quad
                \subfloat[$K_u=300 \quad T_i=0.067min$]
                {\includegraphics[width=1\textwidth]{PSD_PI_2.png}\label{fig:PSD_PI_2}}
                \quad
                \subfloat[$K_u=300 \quad T_i=0.067min \quad T_d=0.025min$]
                {\includegraphics[width=1\textwidth]{PSD_PID_2.png}\label{fig:PSD_PID_2}}
                \quad
        
                \caption{第二组PID参数控温频域图像,设置控温温度35℃}
                \label{fig:PID_2b}
            \end{figure}

            由于使用正常比例坐标可能控温效果更直观,故还使用了正常比例坐标,所作图像如\cref{fig:PID_Normal_1c}、\cref{fig:PID_Normal_2c}:

            \begin{figure}[htbp]
                \centering
                \subfloat[$K_u=600$]
                {\includegraphics[width=1\textwidth]{PSD_P_Normal_1.png}\label{fig:PSD_P_Normal_1}}
                \quad
                \subfloat[$K_u=300 \quad T_i=0.198min$]
                {\includegraphics[width=1\textwidth]{PSD_PI_Normal_1.png}\label{fig:PSD_PI_Normal_1}}
                \quad
                \subfloat[$K_u=300 \quad T_i=0.25min \quad T_d=0.025min$]
                {\includegraphics[width=1\textwidth]{PSD_PID_Normal_1.png}\label{fig:PSD_PID_Normal_1}}
                \quad
        
                \caption{第一组PID参数控温频域图像,设置控温温度20℃}
                \label{fig:PID_Normal_1c}
            \end{figure}

            \begin{figure}[htbp]
                \centering
                \subfloat[$K_u=300$]
                {\includegraphics[width=1\textwidth]{PSD_P_Normal_2.png}\label{fig:PSD_P_Normal_1}}
                \quad
                \subfloat[$K_u=300 \quad T_i=0.067min$]
                {\includegraphics[width=1\textwidth]{PSD_PI_Normal_2.png}\label{fig:PSD_PI_Normal_2}}
                \quad
                \subfloat[$K_u=300 \quad T_i=0.067min \quad T_d=0.025min$]
                {\includegraphics[width=1\textwidth]{PSD_PID_Normal_2.png}\label{fig:PSD_PID_Normal_2}}
                \quad
        
                \caption{第二组PID参数控温频域图像,设置控温温度35℃}
                \label{fig:PID_Normal_2c}
            \end{figure}

            
    \end{enumerate}



\subsubsection{分析和讨论}

    \begin{enumerate}
        \item 时域分析:

            对不同的PID控温参数下的控温结果进行标准差分析,结果如\cref{tbl:table1}:


            
            \begin{table}[htbp]
                \centering
                \begin{tblr}{
                cells = {c},
                cell{1}{2} = {c=3}{},
                cell{1}{5} = {c=3}{},
                vline{1-3,6,8} = {1}{},
                vline{1-2,5,8} = {2-4}{},
                hline{1,3,5} = {-}{},
                hline{2} = {2-7}{},
                }
                    & 第一组   &       &       & 第二组   &       &       \\
                    & P     & PI    & PID   & P     & PI    & PID   \\
                平均值/℃ & 20.12 & 20.13 & 20.18 & 35.12 & 35.15 & 35.14 \\
                标准差/℃ & 0.25  & 0.21  & 0.23  & 0.29  & 0.29  & 0.31  
                \end{tblr}
                \caption{控温结果和标准差分析}
                \label{tbl:table1}
            \end{table}

            
            \begin{enumerate}
                \item \textbf{第一组}:
                \begin{enumerate}
                    \item \textbf{PI控制}:温度波动最小,控温效果最佳(标准差0.21°C)。
                    \item \textbf{PID控制}:温度波动较小(标准差0.23°C),控温效果次佳。
                    \item \textbf{P控制}:温度波动最大(标准差0.25°C),控温效果最差。
                \end{enumerate}
            
                \item \textbf{第二组}:
                \begin{enumerate}
                    \item \textbf{P和PI控制}:控温效果相同(标准差均为0.29°C),温度波动较小。
                    \item \textbf{PID控制}:温度波动略大(标准差0.31°C),控温效果稍差。
                \end{enumerate}
            
                \item 总体来看,PI控制方式在第一组数据中表现最佳,在第二组数据中与P控制效果相同,均优于PID控制。因此,根据具体情况选择PI或PID控制方式可以实现较好的控温效果。
            \end{enumerate}

        \item 频域分析:
        
            通过对比\cref{fig:PID_1b}和\cref{fig:PID_2b},可以对不同PID控温参数下的控温效果进行分析。功率谱密度(PSD)图显示了系统在不同频率下的能量分布情况,反映了控温系统的稳定性和响应特性。低频和高频段功率谱密度不同的纵坐标值有不同的物理意义:

            \begin{enumerate}
                \item \textbf{低频段(10$^{-2}$ Hz)}:
                \begin{itemize}
                    \item 低频段的高能量分布通常表示系统对低频扰动的响应较差,存在较大的温度波动。这些低频扰动通常与环境温度变化、系统的慢速响应特性等有关。
                    \item 低频段能量分布较低则表示系统能够较好地抑制低频扰动,温度波动较小,控温效果较好。
                \end{itemize}
            
                \item \textbf{高频段(10$^{-1}$ Hz)}:
                \begin{itemize}
                    \item 高频段的高能量分布通常表示系统对高频扰动的响应较差,存在较大的温度波动。这些高频扰动通常与传感器噪声、控制器高频响应等有关。
                    \item 高频段能量分布较低则表示系统能够较好地抑制高频扰动,温度波动较小,控温效果较好。
                \end{itemize}
            \end{enumerate}
        
            以下是具体的分析结果:
        
        \begin{enumerate}
            \item \textbf{第一组(设定控温温度20°C)}
            \begin{enumerate}
                \item \cref{fig:PSD_P_1} \( K_u = 600 \):
                \begin{itemize}
                    \item PSD在低频段(10$^{-2}$ Hz)和高频段(10$^{-1}$ Hz)均有较高的能量分布,显示出系统在这两个频段内的温度波动较大,控温效果较差。
                \end{itemize}
                
                \item \cref{fig:PSD_PI_1} \( K_u = 300, T_i = 0.198 \text{ min} \):
                \begin{itemize}
                    \item PSD在低频段能量分布较低,显示出系统对低频扰动的良好抑制效果,温度波动较小。
                    \item 高频段有一些能量波动,但整体波动较为平稳,控温效果较好。
                \end{itemize}
                
                \item \cref{fig:PSD_PID_1} \( K_u = 300, T_i = 0.25 \text{ min}, T_d = 0.025 \text{ min} \):
                \begin{itemize}
                    \item PSD在低频段的能量分布较低,中频和高频段的能量分布较为均匀,整体波动较小,显示出系统的稳定性较好,控温效果最佳。
                \end{itemize}
            \end{enumerate}
        
            \item \textbf{第二组(设定控温温度35°C)}
            \begin{enumerate}
                \item \cref{fig:PSD_P_2} \( K_u = 300 \):
                \begin{itemize}
                    \item PSD在低频段(10$^{-2}$ Hz)有较高的能量分布,显示出系统在低频段内的温度波动较大,控温效果较差。
                    \item 高频段能量波动较大,显示出系统对高频扰动的响应较差。
                \end{itemize}
                
                \item \cref{fig:PSD_PI_2} \( K_u = 300, T_i = 0.067 \text{ min} \):
                \begin{itemize}
                    \item PSD在低频段和中频段的能量分布较低,显示出系统对低频和中频扰动的良好抑制效果,温度波动较小。
                    \item 高频段能量分布较高,但整体波动较为平稳,控温效果较好。
                \end{itemize}
                
                \item \cref{fig:PSD_PID_2} \( K_u = 300, T_i = 0.067 \text{ min}, T_d = 0.025 \text{ min} \):
                \begin{itemize}
                    \item PSD在低频段和中频段的能量分布均较低,高频段有少量能量波动,显示出系统的稳定性较好,控温效果最佳。
                \end{itemize}
            \end{enumerate}
        
            \item \textbf{总体分析}
            \begin{enumerate}
                \item 第一组:
                \begin{itemize}
                    \item \( K_u = 300, T_i = 0.25 \text{ min}, T_d = 0.025 \text{ min} \)(\cref{fig:PSD_PID_1})的控温效果最好,整体能量分布较为均匀,温度波动较小。
                    \item \( K_u = 600 \)(\cref{fig:PSD_P_1})的控温效果较差,能量分布不均匀,温度波动较大。
                \end{itemize}
                
                \item 第二组:
                \begin{itemize}
                    \item \( K_u = 300, T_i = 0.067 \text{ min}, T_d = 0.025 \text{ min} \)(\cref{fig:PSD_PID_2})的控温效果最好,低频和中频段的能量分布较低,整体波动最小。
                    \item \( K_u = 300 \)(\cref{fig:PSD_P_2})的控温效果较差,高频段的能量波动较大。
                \end{itemize}
            \end{enumerate}
        \end{enumerate}
        
        综上所述,添加积分和微分控制参数(PI和PID控制)能够有效降低温度波动,改善系统的稳定性。在实际应用中,选择合适的PID参数可以显著提升控温效果。
        

    \end{enumerate}





		

\subsection{实验后思考题}	
	
    \begin{question}
        影响控温精度和稳定度的因素都有哪些, 如何进行改进?
    \end{question}

    影响控温精度和稳定度的因素主要包括以下几个方面:

        \begin{enumerate}
            \item TEC的性能和选型:TEC的性能直接影响着控温系统的稳定性和精度。选择合适功率、制冷能力、温度范围等特性的TEC对于控温系统的性能至关重要。
            \item 环境温度和散热:环境温度的变化以及散热效率会影响到TEC的工作状态,从而影响控温精度和稳定度。良好的散热设计和环境温度控制可以提高控温系统的稳定性。
            \item 电源质量和稳定性:TEC控温系统的电源质量和稳定性对系统的工作效果有很大影响。电源的电压稳定性、纹波、噪声等因素都会对控温系统产生影响。
            \item 传感器的准确性:控温系统所采用的传感器的准确性和灵敏度对于控温精度至关重要。选择高精度、高灵敏度的传感器可以提高控温系统的精度和稳定性。
            \item 控制算法:控制算法的设计对于控温系统的性能影响很大。采用合适的PID控制算法,并根据实际情况进行参数调整,可以提高控温系统的稳定性和精度。
        \end{enumerate}
    

    可改进措施如下:
    
        \begin{enumerate}
            \item 优化散热设计:改善散热系统,增加散热面积,提高散热效率,从而降低环境温度对于TEC控温系统的影响。
            \item 选择优质电源:选择稳定性好、纹波小的电源,并根据TEC的工作电压要求选用合适的电源,确保电源的稳定性和质量。
            \item 优化传感器:选择准确性高、响应速度快的传感器,并根据实际情况进行校准,确保传感器的准确性和稳定性。
            \item 改进控制算法:针对不同的实验需求,优化控制算法的设计,采用合适的PID参数调整策略,提高控温系统的性能。
            \item 定期维护和校准:定期对控温系统进行维护和校准,检查各个部件的工作状态,确保系统的稳定性和精度。
        \end{enumerate}


    综上所述,通过对TEC控温系统各个方面进行综合优化和改进,可以提高控温系统的精度和稳定性,从而更好地满足实验需求。
    


    \begin{question}
        本实验中直流电源的电流输出建议设置为 2A, 为了减小温度的波动, 电流设置应该增大还是减小, 为什么?
    \end{question}

    在TEC控温实验中,直流电源的电流输出建议设置为2A。要减小温度的波动,电流设置应该适当增大。
    
    增大电流可以提高TEC的制冷能力,从而在外部环境温度变化或负载变化时更快地响应并稳定温度。当温度波动较大时,增大电流输出可以增加TEC的制冷量,使得系统更能够快速地回到设定温度。

    需要注意的是,增大电流也会增加TEC的功耗和热量产生,因此需要注意系统的散热设计和电源能力是否能够支持增大电
    流。同时,增大电流也可能会加速TEC的寿命损耗,因此需要在提高温度稳定性与设备寿命之间进行权衡。

    总的来说,适当增大电流输出可以在一定程度上减小温度的波动,但需要考虑到系统的散热能力、电源能力和设备寿命等因素。



    \begin{question}
        实验是否达到了控温的目的, 还有哪些可以改进的地方?
    \end{question}

    实验达到了控温的目的,但对于温度的控制还可以更加精准,可以改进的地方:

        \begin{enumerate}
            \item 优化控制算法:根据实验需求和系统特性,优化控制算法,提高系统的温度控制精度和稳定性。
            \item 改进散热设计:优化散热系统,提高散热效率,以应对系统工作时产生的热量,确保系统在长时间运行时的稳定性。
            \item 选择更好的传感器:选择更加准确、响应速度更快的温度传感器,提高系统对温度变化的检测和响应能力。
            \item 定期维护和校准:定期对控温系统进行维护和校准,确保各个部件的工作状态良好,保证系统的性能稳定。
            \item 结合实验需求进行优化:根据具体的实验需求,结合以上改进措施进行综合优化,以达到更好的控温效果。
        \end{enumerate}






    \begin{question}
        温度控制程序是否工作正常, 是否还有优化的地方?
    \end{question}

    温度控制程序工作正常。设定目标温度后,经过一段时间,实际温度可以较好地与目标温度吻合,控温程序能够相对准确地控制温度。达到设定温度后,实际温度大致在1℃的范围内波动,波动在正常范围内。

    可以优化的地方:

        \begin{enumerate}
            \item 控制算法优化:根据实际情况优化控制算法,例如调整PID参数、改进控制逻辑等,以提高温度控制的精度和稳定性。
            \item 温度传感器校准:定期对温度传感器进行校准,确保获取的温度数据准确可靠。
            \item 系统响应速度改进:优化系统设计或者增加控制策略,以提高系统的响应速度,使得系统能够更快地达到设定温度。
            \item 故障诊断与预防:加入故障诊断机制,及时发现并解决系统故障,以保证系统的稳定性和可靠性。
            \item 能耗优化:在保证控温效果的前提下,优化系统设计和控制策略,减少能耗,提高系统的能源利用效率。
        \end{enumerate}




        \begin{figure}[htbp]
            \centering
            \includegraphics[width=0.6\textwidth]{table.jpg}
            \caption{整理后实验桌照片}
            \label{fig:table}
        \end{figure}




\end{document}
