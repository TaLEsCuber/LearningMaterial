%!TEX program = xelatex
\documentclass[dvipsnames, svgnames,a4paper,11pt]{article}
% ----------------------------------------------------
%   中山大学物理与天文学院本科实验报告模板
%   作者:Huanyu Shi,2019级
%   知乎:https://www.zhihu.com/people/za-ran-zhu-fu-liu-xing
%   Github:https://github.com/huanyushi/SYSU-SPA-Labreport-Template
%   Last update : 2023.4.10
% ----------------------------------------------------

\input{Settings} % 导入模板的相关设置
\usepackage{lipsum}
\usepackage{enumitem}
\usepackage{tabularray}  %绘制表格时可以更加方便添加框线
\setlist[enumerate]{label=\textup{(\arabic*)}}



%---------------------------------------------------------------------
%	正文
%---------------------------------------------------------------------

\begin{document}


\begin{table}
	\renewcommand\arraystretch{1.7}
	\begin{tabularx}{\textwidth}{
		|X|X|X|X
		|X|X|X|X|}
	\hline
	\multicolumn{2}{|c|}{预习报告}&\multicolumn{2}{|c|}{实验记录}&\multicolumn{2}{|c|}{分析讨论}&\multicolumn{2}{|c|}{总成绩}\\
	\hline
	\LARGE25 & & \LARGE30 & & \LARGE25 & & \LARGE80 & \\
	\hline
	\end{tabularx}
\end{table}


\begin{table}
	\renewcommand\arraystretch{1.7}
	\begin{tabularx}{\textwidth}{|X|X|X|X|}
	\hline
	专业:& 物理学 &年级:& 2022级\\
	\hline
	姓名:& 戴鹏辉\&喻欣宸  & 学号: & 2344016\&22344014 \\
	\hline
	日期:& 2024/5/20 & 教师签名:& \\
	\hline
	\end{tabularx}
\end{table}

\begin{center}
	\LARGE BD4  \quad TEC 半导体控温实验
\end{center}

\textbf{【实验报告注意事项】}
\begin{enumerate}
	\item 实验报告由三部分组成:
	\begin{enumerate}
		\item 预习报告:(提前一周)认真研读\underline{\textbf{实验讲义}},弄清实验原理;实验所需的仪器设备、用具及其使用(强烈建议到实验室预习),完成课前预习思考题;了解实验需要测量的物理量,并根据要求提前准备实验记录表格(第一循环实验已由教师提供模板,可以打印)。预习成绩低于10分(共20分)者不能做实验。
	    \item 实验记录:认真、客观记录实验条件、实验过程中的现象以及数据。实验记录请用珠笔或者钢笔书写并签名(\textcolor{red}{\textbf{用铅笔记录的被认为无效}})。\textcolor{red}{\textbf{保持原始记录,包括写错删除部分,如因误记需要修改记录,必须按规范修改。}}(不得输入电脑打印,但可扫描手记后打印扫描件);离开前请实验教师检查记录并签名。
	    \item 分析讨论:处理实验原始数据(学习仪器使用类型的实验除外),对数据的可靠性和合理性进行分析;按规范呈现数据和结果(图、表),包括数据、图表按顺序编号及其引用;分析物理现象(含回答实验思考题,写出问题思考过程,必要时按规范引用数据);最后得出结论。
	\end{enumerate}
	\textbf{实验报告就是将预习报告、实验记录、和数据处理与分析合起来,加上本页封面。}
	\item 每次完成实验后的一周内交\textbf{实验报告}(特殊情况不能超过两周)。
	\item 实验报告注意事项
		\begin{enumerate}[label=\roman*.]
			\item 本实验电路比较复杂, 需要耐心连接, 认真检查。
			\item 程序编写和调试过程中, 遇到问题及时查看 LabVIEW 帮助, 留意各个器件时刻处于可控状态。
			\item 直流稳压电源的输出,要根据实验过程及时地手动打开或关闭,不要在程序没有正常运行的情况下,让直流稳压电源有持续的输出。
			\item 注意根据信号电流大小选取合适直接的导线,杜邦线一般只用于小电流和信号传输,电流大时则需要选用直径大的导线。
		\end{enumerate}
\end{enumerate}


\clearpage
\tableofcontents
\clearpage

\setcounter{section}{0}
\section{BD4 \quad TEC 半导体控温实验 \quad\heiti 预习报告}
	
\subsection{实验目的}
\begin{enumerate}
	\item 了解TEC半导体制冷的原理
	\item PID反馈控制的参数调节。
	\item 基于LabVIEW和myDAQ设计和搭建温度采集和控制系统。
	
	
\end{enumerate}

\subsection{仪器用具}
\begin{table}[htbp]
	\centering
	\renewcommand\arraystretch{1.6}
	% \setlength{\tabcolsep}{10mm}
	\begin{tabular}{p{0.05\textwidth}|p{0.20\textwidth}|p{0.05\textwidth}|p{0.5\textwidth}}
		\hline
		编号& 仪器用具名称 			& 数量 		&  主要参数(型号,测量范围,测量精度等) \\
		\hline
		1	&	计算机 				&1 			& Windows 系统, 安装 LabVIEW 程序\\

		2	&	myDAQ 				&1 			& 		 \\
		
		3	&	直流稳压电源 		& 1 		&	DP832 \\
		
		4	&	控温实验平台		&1 			&   \\
		
		5	&	$IBT\_2$电机驱动器	&	1 		& 12V供电,最大电流2A \\

		6	&	温度变送器			&	1 		&		24V 供电, 测温范围 0-100℃, 输出电压 0-10V \\

		7	&	电路元器件			&	1套 	& 面包板, 连接线等 \\
		\hline
	\end{tabular}
\end{table}

\subsection{原理概述}

    \subsubsection{PID反馈控制原理}

        PID控制器是一种广泛应用于工业控制系统的反馈控制器,PID代表比例(Proportional),积分(Integral)和微分(Derivative)。PID控制器通过调节这三个参数,使系统输出跟随设定值,达到最佳控制效果。

        \begin{figure}[htbp]
            \centering
            \includegraphics[width=0.6\textwidth]{graph1-1.png}
            \caption{PID控制原理图}
            \label{fig:fig1-1}
        \end{figure}

        \paragraph{比例控制(P)}

            比例控制器根据当前误差(设定值与实际值之间的差值)进行调节。比例控制器的输出与误差成正比,即
            \[ u(t) = K_p e(t) \]
            其中,$u(t)$为控制输出,$K_p$为比例增益,$e(t)$为误差。

        \paragraph{积分控制(I)}

        积分控制器根据误差的累积进行调节,主要用于消除稳态误差。积分控制器的输出与误差的积分成正比,即
        \[ u(t) = K_i \int_0^t e(\tau) d\tau \]
        其中,$K_i$为积分增益。

        \paragraph{微分控制(D)}

        微分控制器根据误差的变化率进行调节,主要用于预见误差的变化趋势。微分控制器的输出与误差的导数成正比,即
        \[ u(t) = K_d \frac{de(t)}{dt} \]
        其中,$K_d$为微分增益。

        \paragraph{PID控制器的总输出}

        综合比例、积分和微分控制的作用,PID控制器的总输出为
        \[ u(t) = K_p e(t) + K_i \int_0^t e(\tau) d\tau + K_d \frac{de(t)}{dt} \]


        \paragraph*{调节 PID 参数的一般方法}

            \begin{enumerate}
                \item \textbf{初始设置:}
                \begin{itemize}
                    \item 将积分增益 $I$ 和微分增益 $D$ 设置为零。
                    \item 增大比例增益 $P$,直到输出信号(或被控物理量)出现振荡。然后将比例增益值减小至目前数值的一半左右。
                \end{itemize}
                
                \item \textbf{调整积分增益:}
                \begin{itemize}
                    \item 设置好比例增益后,增大积分增益 $I$,直到任何漂移都根据用户系统所适用的时间尺度进行校正。如果积分增益 $I$ 过大,将会观察到设定值的严重超调和系统不稳定现象。
                \end{itemize}
                
                \item \textbf{调整微分增益:}
                \begin{itemize}
                    \item 积分增益 $I$ 设定好后,开始增大微分增益 $D$。微分增益 $D$ 可以减小超调,并阻碍系统振荡,使其快速稳定达到目标值。如果微分增益 $D$ 过大,会观察到大幅度的超调(因为系统响应过于缓慢)。
                \end{itemize}
            \end{enumerate}
        
            通过调节 PID 增益,可以使系统达到最佳工作状态,实现快速响应,并有效抑制目标值附近的振荡。
        
            专业的 PID 调节方法中,齐格勒-尼柯尔斯 (Ziegler-Nichols) 方法应用较广。具体步骤如下:
        
            \begin{enumerate}
                \item 将积分增益 $I$ 和微分增益 $D$ 设置为零。
                \item 增大比例增益 $P$,直到系统开始出现振荡,测出此时的增益 $K_u$ 和振荡周期 $T_u$。
                \item 根据不同控制电路类型的增益,参照下表设置 $P$、$I$ 和 $D$ 值。
            \end{enumerate}
        
            PID 控制器的输出可以表示为:
            \[
            u(t) = K_p \left( e(t) + \frac{1}{T_i} \int_0^t e(\tau) \, d\tau + T_d \frac{d e(t)}{d t} \right)
            \]
        
            齐格勒-尼柯尔斯方法的一些典型设置如下:
            
            \[
            \begin{array}{|c|c|c|c|}
            \hline
            \text{控制类型} & K_p & T_i & T_d \\
            \hline
            \text{P} & 0.5 K_u & - & - \\
            \text{PI} & 0.45 K_u & \frac{T_u}{1.2} & - \\
            \text{PID} & 0.6 K_u & \frac{T_u}{2} & \frac{T_u}{8} \\
            \hline
            \end{array}
            \]
        
            这套方法可以作为设定 PID 值的指南,用以帮助实现系统的最佳控制效果。
        


    \subsubsection{半导体控温原理}

        半导体控温基于热电效应,主要包括帕尔帖效应(Peltier Effect)和塞贝克效应(Seebeck Effect)。

        \paragraph*{帕尔帖效应}

        帕尔帖效应描述了当电流通过两个不同类型的半导体材料的接合处时,会吸收或释放热量。具体来说,当电流从p型半导体流向n型半导体时,接合处会吸收热量,形成制冷效果;反之,接合处会释放热量,形成加热效果。

        \paragraph*{塞贝克效应}

        塞贝克效应描述了当两个不同类型的半导体材料的接合处存在温差时,会产生电动势。这种效应用于温度测量和热电发电。

        \paragraph*{两者之间的关系}

        帕尔帖效应和塞贝克效应是热电效应的两个方面。帕尔帖效应用于温度控制,通过电流驱动实现制冷或加热;塞贝克效应用于温度测量或能量回收,通过温差产生电动势。这两种效应都依赖于材料的热电性能,是热电技术的基础。

    \subsubsection{傅里叶变换基本原理}

        傅里叶变换是一种将时间域信号转换为频率域信号的数学变换。它基于任何周期信号都可以表示为一组正弦波和余弦波的叠加。

        \paragraph*{傅里叶变换公式}

        傅里叶变换的数学表达式为
        \[ X(f) = \int_{-\infty}^{\infty} x(t) e^{-j 2 \pi f t} dt \]
        其中,$X(f)$为频率域表示,$x(t)$为时间域信号,$j$为虚数单位,$f$为频率。

        \paragraph*{逆傅里叶变换}

        逆傅里叶变换用于将频率域信号转换回时间域信号,其表达式为
        \[ x(t) = \int_{-\infty}^{\infty} X(f) e^{j 2 \pi f t} df \]

    \subsubsection*{功率谱密度或幅度谱密度基本原理}

    功率谱密度(Power Spectral Density, PSD)和幅度谱密度(Amplitude Spectral Density, ASD)用于描述信号在频率域的能量分布。

        \paragraph*{功率谱密度}

        功率谱密度表示每单位频率区间内的功率分布,其定义为
        \[ S_{xx}(f) = \lim_{T \to \infty} \frac{1}{T} \mathbb{E} \left[ \left| X_T(f) \right|^2 \right] \]
        其中,$X_T(f)$为信号$x(t)$在区间$[-T/2, T/2]$上的傅里叶变换,$\mathbb{E}$为期望值。

        \paragraph*{幅度谱密度}

        幅度谱密度表示每单位频率区间内的振幅分布,其定义为
        \[ A(f) = \sqrt{S_{xx}(f)} \]

        \paragraph*{应用}

        功率谱密度用于分析信号的频率成分和噪声特性,广泛应用于信号处理、通信系统和振动分析。幅度谱密度用于表征信号的振幅分布,常用于振动分析和频谱测量。







% \subsubsection{误差反馈控制:PID 简介}

% 	PID(Proportional-Integral-Derivative)控制是一种经典的反馈控制策略,广泛应用于工业控制系统中。PID 控制器通过调整控制变量来最小化系统的误差,使输出达到期望值。

% 	\paragraph*{比例控制 (P)}

% 		比例控制的作用是通过控制器的输出与误差成正比来减小误差。当误差较大时,比例控制会产生较大的控制信号。比例控制的公式为:
			
% 			\begin{equation}
% 				P = K_p \cdot e(t)
% 			\end{equation}
% 		其中,$K_p$ 是比例增益,$e(t)$ 是系统的瞬时误差。

% 	\paragraph*{积分控制 (I)}

% 	积分控制通过累积误差随时间的积分来消除长期的稳态误差,使系统达到精确的控制。积分控制的公式为:
		
% 		\begin{equation}
% 			I = K_i \cdot \int_0^t e(\tau) \, d\tau
% 		\end{equation}
	
% 		其中,$K_i$ 是积分增益。

% 	\paragraph*{微分控制 (D)}

% 	微分控制通过预测误差变化的趋势来提供预测性修正,以减少系统的超调和提高系统的响应速度。微分控制的公式为:
% 		\begin{equation}
% 			D = K_d \cdot \frac{d e(t)}{d t}
% 		\end{equation}

% 	其中,$K_d$ 是微分增益。

% 	PID 控制器的总输出可以表示为这三部分之和:

% 	\begin{equation}
% 		u(t) = K_p \cdot e(t) + K_i \cdot \int_0^t e(\tau) \, d\tau + K_d \cdot \frac{d e(t)}{d t}
% 	\end{equation}

% 	\subsubsection{调节 PID 参数的一般方法}

% 	调节 PID 参数是使控制系统达到最佳性能的关键步骤。以下是几种常见的调节方法:

% 	\paragraph*{Ziegler-Nichols 方法}

% 	通过设定 $K_i$ 和 $K_d$ 为零,逐渐增加 $K_p$ 直到系统达到临界振荡点,然后根据经验公式计算 $K_i$ 和 $K_d$ 的值。

% 	\paragraph*{手动调节}

% 	先调整 $K_p$,直到系统在无稳态误差下快速响应;再调整 $K_i$,以消除稳态误差;最后调整 $K_d$,以减少超调和振荡。

% 	\paragraph*{软件调节}

% 	使用软件工具自动调节 PID 参数,这种方法通常基于优化算法,如遗传算法、粒子群优化等。

% 	\subsubsection{基于虚拟仪器的 PID 控制}

% 	虚拟仪器是一种基于计算机的软件工具,通过模拟传统硬件仪器来实现数据采集、分析和控制。基于虚拟仪器的 PID 控制具有以下优点:

% 	\begin{itemize}
% 		\item \textbf{灵活性}:可以方便地调整和测试不同的控制算法和参数。
% 		\item \textbf{可视化}:通过图形化界面实时监测和调节控制参数,方便用户操作。
% 		\item \textbf{数据记录和分析}:可以记录系统响应数据并进行详细分析,以优化控制性能。
% 	\end{itemize}

% 	LabVIEW 是一个常用的虚拟仪器开发环境,通过其内置的 PID 控制工具包,用户可以轻松实现复杂的 PID 控制系统。

% 	\subsubsection{PWM 信号简介}

% 	PWM(Pulse Width Modulation,脉冲宽度调制)是一种通过调节脉冲宽度来控制输出功率的技术。PWM 信号主要应用在以下几个方面:

% 	\begin{itemize}
% 		\item \textbf{电机控制}:通过调节 PWM 信号的占空比来控制电机的转速和方向。
% 		\item \textbf{LED 调光}:通过改变 PWM 信号的占空比来调节 LED 的亮度。
% 		\item \textbf{电源控制}:在开关电源中使用 PWM 信号来控制输出电压。
% 	\end{itemize}

% 	PWM 信号的关键参数是空比(Duty Cycle),它定义了一个周期内脉冲高电平时间与总周期时间的比值。通过改变占空比,可以精确控制输出信号的平均电压和功率。

% 	PID 控制和 PWM 信号常结合使用,通过 PID 控制器产生 PWM 信号,进而控制执行器(如电机、加热器等)的输出,实现精确的闭环控制。

% 	\subsubsection{总结}

% 	PID 控制是一种重要的反馈控制策略,通过比例、积分和微分控制实现精确的系统控制。调节 PID 参数的方法有多种,基于虚拟仪器的 PID 控制提供了灵活性和可视化的优势,而 PWM 信号则广泛应用于各种电机和电源控制中。理解和应用这些控制技术可以大大提高系统的稳定性和性能。

% 	\subsubsection{半导体制冷器结构及工作原理简介}

% 	半导体制冷器,也称为热电制冷器(Thermoelectric Cooler, TEC),是一种基于帕尔帖效应(Peltier Effect)的制冷设备。其结构和工作原理如下:

% 	\paragraph*{结构}

% 	一个典型的半导体制冷器由多个热电偶元件组成,这些热电偶元件通常排列成两个相对的矩阵,分别连接在两个陶瓷基板上。每个热电偶元件由两种不同类型的半导体材料(n型和p型)组成,这些材料通过导电连接器交替排列。具体结构如下:

% 	\begin{itemize}
% 		\item \textbf{热电偶元件}:由n型和p型半导体材料组成。n型材料具有多余的电子,而p型材料具有空穴。
% 		\item \textbf{陶瓷基板}:用于支撑热电偶元件,并提供机械强度和电绝缘性。
% 		\item \textbf{导电连接器}:连接n型和p型材料,使得电流能够通过整个热电偶阵列。
% 	\end{itemize}

% 	\paragraph{工作原理}

% 	半导体制冷器的工作原理基于帕尔帖效应。当电流通过n型和p型半导体材料时,会发生热量传输,这一效应可以用于制冷和加热。具体工作原理如下:

% 	\begin{itemize}
% 		\item \textbf{帕尔帖效应}:当电流通过不同类型的半导体材料接口时,会吸收或释放热量。对于n型材料,当电流从金属流向半导体时,会吸收热量;对于p型材料,当电流从半导体流向金属时,也会吸收热量。
% 		\item \textbf{热传输}:当电流通过连接的n型和p型半导体材料时,一侧会吸收热量(制冷侧),另一侧会释放热量(加热侧)。通过调节电流的方向,可以控制制冷和加热的方向。
% 		\item \textbf{热泵效应}:半导体制冷器实际上是一个热泵,将热量从制冷侧转移到加热侧,从而达到制冷的目的。
% 	\end{itemize}

% 	\paragraph{应用}

% 	半导体制冷器具有结构简单、无运动部件、可逆性和高精度控温等优点,广泛应用于以下领域:

% 	\begin{itemize}
% 		\item \textbf{电子设备散热}:用于精密电子设备(如激光器、光电探测器等)的温度控制。
% 		\item \textbf{医疗设备}:用于便携式医疗设备(如血液分析仪、DNA扩增仪等)的制冷。
% 		\item \textbf{消费电子}:用于便携式冰箱、冷却杯垫等消费电子产品。
% 	\end{itemize}

% 	\subsubsection{总结}

% 	半导体制冷器是一种基于帕尔帖效应的制冷设备,通过电流控制热量传输,实现高效制冷。其简单的结构和高效的制冷能力使其在许多领域得到广泛应用。







% \subsubsection*{误差反馈控制:PID 简介}

% PID(Proportional-Integral-Derivative)控制是一种经典的反馈控制策略,广泛应用于工业控制系统中。PID 控制器通过调整控制变量来最小化系统的误差,使输出达到期望值。

% \paragraph*{比例控制 (P)}

% 比例控制的作用是通过控制器的输出与误差成正比来减小误差。当误差较大时,比例控制会产生较大的控制信号。比例控制的公式为:
% \begin{equation}
%     P = K_p \cdot e(t)
% \end{equation}
% 其中,$K_p$ 是比例增益,$e(t)$ 是系统的瞬时误差。

% \paragraph*{积分控制 (I)}

% 积分控制通过累积误差随时间的积分来消除长期的稳态误差,使系统达到精确的控制。积分控制的公式为:
% \begin{equation}
%     I = K_i \cdot \int_0^t e(\tau) \, d\tau
% \end{equation}
% 其中,$K_i$ 是积分增益。

% \paragraph*{微分控制 (D)}

% 微分控制通过预测误差变化的趋势来提供预测性修正,以减少系统的超调和提高系统的响应速度。微分控制的公式为:
% \begin{equation}
%     D = K_d \cdot \frac{d e(t)}{d t}
% \end{equation}
% 其中,$K_d$ 是微分增益。

% PID 控制器的总输出可以表示为这三部分之和:
% \begin{equation}
%     u(t) = K_p \cdot e(t) + K_i \cdot \int_0^t e(\tau) \, d\tau + K_d \cdot \frac{d e(t)}{d t}
% \end{equation}

% \subsubsection*{调节 PID 参数的一般方法}

% 调节 PID 参数是使控制系统达到最佳性能的关键步骤。以下是几种常见的调节方法:

% \paragraph*{Ziegler-Nichols 方法}

% 通过设定 $K_i$ 和 $K_d$ 为零,逐渐增加 $K_p$ 直到系统达到临界振荡点,然后根据经验公式计算 $K_i$ 和 $K_d$ 的值。

% \paragraph*{手动调节}

% 先调整 $K_p$,直到系统在无稳态误差下快速响应;再调整 $K_i$,以消除稳态误差;最后调整 $K_d$,以减少超调和振荡。

% \paragraph*{软件调节}

% 使用软件工具自动调节 PID 参数,这种方法通常基于优化算法,如遗传算法、粒子群优化等。

% \subsubsection*{基于虚拟仪器的 PID 控制}

% 虚拟仪器是一种基于计算机的软件工具,通过模拟传统硬件仪器来实现数据采集、分析和控制。基于虚拟仪器的 PID 控制具有以下优点:
% \begin{itemize}
%     \item \textbf{灵活性}:可以方便地调整和测试不同的控制算法和参数。
%     \item \textbf{可视化}:通过图形化界面实时监测和调节控制参数,方便用户操作。
%     \item \textbf{数据记录和分析}:可以记录系统响应数据并进行详细分析,以优化控制性能。
% \end{itemize}
% LabVIEW 是一个常用的虚拟仪器开发环境,通过其内置的 PID 控制工具包,用户可以轻松实现复杂的 PID 控制系统。

% \subsubsection*{PWM 信号简介}

% PWM(Pulse Width Modulation,脉冲宽度调制)是一种通过调节脉冲宽度来控制输出功率的技术。PWM 信号主要应用在以下几个方面:
% \begin{itemize}
%     \item \textbf{电机控制}:通过调节 PWM 信号的占空比来控制电机的转速和方向。
%     \item \textbf{LED 调光}:通过改变 PWM 信号的占空比来调节 LED 的亮度。
%     \item \textbf{电源控制}:在开关电源中使用 PWM 信号来控制输出电压。
% \end{itemize}
% PWM 信号的关键参数是占空比(Duty Cycle),它定义了一个周期内脉冲高电平时间与总周期时间的比值。通过改变占空比,可以精确控制输出信号的平均电压和功率。PID 控制和 PWM 信号常结合使用,通过 PID 控制器产生 PWM 信号,进而控制执行器(如电机、加热器等)的输出,实现精确的闭环控制。

% \subsubsection*{总结}

% PID 控制是一种重要的反馈控制策略,通过比例、积分和微分控制实现精确的系统控制。调节 PID 参数的方法有多种,基于虚拟仪器的 PID 控制提供了灵活性和可视化的优势,而 PWM 信号则广泛应用于各种电机和电源控制中。理解和应用这些控制技术可以大大提高系统的稳定性和性能。

% \subsubsection*{半导体控温原理}

% 半导体控温基于热电效应,主要包括帕尔帖效应(Peltier Effect)和塞贝克效应(Seebeck Effect)。

% \paragraph*{帕尔帖效应}

% 帕尔帖效应描述了当电流通过两个不同类型的半导体材料的接合处时,会吸收或释放热量。具体来说,当电流从p型半导体流向n型半导体时,接合处会吸收热量,形成制冷效果;反之,接合处会释放热量,形成加热效果。

% \paragraph*{塞贝克效应}

% 塞贝克效应描述了当两个不同类型的半导体材料的接合处存在温差时,会产生电动势。这种效应用于温度测量和热电发电。

% \paragraph*{两者之间的关系}

% 帕尔帖效应和塞贝克效应是热电效应的两个方面。帕尔帖效应用于温度控制,通过电流驱动实现制冷或加热;塞贝克效应用于温度测量或能量回收,通过温差产生电动势。这两种效应都依赖于材料的热电性能,是热电技术的基础。

% \subsubsection*{傅里叶变换基本原理}

% 傅里叶变换是一种将时间域信号转换为频率域信号的数学变换。它基于任何周期信号都可以表示为一组正弦波和余弦波的叠加。

% \paragraph*{傅里叶变换公式}

% 傅里叶变换的数学表达式为:
% \begin{equation}
%     X(f) = \int_{-\infty}^{\infty} x(t) e^{-j 2 \pi f t} dt
% \end{equation}
% 其中,$X(f)$为频率域表示,$x(t)$为时间域信号,$j$为虚数单位,$f$为频率。

% \paragraph*{逆傅里叶变换}

% 逆傅里叶变换用于将频率域信号转换回时间域信号,其表达式为:
% \begin{equation}
%     x(t) = \int_{-\infty}^{\infty} X(f) e^{j 2 \pi f t} df
% \end{equation}

% \subsubsection*{功率谱密度或幅度谱密度基本原理}

% 功率谱密度(Power Spectral Density, PSD)和幅度谱密度(Amplitude Spectral Density, ASD)用于描述信号在频率域的能量分布。

% \paragraph*{功率谱密度}

% 功率谱密度表示每单位频率区间内的功率分布,其定义为:
% \begin{equation}
%     S_{xx}(f) = \lim_{T \to \infty} \frac{1}{T} \mathbb{E} \left[ \left| X_T(f) \right|^2 \right]
% \end{equation}
% 其中,$X_T(f)$为信号$x(t)$在区间$[-T/2, T/2]$上的傅里叶变换,$\mathbb{E}$为期望值。

% \paragraph*{幅度谱密度}

% 幅度谱密度表示每单位频率区间内的振幅分布,其定义为:
% \begin{equation}
%     A(f) = \sqrt{S_{xx}(f)}
% \end{equation}

% \paragraph*{应用}

% 功率谱密度用于分析信号的频率成分和噪声特性,广泛应用于信号处理、通信系统和振动分析。幅度谱密度用于表征信号的振幅分布,常用于地震学和音频分析。

% \subsubsection*{半导体制冷器结构及工作原理简介}

% 半导体制冷器,也称为热电制冷器(Thermoelectric Cooler, TEC),是一种基于帕尔帖效应(Peltier Effect)的制冷设备。其结构和工作原理如下:

% \paragraph*{结构}

% 一个典型的半导体制冷器由多个热电偶元件组成,这些热电偶元件通常排列成两个相对的矩阵,分别连接在两个陶瓷基板上。每个热电偶元件由两种不同类型的半导体材料(n型和p型)组成,这些材料通过导电连接器交替排列。具体结构如下:

% \begin{itemize}
%     \item \textbf{热电偶元件}:由n型和p型半导体材料组成。n型材料具有多余的电子,而p型材料具有空穴。
%     \item \textbf{陶瓷基板}:用于支撑热电偶元件,并提供机械强度和电绝缘性。
%     \item \textbf{导电连接器}:连接n型和p型材料,使得电流能够通过整个热电偶阵列。
% \end{itemize}

% \paragraph{工作原理}

% 半导体制冷器的工作原理基于帕尔帖效应。当电流通过n型和p型半导体材料时,会发生热量传输,这一效应可以用于制冷和加热。具体工作原理如下:

% \begin{itemize}
%     \item \textbf{帕尔帖效应}:当电流通过不同类型的半导体材料接口时,会吸收或释放热量。对于n型材料,当电流从金属流向半导体时,会吸收热量;对于p型材料,当电流从半导体流向金属时,也会吸收热量。
%     \item \textbf{热传输}:当电流通过连接的n型和p型半导体材料时,一侧会吸收热量(制冷侧),另一侧会释放热量(加热侧)。通过调节电流的方向,可以控制制冷和加热的方向。
%     \item \textbf{热泵效应}:半导体制冷器实际上是一个热泵,将热量从制冷侧转移到加热侧,从而达到制冷的目的。
% \end{itemize}

% \paragraph{应用}

% 半导体制冷器具有结构简单、无运动部件、可逆性和高精度控温等优点,广泛应用于以下领域:

% \begin{itemize}
%     \item \textbf{电子设备散热}:用于精密电子设备(如激光器、光电探测器等)的温度控制。
%     \item \textbf{医疗设备}:用于便携式医疗设备(如血液分析仪、DNA扩增仪等)的制冷。
%     \item \textbf{消费电子}:用于便携式冰箱、冷却杯垫等消费电子产品。
% \end{itemize}

% \subsubsection*{总结}

% 半导体制冷器是一种基于帕尔帖效应的制冷设备,通过电流控制热量传输,实现高效制冷。其简单的结构和高效的制冷能力使其在许多领域得到广泛应用。




\clearpage

\subsection{实验前思考题}



% 思考题1
\begin{question}
	热敏电阻测温点位置不同, 是否会影响PID 参数的设定, 为什么?
\end{question}


热敏电阻测温点的位置确实会影响PID参数的设定,原因如下:

\paragraph*{系统响应时间的变化}

热敏电阻的位置会影响系统的响应时间。具体来说,测温点越靠近热源,响应时间越快,测温点越远离热源,响应时间越慢。不同的响应时间会直接影响PID控制器的调节效果:

\begin{itemize}
    \item \textbf{靠近热源的测温点}:由于温度变化快,系统响应时间短,需要设置较小的比例增益($K_p$)和微分增益($K_d$)以避免超调和振荡。同时,可能需要较小的积分增益($K_i$)以防止积分饱和。
    \item \textbf{远离热源的测温点}:温度变化慢,系统响应时间长,需要设置较大的比例增益($K_p$)和微分增益($K_d$)以提高系统响应速度,同时可能需要较大的积分增益($K_i$)以减少稳态误差。
\end{itemize}

\paragraph*{温度梯度的影响}

在实际应用中,测温点位置不同会导致测得的温度值有所差异,尤其是在存在明显温度梯度的系统中。温度梯度会影响PID控制的精确度和稳定性:

\begin{itemize}
    \item \textbf{温度梯度较大的系统}:测温点位置差异较大,可能会导致测得的温度不准确,从而影响PID控制器的输出。需要通过调整PID参数来补偿这一影响。
    \item \textbf{温度梯度较小的系统}:测温点位置对测温精度影响较小,PID参数的调整幅度较小。
\end{itemize}

\paragraph*{热传导路径的影响}

测温点的位置决定了热传导路径的长短,进而影响系统的热惯性和响应速度:

\begin{itemize}
    \item \textbf{热传导路径短}:热惯性小,响应速度快,PID参数需要更加灵敏的调整。
    \item \textbf{热传导路径长}:热惯性大,响应速度慢,PID参数需要更平缓的调整以防止振荡。
\end{itemize}

\paragraph*{总结}

综上所述,热敏电阻测温点位置不同会显著影响PID参数的设定。测温点位置靠近热源或远离热源、系统的温度梯度大小和热传导路径的长短都会改变系统的响应特性,从而需要通过调整PID参数来优化控制效果。




% 思考题2
\begin{question}
	被控物体材料不同, 是否会影响PID 参数的设定, 为什么?
\end{question}



被控物体的材料不同会影响PID参数的设定,原因如下:

\paragraph*{热物性参数的差异}

不同材料的热物性参数(如热导率、比热容和密度)不同,会导致系统的热响应特性差异:

\begin{itemize}
    \item \textbf{热导率}:热导率高的材料传热快,系统的响应时间短,可能需要较小的比例增益($K_p$)和微分增益($K_d$)以防止过调和振荡。相反,热导率低的材料传热慢,响应时间长,可能需要较大的比例增益和微分增益以提高响应速度。
    \item \textbf{比热容}:比热容大的材料吸收热量多,温度变化慢,需要较大的积分增益($K_i$)以消除稳态误差。而比热容小的材料温度变化快,可能需要较小的积分增益以防止积分饱和。
    \item \textbf{密度}:材料的密度影响其热惯性。密度大的材料热惯性大,响应慢,PID参数需要更平缓的调整。密度小的材料热惯性小,响应快,PID参数需要更加灵敏的调整。
\end{itemize}

\paragraph*{热惯性的影响}

材料的热惯性决定了系统对温度变化的响应速度。热惯性大的材料(如金属)响应慢,可能需要较大的$K_p$和$K_d$以提高响应速度,同时需要较大的$K_i$以减少稳态误差。而热惯性小的材料(如塑料)响应快,需要较小的$K_p$和$K_d$以防止过调和振荡,$K_i$也需要相应调整以防止积分饱和。

\paragraph*{热传导路径的差异}

不同材料的热传导路径也会影响PID参数的设定。例如,在热传导路径较短的材料中,系统的热响应速度快,需要更加灵敏的PID参数调整。而在热传导路径较长的材料中,系统的热响应速度慢,需要更平缓的PID参数调整。

\paragraph*{热容的差异}

材料的热容影响其温度变化的幅度和速度。热容大的材料吸收热量多,温度变化慢,可能需要较大的$K_i$以减少稳态误差。热容小的材料温度变化快,需要较小的$K_i$以防止积分饱和。

\paragraph*{总结}

综上所述,被控物体的材料不同会显著影响PID参数的设定。不同材料的热物性参数(热导率、比热容、密度)、热惯性和热传导路径会导致系统的热响应特性差异,从而需要通过调整PID参数来优化控制效果。




% 思考题3
\begin{question}
	除了脉冲宽度调制控温, 可以使用连续信号控温吗? 如何实现?
\end{question}



连续信号控温通过调节一个连续变化的控制信号(如电压或电流)来控制加热或制冷元件的输出功率,从而实现温度调节。以下是几种常见的连续信号控温方法及其实现方式:

\paragraph*{模拟电压控制}

通过调节模拟电压来控制加热器或制冷器的输出功率。这种方法通常使用模拟控制器,如运算放大器和线性调节器。

\begin{itemize}
    \item \textbf{电压源}:提供一个可调节的电压信号。
    \item \textbf{线性调节器}:根据输入的模拟电压信号调整输出功率。
    \item \textbf{温度传感器}:测量当前温度,并将温度信号反馈给控制器。
    \item \textbf{PID控制器}:根据温度反馈信号调整输出电压,实现温度控制。
\end{itemize}

实现步骤:

\begin{enumerate}
    \item 将温度传感器连接到PID控制器,以测量当前温度。
    \item PID控制器根据设定的目标温度计算误差,并输出一个相应的控制电压信号。
    \item 线性调节器根据PID控制器输出的控制电压信号调整加热器或制冷器的功率。
    \item 温度传感器实时测量温度,并将反馈信号传递给PID控制器,形成闭环控制。
\end{enumerate}



\paragraph*{模拟电流控制}

类似于模拟电压控制,通过调节模拟电流来控制加热或制冷元件的输出功率。这种方法通常使用电流源和电流控制器。

\begin{itemize}
    \item \textbf{电流源}:提供一个可调节的电流信号。
    \item \textbf{电流控制器}:根据输入的模拟电流信号调整输出功率。
    \item \textbf{温度传感器}:测量当前温度,并将温度信号反馈给控制器。
    \item \textbf{PID控制器}:根据温度反馈信号调整输出电流,实现温度控制。
\end{itemize}

实现步骤:

\begin{enumerate}
    \item 将温度传感器连接到PID控制器,以测量当前温度。
    \item PID控制器根据设定的目标温度计算误差,并输出一个相应的控制电流信号。
    \item 电流控制器根据PID控制器输出的控制电流信号调整加热器或制冷器的功率。
    \item 温度传感器实时测量温度,并将反馈信号传递给PID控制器,形成闭环控制。
\end{enumerate}



\paragraph*{线性放大器控制}

使用线性放大器(如晶体管或运算放大器)来调节加热或制冷元件的功率输出。这种方法可以提供更平滑的控制信号,减少功率损耗。

\begin{itemize}
    \item \textbf{线性放大器}:放大控制信号,并驱动加热器或制冷器。
    \item \textbf{温度传感器}:测量当前温度,并将温度信号反馈给控制器。
    \item \textbf{PID控制器}:根据温度反馈信号调整输出信号,实现温度控制。
\end{itemize}

实现步骤:

\begin{enumerate}
    \item 将温度传感器连接到PID控制器,以测量当前温度。
    \item PID控制器根据设定的目标温度计算误差,并输出一个相应的控制信号。
    \item 线性放大器根据PID控制器输出的控制信号调整加热器或制冷器的功率。
    \item 温度传感器实时测量温度,并将反馈信号传递给PID控制器,形成闭环控制。
\end{enumerate}


\paragraph*{总结}

使用连续信号控温是一种有效的温度控制方法,可以通过模拟电压控制、模拟电流控制和线性放大器控制等方式实现。连续信号控温具有平滑的输出信号和较低的功率损耗,适用于各种精密温度控制应用。通过合理设置PID参数和反馈机制,可以实现高效稳定的温度控制。







\clearpage
\begin{table}
	\renewcommand\arraystretch{1.7}
	\centering
	\begin{tabularx}{\textwidth}{|X|X|X|X|}
	\hline
	专业:& 物理学 &年级:& 2022级 \\
	\hline
	姓名:& 戴鹏辉\&喻欣宸 & 学号:& 22344016\&22344014 \\
	\hline
	室温:& 26℃ & 实验地点: & A515 \\
	\hline
	学生签名:& & 评分: &\\
	\hline
	实验时间:& 2024/5/24 & 教师签名:&\\
	\hline
	\end{tabularx}
\end{table}

\section{BD4 \quad TEC 半导体控温实验 \quad\heiti 实验记录}
\subsection{实验内容和步骤}

	\subsubsection{实验一 \quad 连接电路和测试}
	



	\subsubsection{实验二 \quad 学习并利用提供的 Labview 程序进行控温实验}





	\subsubsection{实验三 \quad 自行设计控制程序,利用提供的温度变送器进行温度采集,通过 myDAQ的模拟电压输入端进行电压采集,开展控温实验,并附前面板和程序框图的截图。将控温的结果和实验内容二(利用 myDAQ 数字万用表进行温度采集)的结果进行对比。}




\subsection{实验数据记录}

    见\cref{fig:data1}

    \begin{figure}[htbp]
        \centering
        % \subfloat[原始数据1]
        {\includegraphics[width=0.35\textwidth]{Data1.jpg}\label{fig:data1}}
        \quad

        \caption{原始数据}
        \label{fig:data}
    \end{figure}



%\subsection{原始数据记录}



\subsection{实验过程中遇到的问题记录}

\begin{enumerate}
	\item 
\end{enumerate}
	

\clearpage
\begin{table}
	\renewcommand\arraystretch{1.7}
	\begin{tabularx}{\textwidth}{|X|X|X|X|}
	\hline
	专业:& 物理学 &年级:& 2022级\\
	\hline
	姓名: & 戴鹏辉\&喻欣宸 & 学号:& 22344016\&22344014\\
	\hline
    日期:& 2024/5/12 & 评分: &\\
	\hline
	\end{tabularx}
\end{table}

\section{BD4 \quad TEC 半导体控温实验 \quad\heiti 分析与讨论}

\subsection{实验数据分析}


\subsubsection{实验一 \quad 连接电路和测试}
	



\subsubsection{实验二 \quad 学习并利用提供的 Labview 程序进行控温实验}





\subsubsection{实验三 \quad 自行设计控制程序,利用提供的温度变送器进行温度采集,通过 myDAQ的模拟电压输入端进行电压采集,开展控温实验,并附前面板和程序框图的截图。将控温的结果和实验内容二(利用 myDAQ 数字万用表进行温度采集)的结果进行对比。}

				


	
	






\end{document}
