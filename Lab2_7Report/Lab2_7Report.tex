%!TEX program = xelatex
\documentclass[dvipsnames, svgnames,a4paper,11pt]{article}
% ----------------------------------------------------
%   中山大学物理与天文学院本科实验报告模板
%   作者:Huanyu Shi,2019级
%   知乎:https://www.zhihu.com/people/za-ran-zhu-fu-liu-xing
%   Github:https://github.com/huanyushi/SYSU-SPA-Labreport-Template
%   Last update : 2023.4.10
% ----------------------------------------------------

\input{Settings} % 导入模板的相关设置
\usepackage{lipsum}
\usepackage{enumitem}
\usepackage{tabularray}  %绘制表格时可以更加方便添加框线
\usepackage{siunitx}
% \usepackage{ulem}



\setlist[enumerate]{label=\textup{(\arabic*)}}



%---------------------------------------------------------------------
%	正文
%---------------------------------------------------------------------

\begin{document}


\begin{table}
	\renewcommand\arraystretch{1.7}
	\begin{tabularx}{\textwidth}{
		|X|X|X|X
		|X|X|X|X|}
	\hline
	\multicolumn{2}{|c|}{预习报告}&\multicolumn{2}{|c|}{实验记录}&\multicolumn{2}{|c|}{分析讨论}&\multicolumn{2}{|c|}{总成绩}\\
	\hline
	\LARGE25 & & \LARGE30 & & \LARGE25 & & \LARGE80 & \\
	\hline
	\end{tabularx}
\end{table}


\begin{table}
	\renewcommand\arraystretch{1.7}
	\begin{tabularx}{\textwidth}{|X|X|X|X|}
	\hline
	专业:& 物理学 &年级:& 2022级\\
	\hline
	姓名:& 戴鹏辉  & 学号: & 2344016 \\
	\hline
	日期:& 2024/4/24 & 教师签名:& \\
	\hline
	\end{tabularx}
\end{table}

\begin{center}
	\LARGE CC1 \quad 热辐射实验
\end{center}

\textbf{【实验报告注意事项】}
\begin{enumerate}
	\item 实验报告由三部分组成:
	\begin{enumerate}
		\item 预习报告:(提前一周)认真研读\underline{\textbf{实验讲义}},弄清实验原理;实验所需的仪器设备、用具及其使用(强烈建议到实验室预习),完成课前预习思考题;了解实验需要测量的物理量,并根据要求提前准备实验记录表格(第一循环实验已由教师提供模板,可以打印)。预习成绩低于10分(共20分)者不能做实验。
	    \item 实验记录:认真、客观记录实验条件、实验过程中的现象以及数据。实验记录请用珠笔或者钢笔书写并签名(\textcolor{red}{\textbf{用铅笔记录的被认为无效}})。\textcolor{red}{\textbf{保持原始记录,包括写错删除部分,如因误记需要修改记录,必须按规范修改。}}(不得输入电脑打印,但可扫描手记后打印扫描件);离开前请实验教师检查记录并签名。
	    \item 分析讨论:处理实验原始数据(学习仪器使用类型的实验除外),对数据的可靠性和合理性进行分析;按规范呈现数据和结果(图、表),包括数据、图表按顺序编号及其引用;分析物理现象(含回答实验思考题,写出问题思考过程,必要时按规范引用数据);最后得出结论。
	\end{enumerate}
	\textbf{实验报告就是将预习报告、实验记录、和数据处理与分析合起来,加上本页封面。}
	\item 每次完成实验后的一周内交\textbf{实验报告}(特殊情况不能超过两周)。
	\item 实验报告注意事项
		\begin{enumerate}[label=\roman*.]
			\item 实验过程中,禁止触摸辐射体表面,一方面是避免在高温时烫伤;另一方面避免污染表面,影响发射系数。
			\item 测量不同辐射表面对辐射强度影响时,辐射温度不要设置太高,更换辐射体时,应带手套。
			\item 实验过程中,计算机在采集数据时不要触摸测试架,以免造成对传感器的干扰。
			\item 在使用控温程序时,要留意程序是否正常运行,若运行有问题,请及时关闭可编程直流电源的输出,防止辐射器温度过高而损坏。停止运行程序时,要用程序的Stop output按钮, 不要用菜单键的红色强制终止按钮。
			\item 辐射传感器上方有一块金属挡板。在测量时将挡板移开;非测量时关上,避免不必要的触碰污染传感器表面。
		\end{enumerate}
\end{enumerate}


\clearpage
\tableofcontents
\clearpage

\setcounter{section}{0}
\section{CC1 \quad 热辐射实验 \quad\heiti 预习报告}
	
\subsection{实验目的}
	\begin{enumerate}
		\item 认识热辐射现象及其本质(普遍存在的一种能量转换与传递的形式)。
		\item 认识影响热辐射强度的各种因素及其与热辐射强度的定量关系,包括:辐射体表面温度、辐射距离、表面的发射系数等。
		\item 了解热辐射传感器(SMTIR9902)原理和结构、使用(含校正)方法。
		\item 学习应用LabView管理由具有NI通信协议的非NI专业仪器(数字多用表)、设备(程控电源)构成的实验系统。
		
		
	\end{enumerate}

\subsection{仪器用具}
\begin{table}[htbp]
	\centering
	\renewcommand\arraystretch{1.6}
	% \setlength{\tabcolsep}{10mm}
	\begin{tabular}{p{0.05\textwidth}|p{0.20\textwidth}|p{0.05\textwidth}|p{0.5\textwidth}}
	\hline
	编号& 仪器用具名称 & 数量 &  主要参数(型号,测量范围,测量精度等) \\
	\hline
	1&黑体辐射与红外测量装置 	&1 	& DHRH-B:含带标尺(60cm)位移导轨、辐射器(三种辐射面:黑面、粗面、光滑金属面)、热辐射传感器(SMTIR9902),\\

	2&数字多用表 	& 2	& RIGOL DM3058E \\
	
	3&程控电源 	& 1 & RIGOL DP831\\
	
	4&计算机 &1 &已安装 LabView 和控温软件 \\
	
	
	
	\hline
\end{tabular}
\end{table}

\subsection{原理概述}
	物体表面向外辐射连续的电磁波的现象称为热辐射。物体在向外辐射的同时,还会吸收来自外界的电磁(能量)辐射,且物体辐射或吸收的能量与它的温度、表面积、黑度等因素有关,是一个多因素系统。

	普朗克在提出了量子假设,即能量以"量子"形式吸收和发射,从而得到了能够完美描述黑体辐射的普朗克辐射公式,即
	\[
	r_{(\lambda,T)}=\frac{2 h c^2}{\lambda^5}\frac{1}{e^{ch/k\lambda T}-1}	
	\]
	该公式描述了在特定温度和波长下,单位立体角单位面积上的辐射通量。其中,𝜆${\displaystyle \lambda \,}$是波长,𝑇${\displaystyle T\,}$是黑体的温度,ℎ${\displaystyle h\,}$是普朗克常数,𝑐${\displaystyle c\,}$是光速,𝑘${\displaystyle k\,}$是玻尔兹曼常数。

	% 将该公式对$\lambda$从0到无穷积分,即可得到斯特凡-玻尔兹曼公式
	% \[
	% 	J =4\pi \int_{0}^{\infty} \frac{2hc^2}{\lambda^5} \frac{1}{e^{ch/(k\lambda T)} - 1} \, d\lambda=\frac{36\pi ek^4}{c^2h^3}T^4=\sigma T^4
	% \]

	斯特藩-玻尔兹曼公式为
	\[
	R_T=\sigma T^4	
	\]

	其中$\sigma=5.673\times10^{-12}W/cm^2K^4$为斯特藩-玻尔兹曼常数。

	韦恩位移定律为:
	\[
	\lambda_{max}T=b	
	\]
		
	其中$b=2.8978\times10^{-3}( m· K )$\\

	DHRH-B测试台由安装在竖直导轨上的辐射源、与辐射源距离可调的热辐射传感器组成。 辐射体由辐射面、安装在辐射面金属(内部)上的并与金属块有良好热接触的陶瓷加热片、 PT1000温度传感器、外围的绝热层以及分别接通加热器和温度传感器的电连接器组成。

	热辐射传感器原理为N个串联的热电偶(热电堆),热电偶一端与吸热面(接近黑体)热连接、另一端与传感器外壳热连接。当吸热面受到热辐射时,吸收热能使温度升高, 从而使热电偶两端\textbf{形成温度差产生热电势},其输出电压信号为每个热电偶之热电势的N倍;辐射通量越大,温差越大,输出电压越高。\textbf{当达到平衡时,热电堆热端从辐射中吸收的热量等于它从热端传到冷端的热量},这一热量又正比于热电偶冷热端的温差,从而正比于热电堆输出的电压
	\[
		\dot{Q} = \dot{q} A = \kappa\Delta T \propto \Delta V
	\]
	因此,SMTIR9902的输出信号正比于当地从辐射方向进入的辐射通量或能流密度,或接收功率$\Delta V=S\dot{Q}$, 其中S为探测灵敏度;从说明书得到,$ S=110\pm20(V/W)@500K$。传感器接收面积已确定$(A= 0.5{mm}^2)$,从而可以计算当地辐射通量和对辐射源的立体角,再进一步计算辐射源辐射强度(公式以 (W/sr) 为单位)
	\[
		I_E= d\varphi /d\Omega=\dot{q}d^2
	\]\\
	
	实验中做了一系列举措以减小误差,如:
	\begin{enumerate}
		\item SMTIR9902系列热辐射传感器面向低温热辐射测量,为过滤环境辐射的影响,增加了滤镜,以过滤可见光的影响;
		\item SMTIR9902SIL自带透镜,减少接收角从而减少非探测物辐射的影响。
		\item 传感器外壳带Ni1000热敏电阻,以便测量冷端(外壳的)温度。
		\item 实验系统中,热辐射传感器安装在可移动的金属铝臂上,保证传感器外壳与铝臂良好的热接触,而传感器的位置可通过导轨上方的手摇臂连续调节;
		\item 辐射传感器上方提供了一块金属挡板。在测量时可将挡板移开;非测量时关上,避免不必要的触碰污染传感器表面,给测量造成误差。
		\item 热辐射传感器的信号除受到接收面发射率、接收角等自身结构因素的限制以外,还受环境(外壳)温度的影响。因此,不能通过一个预设的校正系数把传感器输出信号直接转换为该地点的辐照度,通常建议在使用环境下进行校正。
	\end{enumerate}
	
	




\subsection{实验前思考题}

	\begin{question}
		人体热辐射会对SMTIR9902系列传感器读数产生影响(自己可验证),如何消除这种影响?日光灯是否有影响呢?这种传感器是否适合测量高温热辐射,为什么?
	\end{question}

	SMTIR9902系列传感器是设计用于检测红外辐射的,而人体也会发出红外辐射,所以人体热辐射会对SMTIR9902系列传感器读数产生影响,在测量时可保持人的占位不动作为背景,最后扣除。\\

	日光灯也会发出一定的红外辐射,在测量时可将传感器放置在远离日光灯直接照射的位置,以减小其影响。\\

	根据产品资料,该传感器工作温度范围介于 -40°C 至 100°C 之间,这意味着它适用于在这个温度范围内的非接触式温度测量。所以其不适合测量高温热辐射。


\begin{question}
	如果对辐射体制冷,使辐射表面温度低于室温(传感器温度) ,辐射传感器输出信号会如何变化?
\end{question}

	当辐射体被制冷,使其表面温度低于室温时,辐射传感器的输出信号将会减少。因为辐射强度与物体的表面温度有关。根据斯特藩-玻尔兹曼定律,物体的辐射能量与其表面温度的四次方成正比。因此,如果辐射体的表面温度降低,它发出的红外辐射能量也会相应减少,导致传感器的输出信号减小。	

	% 因为辐射体和传感器都有辐射强度且与温度相关,当二者都处于室温时,传感器的向外辐射功率等于传感器接收的辐射体的辐射功率,二者抵消使得传感器示数为零;而当辐射体被制冷,使其表面温度低于室温时,辐射传感器的输出信号将会减少。


\begin{question}
	按辐射定律,处于室温的物体也有辐射,为何辐射传感器的输出信号为零?如果要直接测量该物体的辐射强度(辐射传感器输出信号正比于辐射强度),环境(传感器外壳)温度应该多高? 如果环境(传感器外壳)温度维持在室温附近但温度有变化, 是否可以通过数学方法扣除(非绝对零度的)环境温度的影响?(对给定辐射源温度, 传感器外壳温度与辐射传感器读数之间是什么关系? 提示,热辐射传感器内置Ni1000可以测量外壳的温度,其分度表见附录2)
\end{question}

	辐射传感器的输出信号为零,可能是因为传感器已经被校准为在环境温度下输出零信号,以确保扣除背景噪声。

	要直接测量物体的辐射强度,环境(传感器外壳)温度应该低于被测物体的温度,以便传感器能够检测到来自物体的辐射而不是环境的辐射。

	如果环境温度维持在室温附近但有变化,可以通过数学方法扣除环境温度的影响,例如,可以使用差分或比例控制算法来调整传感器的输出,从而消除环境温度变化的影响。

	对于给定辐射源温度,传感器外壳温度与辐射传感器读数之间的关系取决于传感器的特性和校准。通常,传感器会有一个指定的工作温度范围,在此范围内,传感器的读数与辐射源的温度成正比。如果外壳温度变化,可能需要进行校准或使用数学模型来确保读数的准确性。






\begin{question}
	辐射体的加热功率与辐射体温度之间呈何关系?与辐射传感器的信号值之间呈何关系?为什么?
\end{question}

	辐射体的加热功率与辐射体温度之间的关系遵循斯特藩-玻尔兹曼定律,该定律指出黑体的辐射功率与其绝对温度的四次方成正比。辐射传感器的信号值与辐射体的温度成正比。

	\textbf{详见前文原理概述}

	


\begin{question}
	(选) (对防热辐射实验内容)透过防热辐射材料后的辐照度随材料的厚度怎么变化?请推导并验证。
\end{question}

	透过防热辐射材料后的辐照度(即辐射通量密度)随材料厚度的变化,可以通过考虑材料的透射率来推导。透射率通常与材料的厚度和材料的光学性质有关。一般来说,透射率会随着材料厚度的增加而指数下降。
	
	假设材料的初始透射率为 ( $\tau_0$ ),则透过厚度为 ( $d$ ) 的材料后的透射率 ( $\tau$ ) 可以表示为:
	\[
		\tau(d)=\tau_0e^{-\alpha d}
	\]
	其中,( $\alpha$ ) 是材料的吸收系数,它与材料的光学性质有关。
	
	辐照度 ( $E$ ) 与透射率 ( $\tau$ ) 成正比,因此,如果初始辐照度为 ( $E_0$ ),则透过材料后的辐照度  $E(d)$  可以表示为:
	\[
		E(d)=E_0\tau(d)=E_0\tau_0 e^{-\alpha d}
	\]
	
	这表明,随着材料厚度的增加,辐照度呈指数衰减。






\clearpage
\begin{table}
	\renewcommand\arraystretch{1.7}
	\centering
	\begin{tabularx}{\textwidth}{|X|X|X|X|}
	\hline
	专业:& 物理学 &年级:& 2022级 \\
	\hline
	姓名:& 戴鹏辉 & 学号:& 22344016 \\
	\hline
	室温:& 26℃ & 实验地点: & A510 \\
	\hline
	学生签名:& & 评分: &\\
	\hline
	实验时间:& 2024/4/18 & 教师签名:&\\
	\hline
	\end{tabularx}
\end{table}

\section{CC1 \quad 热辐射实验 \quad\heiti 实验记录}
\subsection{实验内容和步骤}

\subsubsection{实验一 \quad 实验系统搭建}

	
	\begin{enumerate}
		\item 辐射体中的加热模块中有两个$10\Omega$电阻,他们两个可能是独立的,也可能是共地的,我们需要用万用表欧姆档判断哪两条线之间对应一个电阻;判断完后,将两个电阻并联接入电源,使用并联的目的是增大功率,加快加热速度。
		\item 辐射体中的温度传感器模块同样有两个$PT1000$电阻,同样使用万用表欧姆档判别,随后只需将一个$PT1000$电阻接入一个万用表。
		\item 辐射传感器部分有一个热电堆和一个$NI1000$电阻,在本次实验中使用热电堆最为辐射传感器,将其接入一个万用表。
		\item 所有的接线完成后,将辐射体和辐射传感器安装在实验架上。
		\item 检查数据线是否与电脑程序正常连接。
		\item 所有准备工作完成后,开始下面的实验步骤。
	\end{enumerate}



\subsubsection{实验二 \quad 测量物体的辐射面温度对辐射强度大小的影响}


	保持辐射传感器的位置不动(实验中将辐射传感器固定在124.5mm的位置),按照一定的温度间隔升高辐射体温度,记录对应温度下的\textbf{辐射传感器电压输出值}和\textbf{电源的稳定输出功率值}。实验数据如\cref{tbl:table2-1-1}:


	\begin{table}[htbp]
		\centering
		\begin{tblr}{
		  cells = {c},
		  vlines,
		  hline{1-2,8} = {-}{},
		}
		表面温度/℃ & 辐射传感器电压/mV & 电源功率/W \\
		50     & 0.48       & 1.047  \\
		60     & 0.68       & 1.700  \\
		70     & 0.89       & 2.050  \\
		80     & 1.12       & 2.510  \\
		90     & 1.37       & 3.078  \\
		100    & 1.64       & 3.650  
		\end{tblr}
		\caption{测量物体的辐射面温度对辐射强度大小的影响实验数据}
		\label{tbl:table2-1-1}
	\end{table}


	升温过程中,可在软件控制界面看到\cref{fig:graph2-2}:


	\begin{figure}[htbp]
		\centering
		\subfloat[辐射体温度]
		{\includegraphics[width=0.8\textwidth]{grap2-2-2.png}\label{fig:grap2-2-2}}
		\quad
		\subfloat[电源输出功率]
		{\includegraphics[width=0.8\textwidth]{grap2-2-3.png}\label{fig:grap2-2-3}}
		\quad
		\subfloat[辐射传感器电压]
		{\includegraphics[width=0.8\textwidth]{grap2-2-4.png}\label{fig:grap2-2-4}}
		\quad
		\caption{测量物体的辐射面温度对辐射强度大小的影响实验结果图像}
		\label{fig:graph2-2}
	\end{figure}

	可以从图中明显地看到,温度上升是阶梯式的,达到设定温度后便会趋于稳定;功率在设定新的温度的瞬间便会脉冲式的上升,当温度稳定后便趋于稳定值;同样的,辐射传感器电压也是阶梯式上升。



\subsubsection{实验三 \quad 测量在不同辐射距离在$d$的辐射传感器输出$U$, 拟合给出$U$-$d$之间的关系}

	在上个试验结束后,辐射体温度已经上升到100℃,此时保持温度不变,从近到远改变辐射传感器和辐射体之间的距离,记录对应距离下\textbf{辐射传感器的输出电压值}。实验数据如\cref{tbl:table3-1-1}:

	\begin{table}[htbp]
		\centering
		\begin{tabular}{|cc|} 
		\hline
		距离/mm & 辐射传感器电压/mV  \\ 
		\hline
		124.5 & 1.64        \\
		136   & 1.27        \\
		146   & 1.05        \\
		155.5 & 0.87        \\
		166   & 0.74        \\
		176   & 0.64        \\
		186   & 0.56        \\
		196   & 0.49        \\
		206   & 0.43        \\
		216   & 0.40        \\
		226   & 0.359       \\
		236   & 0.336       \\
		246   & 0.308       \\
		256   & 0.285       \\
		266   & 0.267       \\
		276   & 0.248       \\
		286   & 0.235       \\
		296   & 0.215       \\
		306   & 0.203       \\
		316   & 0.194       \\
		\hline
		\end{tabular}
		\caption{测量在不同辐射距离的辐射传感器输出实验数据}
		\label{tbl:table3-1-1}
	\end{table}










\subsubsection{实验四 \quad 测量不同物体表面的发射系数(选)}

	我们将不同的辐射体升高至相同的温度(保持不同的辐射体均在60℃),将辐射传感器放置在相同的位置上(实验中将辐射传感器固定在124.5mm的位置),通过比较辐射传感器的输出电压值的不同,判断不同辐射体的表面发射系数。实验数据如\cref{tbl:table4-1-1}:

	\begin{table}[htbp]
		\centering
		\begin{tblr}{
		cells = {c},
		vline{1-2,5} = {-}{},
		hline{1,3} = {-}{},
		}
		辐射体        & 黑面   & 光面   & 粗糙面   \\
		辐射传感器电压/mV & 0.68 & 0.24 & 0.302 
		\end{tblr}
		\caption{测量不同物体表面的发射系数实验数据}
		\label{tbl:table4-1-1}
	\end{table}






%\subsection{实验数据记录}



\subsection{原始数据记录}

	见\cref{fig:data}

	\begin{figure}[htbp]
		\centering
		\subfloat[原始数据1]
		{\includegraphics[width=0.35\textwidth]{OriginalData1.jpg}\label{fig:data1}}
		\quad


		\caption{原始数据}
		\label{fig:data}
	\end{figure}



\subsection{实验过程中遇到的问题记录}

\begin{enumerate}
	\item 在正式开始升温测量前,一定要确保接线正确。如果接线错误,可能会导致程序错误判断辐射体当前温度,导致升温过度损坏实验设备。
	\item 一定要确保三台实验仪器与电脑之间正常连接。
	\item 温度越高,PID的控温效果越好,所以起始温度可以设置在50-60℃,以保证实验数据更加准确。
	
\end{enumerate}
	

\clearpage
\begin{table}
	\renewcommand\arraystretch{1.7}
	\begin{tabularx}{\textwidth}{|X|X|X|X|}
	\hline
	专业:& 物理学 &年级:& 2022级\\
	\hline
	姓名: & 戴鹏辉 & 学号:& 22344016\\
	\hline
    日期:& 2024/4/25 & 评分: &\\
	\hline
	\end{tabularx}
\end{table}

\section{CC1 \quad 热辐射实验 \quad\heiti 分析与讨论}

\subsection{实验数据分析}

	\subsubsection{实验一 \quad 实验系统搭建}

	



	\subsubsection{实验二 \quad 测量物体的辐射面温度对辐射强度大小的影响}


	当辐射体的温度恒定时,由斯特藩-玻尔兹曼定律,其向外辐射功率应当满足$J\propto T^4$,而辐射传感器电压也满足$U\propto J$,则$U\propto T^4$。

	使用公式$U=kT^4+bias_2$拟合,$T$为热力学温标,$bias_2$为偏置,用于消除环境的影响,结果如\cref{fig:graph3-2-2}
	\begin{figure}[htbp]
		\centering
		\includegraphics[width=0.8\textwidth]{graph3-2-2.png}
		\caption{辐射传感器电压输出值随辐射体温度变化关系}
		\label{fig:graph3-2-2}
	\end{figure}

	可发现,相关系数$R^2=0.999963$,实验所测定的6个数据点确实满足$U\propto T^4$关系,但是曲线不过坐标原点,而是下移了。




	当辐射体的温度恒定时,电源的稳定输出功率值就等于辐射体向外辐射能量的功率,其应当满足$P\propto T^4$的正比关系。

	使用公式$P=kT^4+bias_2$拟合,$bias_2$为偏置,用于消除环境的影响,结果如\cref{fig:graph3-2-1}
	\begin{figure}[htbp]
		\centering
		\includegraphics[width=0.8\textwidth]{graph3-2-1.png}
		\caption{电源稳定输出功率随辐射体温度变化关系}
		\label{fig:graph3-2-1}
	\end{figure}

	同样的,相关系数$R^2=0.993552$,电源的稳定输出功率也满足$P\propto T^4$的正比关系。但同样的,曲线不过坐标原点,有一个下移。\\


	曲线不过原点的原因可能有:
	\begin{enumerate}
		\item 环境造成的影响。只要温度不等于绝对零度就会发出热辐射,所以实验室中各种仪器的辐射也会被辐射传感器捕捉到。
		

	\end{enumerate}
	





	\subsubsection{实验三 \quad 测量在不同辐射距离在$d$的辐射传感器输出$U$, 拟合给出$U$-$d$之间的关系}

	假设我们考虑辐射体发出的辐射是以球面波的形式传播出去的,则辐射传感器的电压值与距离辐射体的距离应当满足平方反比关系,即$U\propto d^{-2}$。

	使用公式$U=k\frac{1}{(d-bias_1)^{2}}+bias_2$拟合,因为辐射体并不是固定在标尺的零点位置,所以加入一个位移偏置$bias_1$,偏置$bias_2$用于消除环境的影响,结果如\cref{fig:graph3-3-1}

	\begin{figure}[htbp]
		\centering
		\includegraphics[width=0.8\textwidth]{graph3-3-1.png}
		\caption{点状辐射源的辐射传感器输出随距离变换关系}
		\label{fig:graph3-3-1}
	\end{figure}

	可见,辐射传感器的电压值与距离确实满足$U\propto d^{-2}$,通过拟合数据,我们也可知道,辐射传感器为位置为$d_0=45.73mm$。\\


	假设我们考虑辐射体是一个平面辐射源,则可根据讲义中的附录(6),得到探测器接收的总功率为$\dot{q}=\frac{\dot{q_a}}{2\pi}\sin^2\alpha$,其中$\alpha$是辐射传感器视角,有$\tan\alpha=\frac{R}{d}$,$R$是辐射体表面的半径,$d$是辐射传感器与辐射体之间的距离。

	则使用公式$U=k\sin^2(\arctan\frac{R}{d-d_0})+C$拟合,$d_0$是考虑辐射体未设置在标尺零点而加入的偏置,$C$是为了消除环境影响而加入的偏置,得到的拟合结果如\cref{fig:graph3-3-2}

	\begin{figure}[htbp]
		\centering
		\includegraphics[width=0.8\textwidth]{graph3-3-2.png}
		\caption{平面辐射源的辐射传感器输出随距离变换关系}
		\label{fig:graph3-3-2}
	\end{figure}

	可以看到,使用两个不同的函数拟合,相关系数$R^2$相差无几,原因可能是我的辐射传感器距离足够远,\textbf{平面辐射源的特性并没有体现出来}。如果是从很近的距离开始测量,比如距离辐射体1cm的位置开始测量,应该会发现使用平面辐射源的模型拟合结果更精确。

	\begin{question}
		以下距离的定义哪个正确:(1)从辐射表面到传感器表面(红外透镜)的距离;(2)从辐射表面到传感器内部热电堆表面的距离。两者对测量误差的影响有多大?
	\end{question}

	我认为是第二个距离的定义更准确。因为经过透镜只是用作滤去可见光的影响,而不是真正探测信号的元件;且辐射只有到达热电堆表面后,才会产生热量,才会有电压差输出。

	综合以上两点,我认为是第二个距离的定义更准确。

	


	\subsubsection{实验四 \quad 测量不同物体表面的发射系数(选)}


	由实验数据\cref{tbl:table4-1-1}可知,在相同的辐射体温度、辐射传感器放置在相同的距离下,辐射传感器的输出电压值,黑面最高,然后是粗糙面、最小的是光面。这说明黑面辐射体的发射系数最大,第二大的是粗糙面辐射体,最低的是光面辐射体。

	原因分析如下:
	\begin{enumerate}
		\item 黑面辐射体:
			\begin{enumerate}[label=\roman*.]
				\item 黑体是指能够吸收投射到其表面上的所有热辐射的物体。
				\item 黑体的发射率等于其吸收率,这是热力学第二定律的推论。
				\item 因为黑体吸收率最大,所以黑体发射率也最大。
				\item 黑体能够有效地吸收外界的能量,因此它能够迅速发热并辐射出更多的能量。
			\end{enumerate}
			
		\item 粗糙面辐射体:
			\begin{enumerate}[label=\roman*.]
				\item 粗糙表面的物体具有不规则的微观结构,使得光线在其表面上发生多次反射。
				\item 这些多次反射导致了更多的机会,使物体吸收更多的能量。
				\item 因此,粗糙物体的发射率较高,尽管它不如黑体那么高。
			\end{enumerate}
			
		\item 光面辐射体:
			\begin{enumerate}[label=\roman*.]
				\item 光滑表面的物体具有规则的微观结构,使得光线在其表面上只发生少量反射。
				\item 光滑物体的吸收能力较低,因为它无法有效地吸收外界的能量。
				\item 因此,光滑物体的发射率最低。
			\end{enumerate}
			
			
	\end{enumerate}
	


	


			




\subsection{实验后思考题}

\begin{question}
	为何当热辐射传感器太靠近辐射表面时,所测出的辐照度偏离距离平方反比规律?
\end{question}

	因为当距离靠的过近时,传感器能接收到的辐射体面积显著减小,随着距离的增大,能接收到的辐射体面积也增大,当传感器能完全接收辐射体整个面积的辐射量时,从这个位置往后,辐照度原则上将与距离保持平方反比关系。




\begin{question}
	(选) SMTIR9902与SMTIR9902SIL各有什么特点?各适合于什么场景的非接触温度测量?为什么?
\end{question}

	

	SMTIR9902和SMTIR9902SIL是两种红外测温传感器,它们都基于热电堆的硅基红外传感器,用于非接触式的温度测量。

	\begin{enumerate}
		\item SMTIR9902:

		SMTIR9902系列是基于热电堆的硅基红外传感器,具有成熟的半导体工艺,降低成本并使器件小型化。
	
		特点:高精度、高灵敏度(110V/W)、高信噪比(Rs约50kΩ)、快速响应时间(40毫秒)、采用低成本薄膜工艺、高通红外滤波(波长>5.5μm)、标准TO05封装
		
		应用:适用于红外辐射探测。
		\item SMTIR9902SIL:

		SMTIR9902SIL是标准SMTIR9902传感器,配备了硅透镜,用于在小视角(开口视角小于7°)内测量辐射温度。
	
		特点: 高精度、高灵敏度(110V/W)、高信噪比(Rs约50kΩ)、快速响应时间(40毫秒)、低成本薄膜工艺、高通红外滤波(波长>5.5μm)、标准TO05封装、包含用于测量传感器自身温度的NI1000热敏电阻
		
		应用:适用于需要在小视角内测量辐射温度的场景。
	\end{enumerate}

	综上所述,SMTIR9902适用于一般的红外辐射探测,而SMTIR9902SIL适用于需要在小视角内测量辐射温度的特定场景。



\begin{question}
	辐射传感器输出信号与辐射表面绝对温度4次方成线性关系,为何直线不过原点?斯特藩-玻尔兹曼定律是否成立?为什么?
\end{question}

	不过原点的原因可能是因为忽略了环境噪声的影响。实验环境中的日光灯管同样会发出辐射,会被辐射传感器接收。但是关于日光灯管的辐射量难以评估。



\begin{question}
	(设计性实验) 如何评价材料(如透光隔热膜)的防热辐射能力及其在窗户应用中的节能效果?
\end{question}

	透光隔热膜是一种应用于玻璃窗户的材料,具有防热辐射能力。

	透光隔热膜的特点:
	\begin{enumerate}
		\item 隔热功能:透光隔热膜可以有效地阻挡太阳能和紫外线,降低室内的温度。
		\item 节能效果:通过减少透过窗户的太阳辐射热量,透光隔热膜可以降低空调的制冷成本。
		\item 提高舒适度:它可以改善室内环境,使房间保持均匀、适宜的温度。
	\end{enumerate}
	
	实验方案:
	\begin{enumerate}
		\item 目标:评估透光隔热膜在窗户应用中的节能效果。
		\item 步骤:
			\begin{enumerate}[label=\roman*.]
				\item 选择窗户:选择一扇普通的玻璃窗户作为实验对象。
				\item 测量基线数据:在不安装透光隔热膜的情况下,测量室内和室外的温度。记录空调的制冷负荷。
				\item 安装透光隔热膜:在窗户的内表面上安装透光隔热膜。再次测量室内和室外的温度,记录空调的制冷负荷。
				\item 分析结果:比较安装透光隔热膜前后的温度变化。
			\end{enumerate}
	
		\item 计算节能效果,即空调制冷负荷的减少百分比。
	\end{enumerate}
	
	
	通过这个实验方案可以评估透光隔热膜在窗户应用中的实际效果,以及其对节能的贡献。

	
\begin{question}
	(设计性实验) 如何在(变化的)大气环境温度下测量各热辐射源的温度?
\end{question}

	在(变化的)大气环境温度下测量各热辐射源的温度需要一些特定的实验步骤和仪器。以下是一个实验方案,用于测量不同热辐射源的温度:
	
	\begin{enumerate}
		\item 实验目的:测量不同热辐射源的温度;研究辐射体的发射率和吸收率。
		\item 所需仪器:辐射传感器(感应红外辐射);辐射立方体(不同表面状态的辐射源)。
		\item 实验步骤: 
			\begin{enumerate}
				\item 将辐射传感器放置在不同热辐射源的表面,记录输出电压。
				\item 通过改变辐射体的温度,测量不同表面的辐射率。
				\item 测量辐射体的总辐射强度。 
				\item 计算辐射体的吸收率和透射率。 
				\item 使用Stefan-Boltzmann灯,研究Stefan-Boltzmann定律。Stefan-Boltzmann灯(提供高温热辐射)。
			\end{enumerate}
	\end{enumerate}
	
	注意事项:
	\begin{enumerate}
		\item 确保传感器与辐射体表面接触,以保证测量距离相等。
		\item 考虑不同热辐射源对精度的影响,如灯丝是否符合理想黑体等。
	\end{enumerate}
	





\begin{figure}[htbp]
	\centering
	\includegraphics[width=0.4\textwidth]{table.jpg}
	\caption{实验桌整理后照片}
\end{figure}
	

\end{document}
