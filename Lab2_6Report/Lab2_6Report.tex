%!TEX program = xelatex
\documentclass[dvipsnames, svgnames,a4paper,11pt]{article}
% ----------------------------------------------------
%   中山大学物理与天文学院本科实验报告模板
%   作者:Huanyu Shi,2019级
%   知乎:https://www.zhihu.com/people/za-ran-zhu-fu-liu-xing
%   Github:https://github.com/huanyushi/SYSU-SPA-Labreport-Template
%   Last update : 2023.4.10
% ----------------------------------------------------

\input{Settings} % 导入模板的相关设置
\usepackage{lipsum}
\usepackage{enumitem}
\usepackage{tabularray}  %绘制表格时可以更加方便添加框线
\usepackage{siunitx}
% \usepackage{ulem}



\setlist[enumerate]{label=\textup{(\arabic*)}}



%---------------------------------------------------------------------
%	正文
%---------------------------------------------------------------------

\begin{document}


\begin{table}
	\renewcommand\arraystretch{1.7}
	\begin{tabularx}{\textwidth}{
		|X|X|X|X
		|X|X|X|X|}
	\hline
	\multicolumn{2}{|c|}{预习报告}&\multicolumn{2}{|c|}{实验记录}&\multicolumn{2}{|c|}{分析讨论}&\multicolumn{2}{|c|}{总成绩}\\
	\hline
	\LARGE25 & & \LARGE30 & & \LARGE25 & & \LARGE80 & \\
	\hline
	\end{tabularx}
\end{table}


\begin{table}
	\renewcommand\arraystretch{1.7}
	\begin{tabularx}{\textwidth}{|X|X|X|X|}
	\hline
	专业:& 物理学 &年级:& 2022级\\
	\hline
	姓名:& 戴鹏辉  & 学号: & 2344016 \\
	\hline
	日期:& 2024/4/15 & 教师签名:& \\
	\hline
	\end{tabularx}
\end{table}

\begin{center}
	\LARGE CB2 \quad 偏振光实验
\end{center}

\textbf{【实验报告注意事项】}
\begin{enumerate}
	\item 实验报告由三部分组成:
	\begin{enumerate}
		\item 预习报告:(提前一周)认真研读\underline{\textbf{实验讲义}},弄清实验原理;实验所需的仪器设备、用具及其使用(强烈建议到实验室预习),完成课前预习思考题;了解实验需要测量的物理量,并根据要求提前准备实验记录表格(第一循环实验已由教师提供模板,可以打印)。预习成绩低于10分(共20分)者不能做实验。
	    \item 实验记录:认真、客观记录实验条件、实验过程中的现象以及数据。实验记录请用珠笔或者钢笔书写并签名(\textcolor{red}{\textbf{用铅笔记录的被认为无效}})。\textcolor{red}{\textbf{保持原始记录,包括写错删除部分,如因误记需要修改记录,必须按规范修改。}}(不得输入电脑打印,但可扫描手记后打印扫描件);离开前请实验教师检查记录并签名。
	    \item 分析讨论:处理实验原始数据(学习仪器使用类型的实验除外),对数据的可靠性和合理性进行分析;按规范呈现数据和结果(图、表),包括数据、图表按顺序编号及其引用;分析物理现象(含回答实验思考题,写出问题思考过程,必要时按规范引用数据);最后得出结论。
	\end{enumerate}
	\textbf{实验报告就是将预习报告、实验记录、和数据处理与分析合起来,加上本页封面。}
	\item 每次完成实验后的一周内交\textbf{实验报告}(特殊情况不能超过两周)。
	\item 实验报告注意事项
		\begin{enumerate}[label=\roman*.]
			\item 实验中不要用手触摸镜片,以免弄脏镜片;
			\item 避免直视光源,注意光源表面高温;
			\item 注意用电安全。
			
		\end{enumerate}
\end{enumerate}


\clearpage
\tableofcontents
\clearpage

\setcounter{section}{0}
\section{CB2 \quad 偏振光实验 \quad\heiti 预习报告}
	
\subsection{实验目的}
\begin{enumerate}
	\item 理解偏振光的基本概念,了解线偏振光、椭圆偏振光和圆偏振光。
	\item 分析偏振光产生的三种方法:吸收、反射和散射。
	\item 了解在各向异性材料介质中光波的传播,学习偏振光通过各向异性介质后,产生的“相位延迟”($\lambda/ 2$波片和$\lambda/ 4$波片) 。
	
\end{enumerate}

\subsection{仪器用具}
\begin{table}[htbp]
	\centering
	\renewcommand\arraystretch{1.6}
	% \setlength{\tabcolsep}{10mm}
	\begin{tabular}{p{0.05\textwidth}|p{0.20\textwidth}|p{0.05\textwidth}|p{0.5\textwidth}}
	\hline
	编号& 仪器用具名称 & 数量 &  主要参数(型号,测量范围,测量精度等) \\
	\hline
	1&光功率计 	&1 	& \\

	2&白光光源 	& 	&  \\
	
	3&光学导轨 	&  & \\
	
	4&光学测角台& & \\
	
	5&偏振片	&3 & \\

	6&半波片	&1 & \\

	7&1/4波片	&2 & \\

	
	\hline
\end{tabular}
\end{table}

\subsection{原理概述}

\subsubsection{偏振光的产生}
\begin{enumerate}
	\item 吸收法起偏:偏振片\\
	偏振片是一种光学元件,通过将普通光转换为偏振光来实现选择性透过。它由特殊材料制成,内部结构呈现有序排列的纤维状物质。当普通光进入偏振片时,它会沿着偏振片内部纤维状结构的方向进行振动。只有与这个方向相同的光才能透过偏振片,而与这个方向垂直的光则被吸收或转换成其他形式的能量。
	
	与之相关的马吕斯定律是一条关于偏振光的重要定律。它描述了偏振光透过偏振片后的强度变化。

	基本概念:
	马吕斯定律指出,强度为$I_0$​的线偏振光,透过检偏片后,透射光的强度(不考虑吸收)为:
	\[ I = I_0 \cos^2 \theta \]
	其中,$\theta$是入射线偏振光的光振动方向和偏振片偏振化方向之间的夹角。

		\begin{figure}[htbp]
			\centering
			\includegraphics[width=0.4\textwidth]{graph1-2.jpg}
			\caption{马吕斯定律示意图}
			\label{fig:graph1-2}
		\end{figure}
	
	\item 玻璃反射起偏
		\begin{enumerate}[label=\roman*.]
			\item 当自然光从一个介质到另一个介质的分界面上反射或折射时,反射光和折射光的偏振性质与入射光不同。
			\item 反射光中垂直振动强于平行振动,折射光中平行振动强于垂直振动。
			\item 特别地,当入射角是Brewster角时,反射光只有垂直于入射面的振动,也就是线偏振光,无平行振动。
				振幅反射系数的平行分量为:$r_\parallel =\frac{\tan(\theta_i-\theta_t)}{\tan(\theta_i+\theta_t)}$,则当$\theta_i+\theta_t=\pi/2$时,$r_\parallel=0$,由斯涅尔定律得$n_i\sin\theta_1=n_t\sin\theta_t=n_t\sin(\pi/2-\theta_i)\longrightarrow\tan\theta_b=\frac{n_t}{n_i}$
				
				\begin{figure}[htbp]
					\centering
					\includegraphics[width=0.4\textwidth]{graph1-1.png}
					% \caption{出射的折曲波阵面}
					\label{fig:graph1-1}
				\end{figure}


		\end{enumerate}
	\item 散射起偏
		\begin{enumerate}[label=\roman*.]
			\item 物质微粒对光的反射和折射会产生散射现象。
			\item 散射光也是横波,振动垂直于传播方向。
			\item 自然光经过散射后,不同方向的散射光具有不同的偏振性,包括线偏振光和部分偏振光。
			\item 当散射微粒的大小小于光波波长$\lambda$可被视为产生瑞利散射。
			瑞利散射的电磁辐射功率$P_{Ray}\propto\frac{1}{\lambda^4}$,因此, “蓝”光散射比“红”光更强烈。
		\end{enumerate}
\end{enumerate}




\subsubsection{介质的各向异性}

晶体介质的折射率与光波的偏振方向有关,是各向异性的。当晶体介质的折射率在两个方向上不同时,分别记作$n_x$和$n_y$,且$n_x>n_y$时,可以定义两个方向上的相位速度$c_x$和$c_y$,其中$c_x<c_y$。晶体中的x轴被称为慢轴,y轴被称为快轴。在晶体中,一条折射线总符合普通的折射定律,称作寻常光(或o光),而另一条折射线不遵守普通的折射定律,称作非常光(或e光)。o光和e光都是偏振光,且两光束的振动方向相互垂直。

因为通过该偏振片导致的光程差为:$\delta=(n_x-n_y)d$。$d$为偏振片厚度。

\begin{enumerate}
	\item $\lambda/ 2$波片指,对于一个给定的波长$\lambda$平行光正入射波晶片时, o光和e光的光程差$\delta=\frac{\lambda}{2}$。
	另外, $\lambda/ 2$波片是对特定的波长的光通过,否则没有任何意义。
	
	若线偏振光经过$\lambda/ 2$波片,出射光还是线偏振光,但相对于o轴或e轴对称。特别的,平行于o轴或e轴入射的偏振光,经过波片后方向保持不变。
	
	若椭圆偏振光经过$\lambda/ 2$波片,出射光仍是椭圆偏振光,但旋转方向相反。

	\item $\lambda/ 4$波片指,对于一个给定的波长$\lambda$平行光正入射波晶片时, o光和e光的光程差$\delta=\frac{\lambda}{4}$。
	同样的, $\lambda/ 4$波片是对特定的波长的光通过,否则没有任何意义。

	线偏振光通过$\lambda/ 4$波片后,当入射角为0°、90°、180°、270°时,出射光仍然是线偏振光。\\
	当入射角为45°、135°、225°、315°时,出射光变为圆偏振光。\\
	对于其他入射角度,出射光为椭圆偏振光。


\end{enumerate}



\subsection{实验前思考题}

\begin{question}
	解释什么是“瑞利散射”?
\end{question}


瑞利散射是指当光线与物质中的微小颗粒(比如气体、液滴或者固体微粒)相互作用时,由于颗粒尺寸远小于光波长,光波在颗粒表面的散射现象。这种散射是光波与颗粒表面的无规则碰撞结果,使得原本直行的光线方向发生改变,呈现出向各个方向均匀散射的特点。

瑞利散射的强度和波长的四次方成反比,因此瑞利散射对于较短波长(如紫外线和蓝光)的光更为显著。这也是为什么蓝天看起来比较亮,因为大气中的气体分子对蓝色光的瑞利散射比对其他波长的光更有效。



\begin{question}
	举例说明瑞利散射。
\end{question}

举例来说,太阳光穿过大气层时会发生瑞利散射,使得来自各个方向的光线在空气中散射并抵达地面。这就是为什么在白天天空是蓝色的原因,因为蓝光的波长较短,更容易被大气中的气体分子所散射,而其他波长的光则被散射得更少,因此天空呈现出蓝色调。
	

\clearpage
\begin{table}
	\renewcommand\arraystretch{1.7}
	\centering
	\begin{tabularx}{\textwidth}{|X|X|X|X|}
	\hline
	专业:& 物理学 &年级:& 2022级 \\
	\hline
	姓名:& 戴鹏辉 & 学号:& 22344016 \\
	\hline
	室温:& 26℃ & 实验地点: & A510 \\
	\hline
	学生签名:& & 评分: &\\
	\hline
	实验时间:& 2024/4/18 & 教师签名:&\\
	\hline
	\end{tabularx}
\end{table}

\section{CB2 \quad 偏振光实验 \quad\heiti 实验记录}
\subsection{实验内容和步骤}

	\subsubsection{实验一 \quad 研究交叉线性偏振片}
	
	实验器材:白光光源, 凸透镜, 可变光阑,两个线性偏振片,光学导轨, 屏幕。

	光路示意图如\cref{fig:graph2-1-3}所示,实验步骤如下:
	\begin{enumerate}
		\item 用自准直等快速方法产生准直光束。(\textbf{实验中使用$f=150mm$的凸透镜获得平行光})
		\item 放置第一个偏振片(称为$P_1$)并改变透振方向。观察屏幕上的光线强度。 总结白光发出的自然光的性质。

		\item 放置第二个偏振片(称为$P_2$),旋转$P_2$, 改变透振方向,观察屏幕上光强的变化,发现光两次出现消失的现象,确定消光的位置。 这时两个偏振片透振方向相垂直。同样的,旋转偏振片$P_1$,能得到同样的结果。
		\item 在消光的位置,将$P_1$轴旋转20°,证明$P_2$轴旋转同样的角度时,消光现象再次出现。
		\item 在偏振片$P_1$和$P_2$之间放置第3个偏振片$P_3$。先确定$P_3$的透振方向,再转45°的角度,观察并说明现象。
		\item 拿走偏振片$P_3$,在$P_1$和$P_2$之间插入塑料板或三角尺等透明物体, 观察现象;转动其中任何一个元件,观察现象;给塑料板加力,观察并说明这三个现象的变化。(\textbf{实验中因缺少三角尺而未做})

	\end{enumerate}

	\begin{figure}[htbp]
		\centering
		\includegraphics[width=0.6\textwidth]{graph2-1-3.jpg}
		\caption{研究交叉线性偏振片、验证验证马吕斯定律光路示意图}
		\label{fig:graph2-1-3}
	\end{figure}


	\subsubsection{实验二 \quad 验证马吕斯定律}

	实验器材: 白光光源, 2个滤光片(绿色和红色) ,两个会聚透镜(其中一个小焦距的起聚光作用,f值很小) , 可变光阑,两个偏振片,光功率计。

	光路示意图如\cref{fig:graph2-1-3}所示,具体实验步骤如下:
	\begin{enumerate}
		\item 在准直平行光路上摆放偏振片$P_1$和$P_2$。其中$\theta$为两个透振方向的夹角, $I$为光功率计测量的光强值。其中$I_0$为两个偏振器片透振方向平行时测得的光强度,即$I_0=I(\theta=0)$
		\item 改变不同的角度(测量大约10个值),并记录对应的光强值。通过绘制合理的曲线图, 验证马吕斯定律。
		\item 重新开始上述实验。放置红色的滤光片在可变光阑前, 如果需要的话, 可用一个小焦距的聚集透镜,将光束聚焦在光功率计上(如果信号过弱)。
		\item 放置绿色的滤光片在可变光阑前,重复上述实验内容, 比较并解释结果。
	\end{enumerate}

	得到实验数据如\cref{tbl:table2-1}所示:

	% % \usepackage{tabularray}
	% \begin{table}[htbp]
	% 	\centering
	% 	\begin{tblr}{
	% 	cells = {c},
	% 	vline{1,3,5} = {-}{},
	% 	hline{1-2,12} = {-}{},
	% 	}
	% 	$\theta$  & I/mW  & $\theta$   & I/mW  \\
	% 	0  & 0.584 & 100 & 0.768 \\
	% 	10 & 0.590 & 110 & 0.748 \\
	% 	20 & 0.606 & 120 & 0.724 \\
	% 	30 & 0.632 & 130 & 0.693 \\
	% 	40 & 0.666 & 140 & 0.660 \\
	% 	50 & 0.698 & 150 & 0.628 \\
	% 	60 & 0.729 & 160 & 0.603 \\
	% 	70 & 0.751 & 170 & 0.587 \\
	% 	80 & 0.767 & 180 & 0.581 \\
	% 	90 & 0.775 &     &       
	% 	\end{tblr}
	% \end{table}

	% \begin{table}[htbp]
	% 	\centering
	% 	\begin{tabular}{|cc|} 
	% 	\hline
	% 	$\theta$   & $I/\mu W$  \\ 
	% 	\hline
	% 	0   & 15.34  \\
	% 	10  & 15.38  \\
	% 	20  & 15.58  \\
	% 	30  & 15.78  \\
	% 	40  & 16.08  \\
	% 	50  & 16.42  \\
	% 	60  & 16.74  \\
	% 	70  & 16.98  \\
	% 	80  & 17.06  \\
	% 	90  & 17.11  \\
	% 	100 & 16.96  \\
	% 	\hline
	% 	\end{tabular}
	% \end{table}


	% \begin{table}
	% 	\centering
	% 	\begin{tabular}{|cc|} 
	% 	\hline
	% 	$\theta$   & $I/\mu W$  \\
	% 	\hline
	% 	0   & 15.92  \\
	% 	10  & 16.1   \\
	% 	20  & 16.45  \\
	% 	30  & 17.08  \\
	% 	40  & 17.68  \\
	% 	50  & 18.36  \\
	% 	60  & 19.06  \\
	% 	70  & 19.64  \\
	% 	80  & 20.01  \\
	% 	90  & 20.08  \\
	% 	100 & 19.92  \\
	% 	\hline
	% 	\end{tabular}
	% \end{table}


	\begin{table}[htbp]
		\centering
		\begin{minipage}[t]{0.3\linewidth}
			\centering
			\begin{tblr}{
				cells = {c},
				vline{1,3,5} = {-}{},
				hline{1-2,12} = {-}{},
			}
			$\theta$  & $I/mW$  & $\theta$   & $I/mW$  \\
			0  & 0.584 & 100 & 0.768 \\
			10 & 0.590 & 110 & 0.748 \\
			20 & 0.606 & 120 & 0.724 \\
			30 & 0.632 & 130 & 0.693 \\
			40 & 0.666 & 140 & 0.660 \\
			50 & 0.698 & 150 & 0.628 \\
			60 & 0.729 & 160 & 0.603 \\
			70 & 0.751 & 170 & 0.587 \\
			80 & 0.767 & 180 & 0.581 \\
			90 & 0.775 &     &       
			\end{tblr}
			\caption{白光光源实验数据}
		\end{minipage}%
		\hfill
		\begin{minipage}[t]{0.3\linewidth}
			\centering
			\begin{tabular}{|cc|} 
			\hline
			$\theta$   & $I/\mu W$  \\ 
			\hline
			0   & 15.34  \\
			10  & 15.38  \\
			20  & 15.58  \\
			30  & 15.78  \\
			40  & 16.08  \\
			50  & 16.42  \\
			60  & 16.74  \\
			70  & 16.98  \\
			80  & 17.06  \\
			90  & 17.11  \\
			100 & 16.96  \\
			\hline
			\end{tabular}
			\caption{绿光光源实验数据}
		\end{minipage}%
		\hfill
		\begin{minipage}[t]{0.3\linewidth}
			\centering
			\begin{tabular}{|cc|} 
			\hline
			$\theta$   & $I/\mu W$  \\
			\hline
			0   & 15.92  \\
			10  & 16.1   \\
			20  & 16.45  \\
			30  & 17.08  \\
			40  & 17.68  \\
			50  & 18.36  \\
			60  & 19.06  \\
			70  & 19.64  \\
			80  & 20.01  \\
			90  & 20.08  \\
			100 & 19.92  \\
			\hline
			\end{tabular}
			\caption{红光光源实验数据}
		\end{minipage}
		\caption{验证马吕斯定律实验数据}
		\label{tbl:table2-1}
	\end{table}

	\subsubsection{实验三 \quad 玻璃反射起偏和布儒斯特角的测量}


	具体实验步骤如下:
	\begin{enumerate}
		\item 通过旋转偏振片,观察教室的天花板的日光灯管光强的变化,说明日光灯是没有偏振的。
		\item 通过旋转偏振片,观察日光灯管在桌子上玻璃的反射像, 或者在地面瓷砖上的反射像,说明随着偏振片的旋转, 反射光的强度变化。
		\item 改变观察者的位置,也就是改变天花板上日光灯管的入射角度,使光通过偏振片的某方向光强为零时,计算出日光灯管的入射角。 验证玻璃的折射率n= 1.5时, 布儒斯特角值大约为56°。
		证实光的反射规律为:
		\begin{enumerate}[label=\roman*.]
			\item 产生部分偏振光。
			\item 必然产生线性偏振光,因为通过偏振片能消光。
			\item 线偏振光垂直于入射面(xOz)。
		\end{enumerate}
	\end{enumerate}




	\subsubsection{实验四 \quad 光的散射}

	实验材料:白光光源, 凸透镜,可变光阑,偏振片P,水槽,奶粉。

	具体实验步骤如下:
	\begin{enumerate}
		\item 在准直光束产生的平行光中放置水槽,水槽的面相平行。
		\item 倒入少许奶粉(不溢出) 。
		\item 用偏振片观察在(Ox)横向方向的散射光。通过水中倒入的少许粉末的作用,可观察到,在光传播垂直方向上,光(部分地) 在垂直方向(Oy)是线性偏振光。
	\end{enumerate}

	\begin{figure}[htbp]
		\centering
		\includegraphics[width=0.4\textwidth]{graph2-4-1.png}
		\caption{光的散射示意图}
		\label{fig:graph2-4-1}
	\end{figure}



	\subsubsection{实验五 \quad 分析$\lambda/2$波片特性}

	实验器材: 氦氖激光器,短焦距透镜,偏振片$P_1$,偏振片$P_2$, $\lambda/2$波片,小屏幕。

	光路示意图如\cref{fig:graph2-5-1}所示,具体实验步骤如下:
	\begin{enumerate}
		\item 将偏振片$P_1$和$P_2$放置约20厘米的距离,旋转偏振片找到消光位置。然后将$\lambda/2$波片放在两个偏振片之间:观看屏幕,旋转半波片,直到消光重新出现。证明这时有两个相互垂直方向出现消光现象。 这时,$\lambda/2$波片的o轴或e轴方向与偏振片$P_1$的偏振方向平行, 确定对应的位置。
		\item 旋转偏振片$P_1$ 20度。通过旋转偏振片$P_2$,重新出现消光现象。记录偏振片$P_2$的位置。证实通过$\lambda/2$半波片的作用,线性偏振光可产生对称o轴或e轴方向的线偏振光。
	\end{enumerate}


	\begin{figure}[htbp]
		\centering
		\includegraphics[width=0.6\textwidth]{graph2-5-1.jpg}
		\caption{分析$\lambda/2$波片特性光路示意图}
		\label{fig:graph2-5-1}
	\end{figure}


	\subsubsection{实验六 \quad 分析$\lambda/4$波片特性}

	实验材料: 氦氖激光器,短焦镜头,偏光片$P_1$,偏振片$P_2$,四分之一波片$\lambda/4$,小屏幕。

	光路示意图如\cref{fig:graph2-6-1}所示,具体实验步骤如下:
	\begin{enumerate}
		\item 确定两个偏振片的消光位置。
		\item 将$\lambda/4$波片放置在两个偏振片之间,观察屏幕。旋转$\lambda/4$波片,直到消光重新出现, 确定$\lambda/4$波片对应的位置。 说明这时有两个相互垂直方向出现消光现象。 证明此时$\lambda/4$波片的o轴或e轴方向与偏振片P1的偏振方向平行。
	\end{enumerate}


	\begin{figure}[htbp]
		\centering
		\includegraphics[width=0.6\textwidth]{graph2-6-1.jpg}
		\caption{分析$\lambda/4$波片特性光路示意图}
		\label{fig:graph2-6-1}
	\end{figure}


	\subsubsection{实验七 \quad 分析两个$\lambda/4$波片特性}

	实验材料 :氦氖激光器,短焦距透镜,光阑,两个偏振片,两个$\lambda/4$波片。

	光路示意图如\cref{fig:graph2-7-1}所示,具体实验步骤如下:
	\begin{enumerate}
		\item 确定两个偏振片的消光位置
		\item 放置第一个$\lambda/4$波片,重现消光现象。 记录消光位置。
		\item 再放入第二个$\lambda/4$波片, 出现消光时,记录消光的位置。说明这时两个$\lambda/4$波片的o轴或e轴相平行。
		\item 旋转$P_1$偏振片20°,标识旋转的方向。 然后旋转偏振片$P_2$重新达到消光,记录偏振片$P_2$的位置。说明此时两个$\lambda/4$波片的关系(即是两个波片的快慢轴一致?还是一个波片的快轴与另一个波片的慢轴重合?)。
		\item 旋转其中一个$\lambda/4$波片90度,再旋转偏振$P_2$重新达到消光,记录偏振片$P_2$的位置。说明此时两个$\lambda/4$波片的关系。
	\end{enumerate}


	\begin{figure}[htbp]
		\centering
		\includegraphics[width=0.6\textwidth]{graph2-7-1.jpg}
		\caption{分析两个$\lambda/4$波片特性光路示意图}
		\label{fig:graph2-7-1}
	\end{figure}


	\subsubsection{实验八 \quad 研究椭圆偏振光和圆偏振光}

	实验材料: 氦氖激光器,短焦距透镜,可变光阑,两个偏振片,四分之一波片$\lambda/4$。

	光路示意图如\cref{fig:graph2-8-1}所示,具体实验步骤如下:
	\begin{enumerate}
		\item 确定两个偏振片的消光位置
		\item 放置1个$\lambda/4$波片,重现消光现象。旋转$P_1$偏振器片45度,标识旋转方向。旋转偏光片$P_2$,观察屏幕上的光强变化,加以说明。
		\item 再旋转$P_1$偏振片的角度10度, 旋转$P_2$偏光片,观察屏幕上光强的变化。加以说明。

	\end{enumerate}
	
	\begin{figure}[htbp]
		\centering
		\includegraphics[width=0.6\textwidth]{graph2-6-1.jpg}
		\caption{研究椭圆偏振光和圆偏振光光路示意图}
		\label{fig:graph2-8-1}
	\end{figure}


%\subsection{实验数据记录}



\subsection{原始数据记录}

	见\cref{fig:data}

	\begin{figure}[htbp]
		\centering
		\subfloat[原始数据1]
		{\includegraphics[width=0.35\textwidth]{OriginalData1.jpg}\label{fig:data1}}
		\quad
		\subfloat[原始数据2]
		{\includegraphics[width=0.35\textwidth]{OriginalData2.jpg}\label{fig:data2}}
		\quad
		\subfloat[原始数据3]
		{\includegraphics[width=0.35\textwidth]{OriginalData3.jpg}\label{fig:data3}}
		\quad
		\subfloat[原始数据4]
		{\includegraphics[width=0.35\textwidth]{OriginalData4.jpg}\label{fig:data4}}
		\quad

		\caption{原始数据}
		\label{fig:data}
	\end{figure}



\subsection{实验过程中遇到的问题记录}

\begin{enumerate}
	\item 由于实验室仪器设备等原因,某些实验无法观察到消光现象。这时,可使用光功率计,将光强最小的位置定为消光位置。
	\item 由于光路简单,在没有要求元件之间距离的情况下,可以尽可能的将各元件靠近,以达到更好的观察效果。
	\item 在使用光功率计时,一定要让光斑完全覆盖功率计的探头,防止出现误差。
	
\end{enumerate}
	

\clearpage
\begin{table}
	\renewcommand\arraystretch{1.7}
	\begin{tabularx}{\textwidth}{|X|X|X|X|}
	\hline
	专业:& 物理学 &年级:& 2022级\\
	\hline
	姓名: & 戴鹏辉 & 学号:& 22344016\\
	\hline
    日期:& 2024/4/22 & 评分: &\\
	\hline
	\end{tabularx}
\end{table}

\section{CB2 \quad 偏振光实验 \quad\heiti 分析与讨论}

\subsection{实验数据分析}

	\subsubsection{实验一 \quad 研究交叉线性偏振片}

	\begin{enumerate}
		\item 只放置偏振片$P_1$在光路上时,旋转$P_1$方向,观察屏幕上光强的变化,\textbf{发现无明显变化},说明白光光源是\textbf{随机偏振}的。
		\item 偏振片$P_1$、$P_2$均放置在光路上时,首先将$\theta_{P1}=\ang{0}$,旋转$P_2$,观察屏幕上光强的变化,发现光强由亮到暗呈现周期性变化,且在$\theta_{P2}=\ang{22},\ang{201}$时出现消光现象,两个位置之间相差$\ang{180}$左右。
		
		同样的,若固定$\theta_{P2}=\ang{0}$,旋转$P_1$,观察屏幕上光强的变化,同样能观察到光强由亮到暗呈现周期性变化,且在$\theta_{P1}=\ang{156},\ang{344}$时出现消光现象,两个位置之间相差$\ang{180}$左右。

		说明当两个线偏振片的\textbf{光轴垂直}时,会出现消光现象。

		\item 首先设置$\theta_{P1}=\ang{344}$,$\theta_{p2}=\ang{0}$,此时出现消光现象。\\
			  然后旋转$P_1$$\ang{20}$至$\theta_{P1}=\ang{4}$,此时可观察到屏幕上有亮斑。\\
			  再旋转$P_2$$\ang{20}$至$\theta_{P2}=\ang{20}$,可观察到消光现象。

		\item 再在$P_1$和$P_2$之间加入第三个偏振片$P_3$,旋转$P_3$,发现在$\theta_{P_3}=\ang{196}$时出现消光现象。\\
			再旋转$P_3\ang{45}$,发现屏幕上出现光斑;再旋转$P_3\ang{45}$,可观察到最亮的光斑。说明线偏振光再次经过线偏振片后会\textbf{改变偏振方向},具体的偏振方向由$P_3$决定。
	\end{enumerate}




	\subsubsection{实验二 \quad 验证马吕斯定律}

	由马吕斯定律出发,使用公式$I(\theta)=I_0\cos^2(\theta+\phi_0)+C$,对实验所测数据进行拟合,结果如\cref{fig:graph3-2}所示。

	\begin{figure}[htbp]
		\centering
		\subfloat[白光光源图像]
		{\includegraphics[width=0.8\textwidth]{graph3-2-1.png}\label{fig:graph3-2-1}}
		\quad
		\subfloat[绿光光源图像]
		{\includegraphics[width=0.8\textwidth]{graph3-2-2.png}\label{fig:graph3-2-2}}
		\quad
		\subfloat[红光光源图像]
		{\includegraphics[width=0.8\textwidth]{graph3-2-3.png}\label{fig:graph3-2-3}}
		\quad

		\caption{验证马吕斯定律图像}
		% \caption{原始数据}
		\label{fig:graph3-2}
		
	\end{figure}

	由拟合数据可知,光强确实满足马吕斯定律$I(\theta)=I_0\cos^2(\theta+\phi_0)+C$。且红光光强大于绿光光强。
	
	常数$C$的存在是因为即使是在消光时,仍然存在环境光的影响;常数$\phi_0$的存在是因为偏振片的示数角度为0时不一定是光轴垂直地面的情况。
			



	\subsubsection{实验三 \quad 玻璃反射起偏和布儒斯特角的测量}

	\begin{enumerate}
		\item 通过旋转偏振片,观察教室的天花板的日光灯管光强的变化,发现\textbf{亮度无明显变化},说明日光灯是随机偏振的。
		\item 用偏振片观察在地面瓷砖上的反射像,旋转偏振片会发现亮度会由亮到暗变化,说明经过反射后,反射光的平行分量和垂直分量的强度不相等。旋转偏振片,未完全消光,说明此时产生的是\textbf{部分偏振光}。
		\item 通过改变观察位置和偏振片角度,可找到一个完全消光的位置,说明产生的一定是\textbf{线性偏振光}。这个位置对应的入射角即布儒斯特角。利用红外线测距仪可测量观察点到地面像的距离$L$和观察点距离地面的高度$h$,可计算出布儒斯特角。
	\end{enumerate}
	
	

	


	\begin{figure}[htbp]
		\centering
		\includegraphics[width=0.4\textwidth]{graph3-3-1.jpg}
		\caption{布儒斯特角测量光路图}
		\label{fig:graph3-3-1}
	\end{figure}

	由\cref{fig:graph3-3-1},只需测量$L$和$h$即可计算出地面瓷砖对应的布儒斯特角。
	\[
	\theta_b=\arccos(\frac{h}{L})=\frac{1.608m}{2.813m}=\ang{55.14}	
	\]
	
	若取地面瓷砖的折射率为$n=1.5$,则理论值大约为$\ang{56}$,相对误差为$\eta=\frac{56-55.14}{56}=1.536\%$。



	\subsubsection{实验四 \quad 光的散射}

	\begin{enumerate}
		\item 从侧面用偏振片观察水箱,发现亮度降低,旋转偏光片后,发现亮度会由亮到暗的变化。说明经过水箱中的奶粉散射到\textbf{侧面}后,\textbf{原来的随机偏振光变成了线偏振光}。
		\item 从正面用偏振片观察水箱,旋转偏光片发现亮度无明显变化,说明正面的出射光仍然是\textbf{随机偏振光}。
	\end{enumerate}
	
	





	\subsubsection{实验五 \quad 分析$\lambda/2$波片特性}

	\begin{enumerate}
		\item 不放置$\lambda/2$波片,只放置两个偏振片时,可以在$\theta_{P1}=\ang{0},\theta_{P2}=\ang{17},\ang{197}$观察到消光现象。	\\
			在$P_1,P_2$之间插入$\lambda/2$波片,保持两个偏振片角度不变,旋转$\lambda/2$波片,可以在$\theta_{\lambda/2}=\ang{83},\ang{174},\ang{263},\ang{354}$观察到消光现象,四个角度间隔$\ang{90}$。
		\item 旋转$P_1\ang{20}$度后,保持$\lambda/2$波片角度不变,旋转$P_2$,在$\theta_{\lambda/2}=\ang{358},\ang{178}$观察到消光现象。即$P_1,P_2$二者\textbf{反向转动},说明线偏振光经过$\lambda/2$波片后,\textbf{相对0轴或e轴对称了}。
	\end{enumerate}
	







	\subsubsection{实验六 \quad 分析$\lambda/4$波片特性}

	\begin{enumerate}
		\item 不放置$\lambda/4$波片,只放置两个偏振片时,可以在$\theta_{P1}=\ang{0},\theta_{P2}=\ang{17},\ang{197}$观察到消光现象。	\\
		
		在$P_1,P_2$之间插入$\lambda/4$波片,保持两个偏振片角度不变,旋转$\lambda/4$波片,可以在$\theta_{\lambda/2}=\ang{359},\ang{270},\ang{180},\ang{90}$观察到消光现象,四个角度间隔$\ang{90}$。说明$\lambda/4$波片存在两个\textbf{互相垂直}的方向出现消光现象,且在消光现象时$\lambda/4$波片的\textbf{o轴或e轴方向与偏振片$P_1$的偏振方向平行}。

		
	\end{enumerate}







	\subsubsection{实验七 \quad 分析两个$\lambda/4$波片特性}

	\begin{enumerate}
		\item  单独放置两个偏振片时,可以在$\theta_{P1}=\ang{0},\theta_{P2}=\ang{17}$观察到消光现象。此时在两个偏振片之间加入一个$\lambda/4$波片,在$\theta_{\lambda/4}=\ang{270}$是可观察到消光现象;加入第二个$\lambda/4$波片,并旋转,可在$\theta^2_{\lambda/4}=\ang{100},\ang{190},\ang{281},\ang{11}$观察到消光现象。\\
		
			经过上面的操作后,两个$\lambda/4$波片的位置关系,\textbf{有可能是两个快轴重合,或者一个快轴和另一个慢轴重合}。

		\item 正向旋转偏振片$P_1\ang{20}$,再旋转$P_2$,可在$\theta_{P2}=\ang{218},\ang{38}$观察到消光现象,且两个偏振片是同向转动的。说明此时的两个$\lambda/4$波片的位置关系为“\textbf{一个快轴和另一个慢轴重合}”,\textbf{两个$\lambda/4$波片的相移抵消了}。
		\item 先旋转其中一个$\lambda/4$波片$\ang{90}$后,反向旋转偏振片$P_1\ang{20}$,再旋转$P_2$,可在$\theta_{P2}=\ang{214},\ang{34}$观察到消光现象,且两个偏振片是反向转动的。说明此时的两个$\lambda/4$波片的位置关系为“\textbf{两个快轴重合}”,两个$\lambda/4$波片的相移叠加后,等效为一个\textbf{$\lambda/2$波片}。
	\end{enumerate}








	\subsubsection{实验八 \quad 研究椭圆偏振光和圆偏振光}

	\begin{enumerate}
		\item 单独放置两个偏振片时,可以在$\theta_{P1}=\ang{0},\theta_{P2}=\ang{197}$观察到消光现象。此时在两个偏振片中间加入一个$\lambda/4$波片,在$\theta_{\lambda/4}=\ang{280}$是可观察到消光现象。	
		\item 此时旋转$\lambda/4$波片$\ang{45}$后,旋转$P_2$偏振片,再次观察屏幕上的光强,发现无明显变化。因为线偏振光以$\ang{45}$通过$\lambda/4$波片会变成\textbf{圆偏振光},而圆偏振光对于线偏振片是各向同性的,所以经过$P_2$偏振片后的光强与$P_2$的角度无关,保持不变。
		\item 此时在旋转$P_1\ang{10}$,旋转$P_2$偏振片,再次观察屏幕上的光强,发现光强由亮到暗呈现周期性变化,但不会出现消光现象。因为此时经过$P_1$的线偏振光与$\lambda/4$波片的光轴不是$\ang{45}$而是$\ang{55}$,经过$\lambda/4$波片的光为\textbf{椭圆偏振光},再次经过线偏振片后的光强与角度有关。
	\end{enumerate}

	









			
			
\subsection{实验后思考题}

\begin{question}
	通过实验现象, 证实光的反射规律为:
	\begin{enumerate}
		\item 产生部分偏振光。
		\item 必然产生线性偏振光。
	\end{enumerate}

\end{question}

	在实验三“\textbf{玻璃反射起偏和布儒斯特角的测量}”中,用偏振片观察在地面瓷砖上的反射像,发现亮度变暗,旋转偏振片,未完全消光,说明此时产生的是\textbf{部分偏振光}。

	在测量布儒斯特角时,可以通过改变观察位置和偏振片角度,找到一个完全消光的位置,说明产生的一定是\textbf{线性偏振光}。





\begin{question}
	说明用偏振片观察透过水中倒入的少许粉末后的现象产生的原因。
\end{question}
	


在白光光源照射的水箱中导入了少许粉末后,水中的粉末会导致散射。散射是光与物质相互作用后改变方向的过程。粉末颗粒的尺寸与光的波长相近时,散射会更加明显。在这种情况下,观察到的现象可能是:

	\begin{enumerate}
		\item 散射光的偏振现象: 偏振片可以使特定偏振状态的光通过,因此如果散射光的偏振状态与偏振片的方向相匹配,那么这些散射光就能通过偏振片而观察到。
		\item 偏振光强度的变化: 如果粉末引起的散射光的偏振状态与偏振片的方向不匹配,那么这些散射光将被偏振片吸收或减弱,观察到的散射光强度可能会减弱或消失。
		因此,通过观察散射光在偏振片上的表现,可以了解到水箱中的粉末对光的散射特性,以及散射光的偏振状态。
	\end{enumerate}






\begin{question}
	请说明 3D 眼镜和 3D 电影的原理。
\end{question}


3D眼镜和3D电影的原理是基于人类双眼视觉和光的偏振特性。从光学角度来看,3D电影通过模拟人眼观察真实世界时的立体视觉来产生深度感和立体效果。

首先,人类的立体视觉是因为我们有两只眼睛,它们从略微不同的角度观察物体,每只眼睛接收到的图像都有细微的差异。大脑将这两个图像合成,使我们能够感知深度和立体空间。为了在电影中重现这种效果,3D电影使用了两台摄像机从两个略微不同的角度同时拍摄场景,模拟人的左眼和右眼所看到的画面。

在放映时,这两个角度的画面会同时投射到银幕上。如果没有适当的设备,观众会看到两个重叠的图像,这会导致画面模糊和视觉混乱。为了解决这个问题,3D眼镜被引入到观影过程中。

3D眼镜通常使用偏振光技术。自然光包含多个方向上的振动,而偏振光则只在一个特定方向上振动。3D眼镜的每片镜片都是偏振片,它们以特定的方向过滤光线。例如,左眼镜片可能只允许水平方向的偏振光通过,而右眼镜片则只允许垂直方向的偏振光通过。这样,左眼只能看到为左眼设计的画面,右眼只能看到为右眼设计的画面。

在电影院中,放映机通过偏振滤光片投射两个不同角度的画面,这些画面经过偏振处理,以确保每只眼睛只接收到为其设计的画面。当观众戴上3D眼镜时,由于偏振片的过滤作用,左眼和右眼分别接收到两个不同的图像,从而在大脑中合成为一个立体的场景。

这就是3D眼镜和3D电影的基本原理。通过这种方式,3D电影能够提供一种沉浸式的观影体验,让观众感觉到物体仿佛从屏幕中跳出来,进入到现实空间中。



\begin{figure}[H]
	\centering
	\includegraphics[width=0.5\textwidth]{table.jpg}
	\caption{实验桌整理后照片}
\end{figure}
	

\end{document}
