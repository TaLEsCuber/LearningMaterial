%!TEX program = xelatex
\documentclass[dvipsnames, svgnames,a4paper,11pt]{article}
% ----------------------------------------------------
%   中山大学物理与天文学院本科实验报告模板
%   作者:Huanyu Shi,2019级
%   知乎:https://www.zhihu.com/people/za-ran-zhu-fu-liu-xing
%   Github:https://github.com/huanyushi/SYSU-SPA-Labreport-Template
%   Last update : 2023.4.10
% ----------------------------------------------------

\input{Settings} % 导入模板的相关设置
\usepackage{lipsum}
\usepackage{enumitem}
\usepackage{tabularray}  %绘制表格时可以更加方便添加框线
\usepackage{listings}
\usepackage{color}
\setlist[enumerate]{label=\textup{(\arabic*)}}



%---------------------------------------------------------------------
%	正文
%---------------------------------------------------------------------

\begin{document}


\begin{table}
	\renewcommand\arraystretch{1.7}
	\begin{tabularx}{\textwidth}{
		|X|X|X|X
		|X|X|X|X|}
	\hline
	\multicolumn{2}{|c|}{预习报告}&\multicolumn{2}{|c|}{实验记录}&\multicolumn{2}{|c|}{分析讨论}&\multicolumn{2}{|c|}{总成绩}\\
	\hline
	\LARGE25 & & \LARGE30 & & \LARGE25 & & \LARGE80 & \\
	\hline
	\end{tabularx}
\end{table}


\begin{table}
	\renewcommand\arraystretch{1.7}
	\begin{tabularx}{\textwidth}{|X|X|X|X|}
	\hline
	专业:& 物理学 &年级:& 2022级\\
	\hline
	姓名:& 戴鹏辉\&崔瑜  & 学号: & 2344016\&22344017 \\
	\hline
	日期:& 2024/6/3 & 教师签名:& \\
	\hline
	\end{tabularx}
\end{table}

\begin{center}
	\LARGE D4  \quad 塞曼效应
\end{center}

\textbf{【实验报告注意事项】}
\begin{enumerate}
	\item 实验报告由三部分组成:
	\begin{enumerate}
		\item 预习报告:(提前一周)认真研读\underline{\textbf{实验讲义}},弄清实验原理;实验所需的仪器设备、用具及其使用(强烈建议到实验室预习),完成课前预习思考题;了解实验需要测量的物理量,并根据要求提前准备实验记录表格(第一循环实验已由教师提供模板,可以打印)。预习成绩低于10分(共20分)者不能做实验。
	    \item 实验记录:认真、客观记录实验条件、实验过程中的现象以及数据。实验记录请用珠笔或者钢笔书写并签名(\textcolor{red}{\textbf{用铅笔记录的被认为无效}})。\textcolor{red}{\textbf{保持原始记录,包括写错删除部分,如因误记需要修改记录,必须按规范修改。}}(不得输入电脑打印,但可扫描手记后打印扫描件);离开前请实验教师检查记录并签名。
	    \item 分析讨论:处理实验原始数据(学习仪器使用类型的实验除外),对数据的可靠性和合理性进行分析;按规范呈现数据和结果(图、表),包括数据、图表按顺序编号及其引用;分析物理现象(含回答实验思考题,写出问题思考过程,必要时按规范引用数据);最后得出结论。
	\end{enumerate}
	\textbf{实验报告就是将预习报告、实验记录、和数据处理与分析合起来,加上本页封面。}
	\item 每次完成实验后的一周内交\textbf{实验报告}(特殊情况不能超过两周)。
	\item 实验报告注意事项
		\begin{enumerate}[label=\roman*.]
			\small % 设置字体大小为 large
			\item 不要用手直接触摸光学部件的光学表面,如果有污染,请用专用的清洁液和清洁布或者擦镜纸清洁。
			\item F-P 标准具已在出厂前进行校准,请勿自行随意调节三颗螺钉,如需调整,请在实验老师指导下进行。
			\item 为避免缩短笔形汞灯使用寿命,请勿频繁开关电源开关。笔形汞灯插入槽内后严禁左右晃动,以免破碎。
			\item 避免长时间对电磁线圈进行大电流供电,温度较高,请勿触摸。
			\item 每次实验结束后将电流调至零后再关掉电源。
		\end{enumerate}
\end{enumerate}


\clearpage
\tableofcontents
\clearpage

\setcounter{section}{0}
\section{D4 \quad 塞曼效应 \quad\heiti 预习报告}
	
\subsection{实验目的}

	\begin{enumerate}
		\item 通过观察原子谱线在外磁场中的分裂现象,加深对电子“自旋”、“两个角动量的耦合”、“两个电子之间的 LS 耦合”、“角动量守恒”、“多电子原子和电子组态”、“能级跃迁”、“选择定则”等概念的理解,验证原子具有磁矩及其空间取向的量子化,进一步认识原子的内部结构;
		\item 用 F-P 干涉仪观察汞(Hg)原子 546.1nm 谱线在外磁场中的分裂现象(即“反常塞曼效应”),测量电子的荷质比 e/m;
		\item 验证光子具有角动量及角动量守恒定律,了解光的偏振理论及其产生机制,学习偏振片的原理及使用方法。
	\end{enumerate}

\subsection{仪器用具}
\begin{table}[htbp]
    \centering
    \renewcommand\arraystretch{1.6}
    \begin{tabular}{p{0.05\textwidth}|p{0.20\textwidth}|p{0.05\textwidth}|p{0.5\textwidth}}
        \hline
        编号 & 仪器用具名称 & 数量 & 主要参数(型号,测量范围,测量精度等) \\
        \hline
        1 & 塞曼效应 & 1 & BEX-8501 \\
        2 & 电磁体(带电源) & 1 & 可通过调节激励电流大小改变磁场的大小 \\
        3 & \textcolor{red}{磁感应强度测量仪} & 1 & \textcolor{red}{实验中通过电流与磁感应强度的对应关系得到磁感应强度值} \\
        4 & 汞放电管 & 1 & \\
        5 & 滤光片 & 1 & 能通过 546.1nm 及其自由光谱范围内的“单色光” \\
        6 & 法布里-珀罗标准具 & 1 &  \\
        7 & 偏振片 & 1 & 可以通过 0° 刻度线方向上的线偏振光\\
        8 & 数字图像采集器 & 1 & 包含望远镜和 CCD \\
        9 & 台式电脑 & 1 & 安装有“塞曼效应实验分析 VCH5.0”软件,用以分析实验数据 \\
        \hline
    \end{tabular}
\end{table}

\subsection{原理概述}


        
    \subsubsection{电子的磁矩——轨道磁矩和自旋磁矩}

		\begin{enumerate}
			\item 电子的轨道磁矩

				\[
					\mu_l = \sqrt{l(l+1)}\frac{e \hbar}{2 m_e} \equiv \sqrt{l(l+1)}\mu_B \quad l=0,1,2,\cdots , n-1	
				\]

				\[
					\mu_{l,z} = m_l \hbar \cdot \frac{e}{2 m_e} = m_l \mu_B \quad m_l = 0,\pm 1,\cdots ,\pm l	
				\]

				其中$\mu_B \equiv \frac{e \hbar}{2 m_e} =0.9274 \times 10^{23} A\cdot m^2$


				% \begin{figure}[htbp]
				% 	\centering
				% 	\includegraphics[width=0.8\textwidth]{graph1-1.png}
				% 	\label{fig:graph1-1}
				% \end{figure}

			\item 电子的自旋 及自旋磁矩
			
				电子除了轨道角动量外,还有自旋运动,电子的自旋是一种固有的内在属性,电子自旋角动量及其 z 方向分量大小的可能取值为:

				\[
					S = \sqrt{s(s+1)}\hbar \quad s = \frac{1}{2}	
				\]

				\[
					S_z = m_s \hbar \quad m_s = \pm \frac{1}{2}	
				\]


					% \begin{figure}[htbp]
					% 	\centering
					% 	\includegraphics[width=0.8\textwidth]{graph1-2.png}
					% 	\label{fig:graph1-2}
					% \end{figure}

				由电子自旋运动产生的自旋磁矩,其方向及大小为:

				\[
					\vec{\mu_s} = -2 \gamma \vec{S}
				\]

				\[
					\mu_s = 2\sqrt{s(s+1)}\mu_B \quad s = \frac{1}{2}
				\]

				\[
					\mu_{s,z} = 2 m_s \mu_B \quad m_s = \pm \frac{1}{2}
				\]


					% \begin{figure}[htbp]
					% 	\centering
					% 	\includegraphics[width=0.8\textwidth]{graph1-3.png}
					% 	\label{fig:graph1-3}
					% \end{figure}

			\item 磁矩在外磁场 中的势能
					
				在电磁学中,磁矩为的系统,在外磁场中的势能为:
				\[
					U = - \vec{\mu} \cdot \vec{B}
				\]	
				
					% \begin{figure}[H]
					% 	\centering
					% 	\includegraphics[width=0.8\textwidth]{graph1-4.png}
					% 	\label{fig:graph1-4}
					% \end{figure}

					
		\end{enumerate}


	\subsubsection{两个角动量的耦合(叠加)}

		\begin{enumerate}
			\item 同一个电子的情况

				同一个电子,其轨道角动量L和自旋角动量S的叠加或耦合。此即所谓的“自旋—轨道耦合”。自旋—轨道耦合本质上是电子自旋(内禀)磁矩与电子轨道运动产生的磁场之间的相互作用,“自旋—轨道耦合”是导致氢原子和碱金属光谱双线结构的主要原因:

					\[
						\vec{J}=\vec{L}+\vec{S}	
					\]

			\item 不同电子之间的耦合

				\begin{itemize}
					\item LS 耦合:即电子1的$L_1$和电子2的$L_2$耦合成$\vec{L}=\vec{L_1}+\vec{L_2}$,电子1的$S_1$和电子2的$S_2$耦合成$\vec{S}=\vec{S_1}+\vec{S_2}$,然后系统的总轨道角动量$\vec{L}$和总自旋角动量$\vec{S}$再耦合生成系统的总角动量$\vec{J}$
					
					\item jj耦合:即电子1的$L_1$和$S_1$耦合成$\vec{J_1}=\vec{L_1}+\vec{S_1}$,即电子2的$L_2$和$S_2$耦合成$\vec{J_2}=\vec{L_2}+\vec{S_2}$,然后电子1的总角动量$\vec{J_1}$和电子2的总角动量$\vec{J_2}$再耦合成系统的总动量$\vec{J}$。
				\end{itemize}

			\item 多电子原子及电子组态
			
				\begin{figure}[H]
					\centering
					\includegraphics[width=0.8\textwidth]{graph1-5.png}
					\label{fig:graph1-5}
				\end{figure}
				
				
		\end{enumerate}



	\subsubsection{电子的总磁矩 及其在外磁场中的势能}

		\begin{enumerate}
			\item 电子的总磁矩
			
				总磁矩$\vec{\mu_j}$及其大小的一般表达式可写成:
					\[
						\mu_j = -\sqrt{j(j+1)}g_j \mu_B
					\]

				总磁矩$\vec{\mu_j}$在z方向上分量的大小为:
					\[
						\mu_{j,z} = - m_j g_j \mu_B	
					\]

				朗德因子的一般表达式为:
					\[
						g_j=\frac{3}{2}+\frac{s\left(s+1\right)-l\left(l+1\right)}{2j\left(j+1\right)}	
					\]

			\item 磁矩在均匀外磁场 中的运动——拉莫尔进动
			
				磁矩在均匀外磁场中不受力,但是会受到一个力矩的作用:
				\[
					\vec{\tau} = \vec{\mu_J} \cross \vec{B}	
				\]

				由经典力学的知识可知,力矩会引起系统总角动量的变化:
				\[
					\frac{d \vec{J}}{d t} = \vec{\tau} = \vec{\mu_J} \cross \vec{B}
				\]
				\[
					\frac{d \vec{\mu_J}}{d t} = \vec{\omega} \cross \vec{\mu} \quad \vec{\omega} \equiv \gamma \vec{B}
				\]
				
				式中的$\vec{\omega}$称为“拉莫尔频率”,$\vec{\mu_J}$大小不变,绕着外磁场B(z 轴方向)做进动,又称“拉莫尔进动”。

			\item 弱外磁场和强外磁场的区别
			
				\begin{itemize}
					\item 弱外磁场:当外磁场$B_{ext}$的场强远远小于原子内部磁场$B_{int}$的场强,即为弱外磁场时,这时自旋—轨道耦合占主导,电子的总角动量J是守恒量,但总轨道角动量L和总自旋角动量S均不守恒, 系统的守恒量(好量子数)为$J^2,J_z,L^2,S^2\left(n,\ L,\ J,\ mj\right)。$
					\item 强外磁场: 自旋—轨道耦合变得次要,电子磁矩(分为轨道磁矩和自旋磁矩)与外磁场$B_{ext}$的相互作用则占主导。系统的总轨道角动量L和总自旋角动量S不再耦合成总角动量J(仍然可通过$\vec{J}=\vec{L}+\vec{S}$计算总角动量,但此时的总角动量是非守恒量, (24)至(28)式在强外磁场的情况下不适用),此时系统的守恒量(好量子数)为$L^2,L_Z,\ \ S^2,\ {\ S}_z\left(n,\ L,\ ml,\ ms\right)$。
				\end{itemize}
			
			\item 电子总磁矩 在外磁场 中的势能
			
				可得汞原子的总磁矩$\vec{\mu_j}$在沿着 z 轴方向的外磁场B中的势能为:
				\[
					U=-\mu_J \cdot B=m_jg_j\mu_B B	
				\]


		\end{enumerate}





	\subsubsection{汞原子谱线在外磁场 中的能级分裂}

		\begin{enumerate}
			\item 原子谱线在外磁场中的分裂

				当原子处于磁感应强度为B的外磁场中时,原子磁矩与磁场相互作用,使原子系统附加了势能$\Delta E=U$,(由(33)式给出),此时原子的能级将分裂为2J+1层,各层能量为:
				\[
					E = E_0 +m_J g_J \mu_B B
				\]

				又设频率为$\nu$的光子是由原子从能级E2跃迁到能级E1所产生,即有:
				\[
					h\nu = E_2 - E_1 = E_{20} - E_{10} + (m_2 g_2 -m_1 g_1)\mu_B B	
				\]

				分裂后的谱线和原来谱线之间的频率差为:
				\[
					\Delta \nu = (m_2 g_2 -m_1 g_1)\mu_B B / h
				\]

				波数差(波数定义$\widetilde{\upsilon}=1/\lambda$):
				\[
					\Delta \tilde{\nu} = \frac{(m_2 g_2 - m_1 g_1)eB}{4 \pi m c} 	
				\]

			\item 汞原子 546.1nm 谱线的量子数及朗德因子 的计算
			
				谱线是汞原子的外层(两个)电子从能级(组态)6s7s到6s7p的跃迁而产生的。 

					\begin{figure}[H]
						\centering
						\includegraphics[width=0.8\textwidth]{graph1-6.png}
						\label{fig:graph1-6}
					\end{figure}

					\begin{figure}[H]
						\centering
						\includegraphics[width=0.8\textwidth]{graph1-7.png}
						\label{fig:graph1-7}
					\end{figure}
				
			\item 塞曼效应观测中的偏振效应——$\sigma^\pm$偏振(垂直于磁场)和$\pi$偏振(平行于磁场)
			
				若有$\Delta m = 1$,将观察到$\sigma^+$左旋偏振光,若$\Delta m = -1$,将观察到$\sigma^-$右旋偏振光,若$\Delta m = 0$,则为$\pi$偏振。
		\end{enumerate}	





	\subsubsection{基本测量仪器介绍}

		\begin{enumerate}
			\item 法布里—珀罗标准具

				F-P 标准具由平行放置的两块平面板组成的, 在两板相对的平面上镀薄银膜和其他有较高反射系数的薄膜, 若两平行的镀银平面的间隔不可以改变,则称该仪器为“法布里—珀罗干涉仪”。由于两镀膜面平行,若使用扩展光源,则产生等倾干涉条纹。具有相同入射角的光线在垂直于观察方向的平面上的轨迹是一组同心圆。若在光路上放置透镜,则在透镜焦平面上得到一组同心圆环图样。 在透射光束中,相邻光束的光程差为:
				\[
					\Delta = 2 n d cos \phi	
				\]

				产生亮条纹的条件为:
				\[
					2 n d cos \phi = K \lambda
				\]

				其中K为干涉条纹级数,$\lambda$为入射光波长。

				标准具的两个特征参量是自由光谱范围和分辨本领。
					
					\begin{itemize}
						\item 自由光谱范围:同一光源发出的具有微小波长差的单色光$\lambda_1$和$\lambda_2$,入射后将形成各自的圆环系列。对同一级的干涉条纹,波长大的干涉圆环直径小。对于某单色光来说,越靠中心处的干涉条纹的级数 K 越大。
						
						如果$\lambda_1$和$\lambda_2$的波数差逐渐增大,使得$\lambda_1$的第$m$级亮纹与$\lambda_2$的第$m - 1$级亮纹重合,则有
						\[
							2 n d \cos \theta = m \lambda _1 = (m - 1)\lambda_2	
						\]

						得出$\Delta \lambda = \lambda _2 - \lambda_1 = \frac{\lambda_2}{m}$

						由于大多数情况下,$\cos \theta \approx 1$,可得到$m = \approx \frac{2 n d}{\lambda_1}$,得
						\[
							\Delta \lambda = \frac{\lambda_1 \lambda_2}{2 n d} \approx \frac{\lambda^2}{2 n d}	
						\]
						
							% \begin{figure}[H]
							% 	\centering
							% 	\includegraphics[width=0.8\textwidth]{graph1-8.png}
							% 	\label{fig:graph1-8}
							% \end{figure}

						它表明在 F-P 干涉仪中,当给定两平面间隔d后,入射光波长在$\lambda \pm \Delta \lambda$内不会发生干涉圆环重叠。

						\item 分辨本领:对于F-P标准具,它的分辨本领为$\frac{\Delta \lambda}{\lambda} = K N$,K为干涉级次,N为精细度, 它的物理意义是在相邻两个干涉级之间能分辨的最大条纹数。N依赖于平板内表面反射膜的反射率R:
						\[
							N = \frac{\pi \sqrt{R}}{1 -1R}	
						\]

						反射率R越高,精细度N就越高,仪器能分辨开的条纹数就越多。

					\end{itemize}

			\item 波数差与圆环直径之间的关系
			
				通过测量干涉环的直径就可以测量各分裂谱线的波长或波长差。

				如\cref{fig:graph1-9}所示,出射角为$\theta$的的圆环直径D与透镜焦距f间的关系为$\tan{\theta}=\frac{D}{2f}$,对于近中心的圆环$\theta$很小,可以认为$\theta\approx\sin{\theta}\approx\tan{\theta}$,于是有:
				\[
					cos\theta = 1 - 2 \sin^2\frac{\theta}{2} \approx 1 - \frac{\theta^2}{2} = 1 - \frac{D^2}{8f^2}	
				\]

				\begin{figure}[H]
					\centering
					\includegraphics[width=0.8\textwidth]{graph1-9.png}
					\caption{法布里一珀罗(F-P)标准具光路图(图中的透镜指的是望远镜里的透镜)}
					\label{fig:graph1-9}
				\end{figure}

				\[
					2 n d \cos \theta = 2 n d (1 - \frac{D^2}{8f^2}) = K \lambda	
				\]

				由上式可推出同一波长$\lambda$相邻两级$K$和$K - 1$级圆环直径的平方差为
				\[
					\Delta D^2 = D^2_{K-1} - D^2_{K} = \frac{4 f^2 \lambda}{n d}	
				\]

				设不同波长$\lambda$和$\lambda_{\alpha}$的第K级干涉圆环直径分别为$D_K$和$D_{\alpha}$
				\[
					\lambda - \lambda_\alpha = \frac{nd}{4 f^2 K}(D^2_\alpha - D^2_K) = (\frac{D^2_\alpha - D^2_K}{D^2_{K-1} - D^2_{K}})\frac{\lambda}{K}	
				\]

				得出波长差为:
				\[
					\Delta \lambda = \frac{\lambda^2}{2 n d}(\frac{D^2_\alpha - D^2_K}{D^2_{K-1} - D^2_{K}})
				\]
				
				波数差为:
				\[
					\Delta \tilde{\nu} = \frac{1}{2 n d}(\frac{D^2_\alpha - D^2_K}{D^2_{K-1} - D^2_{K}})
				\]

			\item 通过塞曼效应计算电子荷质比e/m
			
				本实验只研究反常塞曼效应的线偏振$\pi$的谱线。
				\[
					\frac{e}{m} = \frac{4\pi c}{n d B}(\frac{D^2_\alpha - D^2_K}{D^2_{K-1} - D^2_{K}})
				\]
				
				若已知$d$和$B$,从塞曼分裂中测量出各环的直径,就可以计算出荷质比。
		\end{enumerate}

   

        



\clearpage

\subsection{实验前思考题}



% 思考题1
\begin{question}
	光子是否具有角动量?试描述光子角动量方向与光的偏振方向之间的关系。
\end{question}

	光子具有角动量,这是由光的波粒二象性决定的。光子的角动量可以分为两部分:轨道角动量和自旋角动量。

		\begin{itemize}
			\item 光子的自旋角动量
			
				光子的自旋角动量与光的偏振状态密切相关。光子的自旋角动量量子数为±1,对应光子的左旋圆偏振和右旋圆偏振。具体来说:

				\begin{itemize}
					\item 左旋圆偏振光:光子的自旋角动量沿传播方向(z轴)为 +ħ。
					\item 右旋圆偏振光:光子的自旋角动量沿传播方向(z轴)为 -ħ。
					\item 对于线偏振光,光子的自旋角动量可以被看作是左旋和右旋圆偏振光子的叠加,因此没有净自旋角动量。
				\end{itemize}
			
			\item 光子的轨道角动量
			
				光子的轨道角动量与光束的空间结构相关,例如涡旋光束(光学涡旋)具有轨道角动量。这些光束的电场和磁场分布呈螺旋状,光子的轨道角动量与光束的涡旋态有关。
		\end{itemize}


	光子的自旋角动量方向与光的偏振方向有密切关系。对于线偏振光,偏振方向就是电场振动的方向,而对于圆偏振光,电场矢量在垂直于传播方向的平面内旋转。
		
		\begin{itemize}
			\item 线偏振光:自旋角动量为零,因为它是左右旋圆偏振光的叠加。
			\item 圆偏振光:自旋角动量沿传播方向,为±ħ,对应于电场矢量在传播方向上方的旋转方向。
		\end{itemize}
	
	对于具有轨道角动量的光(如涡旋光束),轨道角动量方向与光束的螺旋相位前沿的旋转方向相关。


	光子具有角动量,包括自旋角动量和轨道角动量。自旋角动量直接与光的偏振状态相关,而轨道角动量与光束的空间相位结构相关。光子的自旋角动量方向与圆偏振光的旋转方向一致,而线偏振光没有净自旋角动量。理解光子的角动量对于光与物质相互作用、光通信以及量子信息等领域具有重要意义。





% 思考题2
\begin{question}
	用同一级条纹的内外圈分别计算电子的荷质比,结果一样吗?试简述原因。
\end{question}


	F-P标准具的干涉条纹由于光束的多次反射和干涉而形成。对于不同位置的条纹,由于光程差不同,条纹的直径也不同,具体地说,内圈条纹和外圈条纹的厚度存在差异。

	假设我们有两个同一级条纹,其中内圈直径为$D_1$,外圈直径为$D_2$。

	对于同一级条纹,若我们分别用内圈直径$D_1$和外圈直径为$D_2$计算电子的荷质比$\frac{e}{m}$,结果会有所不同。这是因为圆环并不是等厚的。随着条纹级数$N$,的增加,圆环的厚度变化$(\delta D)$会影响直径的测量。具体来说,随着级数的增加,圆环的厚度变化导致直径的变化,从而使得计算的结果有所偏差。




% 思考题3
\begin{question}
	请利用(20)至(23)式,计算汞原子 $^3S_1$(6s7s)和 $^3P_2$(6s7p)能级所对应的量子数(见表 1),并给出详细的计算过程。
\end{question}

	\begin{enumerate}
		\item 先计算 $^3S_1$的情况:
		
			\begin{itemize}
				\item 由$2S + 1 = 3$,得$S = 1$;
				\item 处在S轨道,则$L = 0$
				\item $J = 1$
			\end{itemize}

		\item 再计算$^3P_2$的情况:
			
			\begin{itemize}
				\item 由$2S + 1 = 3$,得$S = 1$;
				\item 处在P轨道,则$L = 1$
				\item $J = 2$
			\end{itemize}
	\end{enumerate}


% 思考题4
\begin{question}
	请利用(2)、(8)和(20)式,并结合$\vec{J} = \vec{L} + \vec{S}$ 和 $\vec{\mu}_J = \vec{\mu}_L + \vec{\mu}_S $(注意此时的是图 5 中的$\vec{\mu}_J$ ,详细见脚注 22),导出朗德因子的一般表达式(28)式,并给出详细的推导过程。
\end{question}

	

	% 首先,总角动量 \( \mathbf{J} \) 是轨道角动量 \( \mathbf{L} \) 和自旋角动量 \( \mathbf{S} \) 的矢量和:

	% 	\[
	% 	\mathbf{J} = \mathbf{L} + \mathbf{S}
	% 	\]


	% 电子的轨道磁矩 \( \boldsymbol{\mu}_L \) 与轨道角动量 \( \mathbf{L} \) 相关:

	% 	\[
	% 	\boldsymbol{\mu}_L = -\frac{e}{2m_e} \mathbf{L}
	% 	\]

	% 电子的自旋磁矩 \( \boldsymbol{\mu}_S \) 与自旋角动量 \( \mathbf{S} \) 相关:

	% 	\[
	% 	\boldsymbol{\mu}_S = -g_s \frac{e}{2m_e} \mathbf{S}
	% 	\]

	% 其中,\( g_s \) 是电子的自旋g因子,数值约为2。



	% 总磁矩 \( \boldsymbol{\mu}_J \) 是轨道磁矩和自旋磁矩的矢量和:

	% 	\[
	% 	\boldsymbol{\mu}_J = \boldsymbol{\mu}_L + \boldsymbol{\mu}_S
	% 	\]
	
	% 我们通常考虑磁矩在外磁场方向上的分量(通常是z轴方向)。因此,磁矩z分量为:

	% 	\[
	% 	\mu_{Jz} = \mu_{Lz} + \mu_{Sz}
	% 	\]

	% 轨道角动量和自旋角动量在z方向的分量分别为:

	% 	\[
	% 	L_z = m_L \hbar
	% 	\]
	% 	\[
	% 	S_z = m_S \hbar
	% 	\]

	% 轨道磁矩和自旋磁矩在z方向的分量分别为:

	% 	\[
	% 	\mu_{Lz} = -\frac{e}{2m_e} L_z = -\frac{e}{2m_e} m_L \hbar
	% 	\]
	% 	\[
	% 	\mu_{Sz} = -g_s \frac{e}{2m_e} S_z = -g_s \frac{e}{2m_e} m_S \hbar
	% 	\]

	% 总磁矩的z分量为:

	% 	\[
	% 	\mu_{Jz} = -\frac{e}{2m_e} m_L \hbar - g_s \frac{e}{2m_e} m_S \hbar
	% 	\]



	朗德因子定义为总磁矩和总角动量的比值,即:

		\[
		\mu_{Jz} = -g_J \frac{e}{2m_e} J_z
		\]

	其中,总角动量z分量 \( J_z \) 为:

		\[
		J_z = m_J \hbar
		\]

	因此,总磁矩z分量可以表示为:

		\[
		\mu_{Jz} = -g_J \frac{e}{2m_e} m_J \hbar
		\]

	将上面两个关于 \(\mu_{Jz}\) 的表达式相等,并解出 \( g_J \):

		\[
		-g_J \frac{e}{2m_e} m_J \hbar = -\frac{e}{2m_e} m_L \hbar - g_s \frac{e}{2m_e} m_S \hbar
		\]

	两边同时除以 \( -\frac{e}{2m_e} \hbar \):

		\[
		g_J m_J = m_L + g_s m_S
		\]

	根据角动量合成规则,有:

		\[
		m_J = m_L + m_S
		\]

	所以:

		\[
		g_J m_J = m_L + g_s (m_J - m_L)
		\]

	化简后得到:

		\[
		g_J m_J = g_s m_J + (1 - g_s) m_L
		\]

	两边同时除以 \( m_J \):

		\[
		g_J = g_s + \left(1 - g_s\right) \frac{m_L}{m_J}
		\]


	根据量子力学中的角动量耦合理论,期望值为:

		\[
		\frac{\langle m_L \rangle}{m_J} = \frac{J(J+1) + L(L+1) - S(S+1)}{2J(J+1)}
		\]

	因此,朗德因子的最终表达式为:

		\[
		g_J = g_s \frac{J(J+1) + S(S+1) - L(L+1)}{2J(J+1)} + \frac{J(J+1) + L(L+1) - S(S+1)}{2J(J+1)}
		\]

	将 \( g_s = 2 \) 代入,得到:

		\[
		g_J = 1 + \frac{J(J+1) + S(S+1) - L(L+1)}{2J(J+1)}
		\]

	这就是朗德因子的一般表达式。


% 思考题5
\begin{question}
	请利用单电子情况下的(36)式,并结合钠双黄线的平均波长及其波长差($\lambda_1$ = 589.0 nm,$\lambda_2$  = 589.6 nm),估算一下钠原子内部的磁感应强度 $B_{int}$ 的值(提示:单电子情况下,两谱线的能级差为势能的两倍,即有 $\Delta E = \Delta U =2 \mu_B B$ ;另需要利用到光子波长和频率之间的关系式。答案:钠原子内部的磁感应强度 $B_{int}$ 的值为 18.5T)。
\end{question}

	由$\nu = \frac{c}{\lambda}$,得$\Delta \nu = \frac{c}{\lambda^2}\Delta \lambda$;再由$\Delta E = \Delta U =2 \mu_B B$,得

	\[
		B = \frac{h \Delta \nu}{2 \mu_B} = \frac{h c \Delta \lambda}{2 \mu_B \lambda^2} = \frac{1240eV \cdot nm \times 0.6nm}{2\times 0.5788\times 10^{-4}ev\cdot T^{-1} \times (589.3nm)^2} \approx 18.5T
	\]




% 思考题6
\begin{question}
	请结合第 5 题的计算结果,说明弱外磁场 $B_{ext} \ll B_{int}$ 成立时弱外磁场 $B_{ext}$ 的取值范围,并确认本实验中电磁体的磁感应强度符合弱外磁场 $B_{ext} \ll B_{int}$ 条件。
\end{question}


	当外磁场 $B_{ext} \ll B_{int}$时,可认为是弱磁场,此处可取$B_{ext} \approx 0.1T$



% 思考题7
\begin{question}
	请结合力与势能的关系式 $\vec{F} = -\nabla U $ 并利用(11)式,试推导磁矩在非均匀外磁场中的受力大小为 $F_z = \mu_z \frac{\partial B_z}{\partial z} \ (B_x = B_y = 0)$(设外磁场方向在 z 轴方向,$F_z$ 为力在 z 方向上分量的大小)(提示:请利用郭硕鸿《电动力学》(第二版)一书附录中的矢量运算公式)。
\end{question}

	由$\vec{F} = -\nabla U \quad U = -\vec{\mu} \cdot \vec{B}$,得$\vec{F} = \nabla(\vec{\mu} \cdot \vec{B}) = \nabla(\mu_z B_z) \Rightarrow F_z = \mu_z \frac{\partial B_z}{\partial z} $




\clearpage

% 思考题8
\begin{question}
	请结合朗德因子的一般表达式(28)式,以及两个角动量耦合的一般规则(20)至(23)式,计算表 3 中汞原子 546.1nm 谱线对应的上下两个能级的各量子数及不同谱线(能级跃迁)的朗德因子(见图 9)。用“格罗春图”33(Grotrain 图)来表示汞原子 546.1nm谱线不同能级之间可能的跃迁。
\end{question}









% 思考题9
\begin{question}
	请回答什么是“反常塞曼效应”和“正常塞曼效应”,两者之间的区别是什么。请思考什么是“帕邢-巴克效应”及其形成的原因。
\end{question}


	塞曼效应是指在磁场作用下,光谱线发生分裂的现象。根据分裂的特征和具体情况,塞曼效应可以分为正常塞曼效应和反常塞曼效应。

		\begin{itemize}
			\item 正常塞曼效应
			
				正常塞曼效应是最简单的情况,出现在某些特定条件下:

				\begin{itemize}
					\item 光谱线在弱磁场中分裂为三个分量,即一个不变的中心线和两个对称移动的旁线。
					\item 这种分裂的光谱线对应于自旋为零的原子,只有轨道角动量的贡献。
					\item 旁线的频率偏移量与磁场强度成正比,且偏移量相等,表现为均匀分裂。
					\item 在正常塞曼效应中,电子的轨道角动量与磁场相互作用导致能级分裂,而电子自旋不参与。结果,在原来未分裂的能级上观察到三个分立的能级,其分裂间隔与外加磁场成正比。
				\end{itemize}
			
			
			\item 反常塞曼效应
			
				反常塞曼效应比正常塞曼效应更复杂,出现的条件如下:

				\begin{itemize}
					\item 光谱线在磁场中分裂为多个分量,且这些分量的数目和位置不能简单地用正常塞曼效应解释。
					\item 这种效应出现在电子具有非零自旋的原子中,电子的轨道角动量和自旋角动量共同作用导致能级分裂。
					\item 分裂的具体特征依赖于电子的自旋和轨道角动量的组合,导致复杂的分裂模式。
					\item 在反常塞曼效应中,自旋角动量与轨道角动量耦合,导致更多能级分裂,并且这些分裂间隔不再是均匀的。这种效应需要采用量子力学中的自旋-轨道耦合理论进行解释。
				\end{itemize}
			
			
		\end{itemize}


	正常塞曼效应与反常塞曼效应的区别:
		
		\begin{enumerate}
			\item 自旋贡献:正常塞曼效应中没有电子自旋贡献(电子自旋为零),而反常塞曼效应中有自旋贡献(电子自旋非零)。
			\item 分裂数量:正常塞曼效应的光谱线分裂为三个分量,而反常塞曼效应的分裂数量更为复杂且不固定。
			\item 分裂间隔:正常塞曼效应的分裂间隔是均匀的,反常塞曼效应的分裂间隔不均匀。
		\end{enumerate}

	帕邢-巴克效应

	帕邢-巴克效应是另一种光谱分裂现象,出现在强磁场条件下:

		\begin{itemize}
			\item 在非常强的磁场中,电子的轨道运动和自旋运动的耦合被强磁场打破,轨道角动量和自旋角动量相互独立地与外加磁场作用。
			\item 这种效应发生时,能级分裂的方式发生改变,从塞曼效应的分裂模式转变为帕邢-巴克效应的分裂模式。
			\item 帕邢-巴克效应的分裂特征更复杂,因为需要考虑轨道角动量和自旋角动量分别与外加磁场的相互作用。
			\item 形成原因:帕邢-巴克效应的形成是因为在强磁场中,自旋-轨道耦合效应变得不再显著,电子的轨道角动量和自旋角动量分别与磁场相互作用,导致能级分裂方式的改变。这种效应出现在非常强的磁场中,通常是比塞曼效应中使用的磁场强度高得多的条件下。
		\end{itemize}


	总结
	\begin{itemize}
		\item 正常塞曼效应:简单的三重分裂,适用于自旋为零的情况。
		\item 反常塞曼效应:复杂的多重分裂,适用于有自旋贡献的情况。
		\item 帕邢-巴克效应:在强磁场中发生的复杂分裂,轨道角动量和自旋角动量分别与磁场相互作用。
	\end{itemize}








% 思考题10
\begin{question}
	请回答电子的“自旋—轨道耦合”的本质是什么?它与电子之间的“LS 耦合”的区别是什么?
\end{question}


	电子的“自旋-轨道耦合”的本质是指电子的自旋角动量(自旋)与其轨道角动量(轨道运动)之间的相互作用。这种相互作用的本质可以从相对论效应来解释,当电子在绕核运动时,它的轨道运动会产生一个有效磁场,这个磁场与电子的自旋磁矩相互作用,导致自旋和轨道角动量之间的耦合。

	具体来说,电子在核周围运动时,感受到的核电场在电子参考系中会表现为一个磁场。这个磁场与电子的自旋磁矩相互作用,导致能级的分裂,并影响电子的能量状态和光谱特性。

	LS耦合(Russell-Saunders耦合)是描述多电子原子中电子之间相互作用的一种方法。LS耦合中,多个电子的轨道角动量首先相加成总轨道角动量L,多个电子的自旋角动量相加成总自旋角动量S,然后再考虑总轨道角动量L与总自旋角动量S之间的相互作用,形成总角动量J。

	具体区别如下:

		\begin{itemize}
			\item 耦合对象:

				\begin{itemize}
					\item 自旋-轨道耦合:单个电子的自旋角动量与其轨道角动量之间的相互作用。
					\item LS耦合:多个电子的总轨道角动量与总自旋角动量之间的相互作用。
				\end{itemize}
			
			\item 适用范围:

				\begin{itemize}
					\item 自旋-轨道耦合:描述单个电子在原子核电场中的相互作用。
					\item LS耦合:描述多电子原子中电子之间的相互作用,特别是轻原子中的多电子系统。
				\end{itemize}
			
			\item 相对强度:

				\begin{itemize}
					\item 在轻原子中,电子之间的库仑相互作用比自旋-轨道耦合强,因此LS耦合适用。
					\item 在重原子中,自旋-轨道耦合变得更强,可能需要使用jj耦合来描述。
				\end{itemize}
			
		\end{itemize}
	
	总结:

		\begin{itemize}
			\item 自旋-轨道耦合描述的是单个电子的自旋和轨道角动量之间的相互作用。
			\item LS耦合描述的是多电子原子中电子的总轨道角动量与总自旋角动量之间的相互作用。
			\item 自旋-轨道耦合适用于描述单电子系统,而LS耦合适用于描述轻原子中的多电子系统。
		\end{itemize}






% 思考题11
\begin{question}
	请结合多电子原子及电子组态的相关知识,思考为什么像汞原子一样有两个价电子的元素(氦 He 和镁 Mg 等第二族(碱土族)元素),会有两套不同的谱线(一套是单线结构,一套是双线结构)。
\end{question}

	在多电子原子中,尤其是像汞原子一样有两个价电子的元素(例如氦(He)和镁(Mg)等第二族(碱土族)元素),会出现两套不同的谱线结构(单线结构和双线结构)。这种现象与电子的自旋-轨道耦合和电子组态有关。

	\textbf{自旋-轨道耦合与谱线结构}
	
	电子的自旋-轨道耦合导致能级的分裂,从而影响光谱线的结构。对于有两个价电子的元素,电子的自旋-轨道耦合使得这些电子的能级分裂为多个子能级,因此我们会观察到不同的谱线结构。

	\textbf{单线结构}
	
	单线结构通常是由电子跃迁中不涉及自旋-轨道耦合的跃迁引起的。这种情况常见于:

		\begin{itemize}
			\item 电子的自旋和轨道角动量之间的耦合较弱或者可以忽略不计。
			\item 电子跃迁发生在没有自旋-轨道耦合分裂的能级之间。
			\item 对于氦(He)和镁(Mg)等第二族元素,在某些电子跃迁过程中,自旋-轨道耦合的效应很小,导致我们观察到单线结构的谱线。
		\end{itemize}


	\textbf{双线结构}
	
	双线结构通常是由自旋-轨道耦合导致的能级分裂引起的。这种情况常见于:

		\begin{itemize}
			\item 电子的自旋和轨道角动量之间的耦合较强。
			\item 电子跃迁发生在经历自旋-轨道耦合分裂的能级之间。
			\item 对于有两个价电子的元素,自旋-轨道耦合使得原本简并的能级分裂为两个或多个子能级。例如,在汞(Hg)原子中,两个价电子的自旋和轨道角动量耦合导致$^1P_1$态和$^3P_{0,1,2}$态的能级分裂,从而在跃迁时产生多条谱线,即双线结构或多线结构。
		\end{itemize}


	% 具体示例
	
	% 以氦原子为例:

	% 在氦原子的$2^1P_1 \rightarrow 1^1S_0$跃迁中,没有自旋-轨道耦合的能级分裂,因此我们会看到单线结构。
	% 在氦原子的$2^3P_{0,1,2} \rightarrow 1^1S_0$跃迁中,由于自旋-轨道耦合,$2^3P$态分裂为$2^3P_0$、$2^3P_1$和$2^3P_2$,导致我们观察到双线结构或多线结构。
	
	总结来说,第二族元素(如氦和镁)会有两套不同的谱线结构,主要是因为自旋-轨道耦合在不同电子跃迁中起不同的作用。自旋-轨道耦合导致能级的分裂,从而在光谱中产生双线结构或多线结构,而在没有自旋-轨道耦合分裂的情况下,则表现为单线结构。








% 思考题12
\begin{question}
	设 F-P 标准具两反射面之间的距离为 d=2 mm,请根据(47)式估计汞原子 546.1nm 谱线的自由光谱范围。
\end{question}


	由$\Delta \lambda = \frac{\lambda_1 \lambda_2}{2 n d} \approx \frac{\lambda^2}{2 n d}$,得自由光谱范围为:

	\[
		\Delta \lambda = \frac{(546.1 nm)^2}{2 \times 1 \times 2mm} \approx 0.074nm
	\]


	即,入射光波长在$(546.1 \pm 0.074) nm$ 间变化,所产生的干涉圆环不发生重叠。



% 思考题13
\begin{question}
	请根据(38)式,估计在外磁场为 B=1T 时观察汞原子 546.1nm 谱线分离所要求的仪器分辨率的 $\dfrac{\lambda}{\Delta \lambda}$ ,并讨论用 F-P 标准具观测的必要性(一般棱镜摄谱仪的理论分辨率为$10^3$~$10^4$,F-P 标准具的理论分辨率为 $10^5$~$10^7$,实际分辨率比理论值要略低一些)。
\end{question}





% 思考题14
\begin{question}
	仔细观察垂直磁场方向观察,旋转偏振片至 45° 角的纪录,会发现同一级条纹在磁场中分离成不只三条,请解释出现这一现象的原因。
\end{question}







% 思考题15
\begin{question}
	本实验要求精度为实验测量误差小于等于 5\%,请分析本实验误差的主要来源,并提出相应的修正方法。
\end{question}








% 思考题16
\begin{question}
	请尝试计算钠双黄线(又称“钠 D 线”,是由钠原子从 $^2P_{\frac{1}{2}, \frac{3}{2}}$ 到 $^2S_{\frac{1}{2}}$ 态的跃迁所产生)谱线的塞曼分裂(如图 21),可能的话,设计具体实验步骤并进行观察验证。
\end{question}











\clearpage
\begin{table}
	\renewcommand\arraystretch{1.7}
	\centering
	\begin{tabularx}{\textwidth}{|X|X|X|X|}
	\hline
	专业:& 物理学 &年级:& 2022级 \\
	\hline
	姓名:& 戴鹏辉\&崔瑜 & 学号:& 22344016\&22344017 \\
	\hline
	室温:& 26℃ & 实验地点: & A508	\\
	\hline
	学生签名:& & 评分: &\\
	\hline
	实验时间:& 2024/6/6 & 教师签名:&\\
	\hline
	\end{tabularx}
\end{table}

\section{D4 \quad 塞曼效应 \quad\heiti 实验记录}
\subsection{实验内容和步骤}


	\subsubsection{实验光路搭建}
	
		\begin{enumerate}
			
			\item 调整相机模块
				\begin{itemize}
					\item 将相机固定在光学导轨上的另一端约 520 mm 处,并锁紧托板。
					\item 松开升降调节架的螺钉,调整杆的高度,并通过旋转来微调升降调节架,确保相机与汞灯处于同一高度,此时汞灯图像位于屏幕中心。
					\item 调节镜头上的焦距和光圈,使图像清晰。
				\end{itemize}
			
			
			\item 调整聚光透镜和偏振片模块
				
				\begin{itemize}
					\item 将聚光透镜和偏振片固定在相机右侧的导轨上,并以同样方法调节高度,此时光斑将充满整个视频窗口。
					\item 松开偏振片旋钮,旋转偏振片,使竖直白色刻度对准 90°刻线。
				\end{itemize}
			
			\item 调节 F-P 标准具和干涉滤光片模块
				
				\begin{itemize}
					\item 将干涉滤光片和 F-P 标准具置于相机和偏振片之间,调节高度一致,并使标准具尽可能靠近相机镜头(避免杂散光),此时视频窗口会出现干涉圆环。
					\item 适当调节聚光透镜的位置和相机镜头光圈,使图像亮度适中。
					\item 调节精密调整架的 XY 旋钮,使干涉圆环处于视频窗口中心。
					\item 调节相机镜头后焦,获得清晰干涉图像。
				\end{itemize}
			
				% \begin{figure}[htbp]
				% 	\centering
				% 	\includegraphics[width=0.6\textwidth]{2.jpg}
				% 	\caption{垂直磁场方向,偏振片90°}
				% 	\label{fig:Zeeman1}
				% \end{figure}
		\end{enumerate}
	
	


	\subsubsection{观察谱线分裂}

		首先观察垂直磁场方向的分裂情况,如\cref{fig:ZeemanVertical}:
		\begin{figure}[htbp]
			\centering
			\subfloat[0A电流,垂直磁场方向,偏振片90°]
			{\includegraphics[width=0.45\textwidth]{2.jpg}\label{fig:Zeeman2}}
			\quad
			\subfloat[3.15A电流,垂直磁场方向,偏振片90°]
			{\includegraphics[width=0.45\textwidth]{3.jpg}\label{fig:Zeeman3}}
			\quad
			\subfloat[4A电流,垂直磁场方向,偏振片90°]
			{\includegraphics[width=0.45\textwidth]{6.jpg}\label{fig:Zeeman6}}
			\quad
			\subfloat[3.15A电流,垂直磁场方向,偏振片0°]
			{\includegraphics[width=0.45\textwidth]{4.jpg}\label{fig:Zeeman4}}
			\quad
			\subfloat[4.5A电流,垂直磁场方向,偏振片45°]
			{\includegraphics[width=0.45\textwidth]{14.jpg}\label{fig:Zeeman14}}
			\quad

			\caption{垂直磁场方向的塞曼效应分裂情况}
			\label{fig:ZeemanVertical}
		\end{figure}

		在观察平行磁场方向的塞曼效应分裂情况,如\cref{fig:ZeemanParallel}:

		\begin{figure}[htbp]
			\centering
			\subfloat[0A电流,平行磁场方向,偏振片90°]
			{\includegraphics[width=0.45\textwidth]{7.jpg}\label{fig:Zeeman7}}
			\quad
			\subfloat[0A电流,平行磁场方向,偏振片45°]
			{\includegraphics[width=0.45\textwidth]{8.jpg}\label{fig:Zeeman8}}
			\quad
			\subfloat[0A电流,平行磁场方向,偏振片0°]
			{\includegraphics[width=0.45\textwidth]{9.jpg}\label{fig:Zeeman9}}
			\quad
			\subfloat[1A电流,平行磁场方向,偏振片0°]
			{\includegraphics[width=0.45\textwidth]{10.jpg}\label{fig:Zeeman10}}
			\quad
			\subfloat[1A电流,平行磁场方向,偏振片45°]
			{\includegraphics[width=0.45\textwidth]{11.jpg}\label{fig:Zeeman11}}
			\quad
			\subfloat[1A电流,平行磁场方向,偏振片90°]
			{\includegraphics[width=0.45\textwidth]{12.jpg}\label{fig:Zeeman12}}
			\quad

			\caption{平行磁场方向的塞曼效应分裂情况}
			\label{fig:ZeemanParallel}
		\end{figure}



\clearpage

\subsection{实验原始数据记录}

	实验所的图像见\cref{fig:ZeemanVertical}、\cref{fig:ZeemanParallel},数据记录见\cref{fig:data}

	\begin{figure}[htbp]
		\centering
		% \subfloat[原始数据1]
		{\includegraphics[width=0.6\textwidth]{OriginalData.jpg}\label{fig:OriginalData}}
		\quad

		\caption{原始数据记录}
		\label{fig:data}
	\end{figure}





	





\clearpage
\begin{table}
	\renewcommand\arraystretch{1.7}
	\begin{tabularx}{\textwidth}{|X|X|X|X|}
	\hline
	专业:& 物理学 &年级:& 2022级\\
	\hline
	姓名: & 戴鹏辉\&崔瑜 & 学号:& 22344016\&22344017\\
	\hline
    日期:& 2024/6/8 & 评分: &\\
	\hline
	\end{tabularx}
\end{table}

\section{D4 \quad 塞曼效应 \quad\heiti 分析与讨论}

\subsection{实验数据分析}

	\subsubsection{分析塞曼分裂情况}

		如\cref{fig:Zeeman2}所示,在不加外磁场的情况下,能较为清晰的看到圆环。
		\begin{itemize}
			\item 首先在磁场横向(垂直)方向上观察,将观察到三个偏振光,$\sigma^+$左旋偏振光和$\sigma^-$右旋偏振光的投影和$\pi$偏振光,$\sigma^+$左旋偏振光和$\sigma^-$右旋偏振光的投影与磁场方向$B$垂直,$\pi$偏振光与磁场$B$平行。
			
				\begin{enumerate}
					\item 电流3.15A,偏振片90°时,如\cref{fig:Zeeman3}所示,谱线分裂为3条,只有一个线偏振光透过偏振片,综合偏振片置于0°时的图像分析,该线偏振光为$\pi$偏振光。
					\item 电流3.15A,偏振片0°时,如\cref{fig:Zeeman4}所示,谱线明显的分为两份圆环,且每份圆环还分裂为较模糊的三条谱线,谱线总共分裂为6条,此时透过0°偏振片的是$\sigma^+$左旋偏振光和$\sigma^-$右旋偏振光的投影。
					\item 电流4.5A,偏振片45°时,如\cref{fig:Zeeman14}所示,三个线偏振光都有通过,谱线应分裂为九条,但因为仪器有误差,导致有两条重叠,实际中只观察到7条。
					
				\end{enumerate}

			\item 然后在磁场竖向(平行)方向上观察,将观察到两个圆偏振,$\sigma^+$左旋偏振光和$\sigma^-$右旋偏振光。
				
				\begin{enumerate}
					\item 电流为0,如\cref{fig:Zeeman7}、\cref{fig:Zeeman8}、\cref{fig:Zeeman9},无论偏振片角度如何变化,图像都没有发生改变,判断此时透过的光为圆偏振光。
					\item 电流为1A,如\cref{fig:Zeeman10}、\cref{fig:Zeeman11}、\cref{fig:Zeeman12},无论偏振片角度如何变化,图像都没有发生改变,判断此时透过的光为圆偏振光,谱线分裂为两条粗环,分别代表$\sigma^+$左旋偏振光和$\sigma^-$右旋偏振光,理论上应能看到六条分裂谱线,但因为仪器有限,只能看到两条粗环。
				\end{enumerate}
		\end{itemize}
		


\clearpage

	\subsubsection{计算荷质比e/m}

		对垂直磁场方向观察、4A电流(即0.965T的磁感应强度),偏振片90°放置的情况详细讨论,即\cref{fig:Zeeman6}。
		测量其各环的半径(对于测量圆环半径,我完成了一个python程序用以辅助,具体代码附后,测量效果图如\cref{fig:measure1})。测量结果如\cref{tbl:table-measure1}:

		\begin{figure}[htbp]
			\centering
			\includegraphics[width=0.6\textwidth]{measure1.png}
			\caption{python程序实现的测量半径的效果图}
			\label{fig:measure1}
		\end{figure}



		\begin{table}
			\centering
			\begin{tblr}{
			  cells = {c},
			  vline{1,5-7} = {-}{},
			  hline{1-2,7-9} = {-}{},
			}
			测量次数 & $D_K$       & $D_{K-1}$     & $D_{\alpha}$       & $ \frac{e}{m} / (C \cdot kg)$ & 相对误差   \\
			1    & 169.2909 & 229.0737 & 176.317  & 1.99159E+11 & 13.29\% \\
			2    & 169.712  & 229.0494 & 176.5046 & 1.94136E+11 & 10.43\% \\
			3    & 170.3164 & 229.3905 & 176.3256 & 1.72312E+11 & -1.98\% \\
			4    & 169.6316 & 228.1239 & 176.7532 & 2.07102E+11 & 17.81\% \\
			5    & 169.8434 & 228.8916 & 176.395  & 1.88189E+11 & 7.05\%  \\
			平均   & 169.7589 & 228.9058 & 176.4591 & 1.92158E+11 & 9.31\%  \\
			标准差  & 0.372638 & 0.473123 & 0.180774 &             &         
			\end{tblr}
			\caption{各圆环半径测量结果}
			\label{tbl:table-measure1}
		\end{table}



		% \begin{table}
		% 	\centering
		% 	\begin{tblr}{
		% 	  cells = {c},
		% 	  vline{1-2,6,7} = {-}{},
		% 	  hline{1-2,7,8} = {-}{},
		% 	}
		% 	测量次数 & $D_K$       & $D_{K-1}$     & $D_{\alpha}$       & $ \frac{e}{m} / (C \cdot kg)$ & 相对误差   \\
		% 	1    & 169.2909 & 229.0737 & 176.317  & 1.99159E+11 & 13.29\% \\
		% 	2    & 169.712  & 229.0494 & 176.5046 & 1.94136E+11 & 10.43\% \\
		% 	3    & 170.3164 & 229.3905 & 176.3256 & 1.72312E+11 & -1.98\% \\
		% 	4    & 169.6316 & 228.1239 & 176.7532 & 2.07102E+11 & 17.81\% \\
		% 	5    & 169.8434 & 228.8916 & 176.395  & 1.88189E+11 & 7.05\% \\
		% 	平均   & 169.7589 & 228.9058 & 176.4591 & 1.92158E+11 & 9.31\%
		% 	\end{tblr}
		% 	\caption{各圆环半径测量结果}
		% 	\label{tbl:table-measure1}
		% \end{table}

		可以看到,测量值相较公认值$1.758 \times 10^{11} / (C \cdot kg)$有一定的差距,下面详细分析测量误差。


		由误差传递公式:

		\[
			\sigma_{\frac{e}{m}} = \sqrt{\left(\frac{\partial\left(\frac{e}{m}\right)}{\partial D_\alpha} \cdot \sigma_{D_\alpha}\right)^2 + \left(\frac{\partial\left(\frac{e}{m}\right)}{\partial D_K} \cdot \sigma_{D_K}\right)^2 + \left(\frac{\partial\left(\frac{e}{m}\right)}{\partial D_{K-1}} \cdot \sigma_{D_{K-1}}\right)^2}
		\]

		其中各项的值为:

		\[
			\frac{\partial\left(\frac{e}{m}\right)}{\partial D_\alpha} = \frac{8\pi c D_\alpha}{n d B (D_{K-1}^2 - D^2_{K})} = 29235304771
		\]

		\[
			\frac{\partial\left(\frac{e}{m}\right)}{\partial D_K} = \frac{8\pi c D_K (D_{\alpha}^2 - D_K^2) }{n d B (D_{K-1}^2 - D^2_{K})^2} = 2766906464
		\]

		\[
			\frac{\partial\left(\frac{e}{m}\right)}{\partial D_{K-1}} = \frac{8\pi c D_{K-1} (D_K^2 - D_{\alpha}^2) }{n d B (D_{K-1}^2 - D^2_{K})^2 } = -3730944971
		\]

		可得间接测量量$e/m$的标准差为:

		\[
			\begin{aligned}
			\sigma_{e/m} &= \sqrt{(29235304771 \times 0.180773994)^2 + (2766906464 \times 0.372637827)^2 + (-3730944971 \times 0.473122503)^2 } \\
			&= 5666570886 = 5.67 \times 10^9
			\end{aligned}
		\]


		展伸不确定度$U = k \sigma_{e/m}$,置信概率取95 \% 时,$k = 1.96$,则$U = k \sigma_{e/m} = 1.11 \times 10^{10}		$
		
		则最后测量结果为:$e/m = (1.92 \pm 0.11) \times 10^{11} C \cdot kg$

	\subsubsection{分析误差来源}

		\begin{enumerate}
			\item 在测量半径时,即使是使用了自己所编写的程序以期望更精确的测量,但从测量结果来看,每次测量之间仍然有不小的测量误差。主要的原因是成像不够清晰,以及圆环本身的宽度较宽,导致选择点的时候偏差较大。
			\item 在计算传递误差时,我们认为除了三个半径之外的量都为常数,但实际上,磁感应强度是通过“电流-磁感应强度”对照表得到的,这其中可能有一定的误差。
		\end{enumerate}











\subsection{实验总结}
    
    \subsubsection{实验问题总结}

        \begin{enumerate}
            \item 由于不同光具的高度不同,以及光源的限制,基本无法得到一个即各光具都共轴、又能够在全屏幕范围成像的图像。最多只能得到图像一半清晰,一半黑的图像。
            \item 在观察平行磁场方向的塞曼效应分裂情况时,由于孔径很小,得到的图像只有部分看得到像,但是这部分的像很清晰。
        \end{enumerate}

    \subsubsection{实验分工}
		实验过程由两人合作完成,报告部分分工如下。
        \begin{enumerate}
            \item 戴鹏辉:实验思考题部分,分析讨论中的荷质比计算、误差分析部分。
            \item 崔瑜:实验原理部分,分析讨论中的塞曼效应偏振情况分析。
        \end{enumerate}

    \begin{figure}[htbp]
        \centering
        \includegraphics[width=0.6\textwidth]{table.jpg}
        \caption{整理后实验桌照片}
        \label{fig:table}
    \end{figure}

	


\subsection{代码附件}

	代码所使用的方法也是通过选取圆上的三个点,计算出圆的半径。

\definecolor{mygreen}{rgb}{0,0.6,0}
\definecolor{mygray}{rgb}{0.5,0.5,0.5}
\definecolor{myorange}{rgb}{1.0,0.4,0}
\definecolor{mymauve}{rgb}{0.58,0,0.82}

\lstset{ 
  backgroundcolor=\color{white},   
  basicstyle=\footnotesize\ttfamily,       
  breakatwhitespace=false,         
  breaklines=true,                 
  captionpos=b,                    
  commentstyle=\color{mygreen},    
  deletekeywords={...},            
  escapeinside={\%*}{*)},          
  extendedchars=true,              
  frame=single,                    
  keepspaces=true,                 
  keywordstyle=\color{blue},       
  language=Python,                 
  morekeywords={*,...},            
  numbers=left,                    
  numbersep=5pt,                   
  numberstyle=\tiny\color{mygray}, 
  rulecolor=\color{black},         
  showspaces=false,                
  showstringspaces=false,          
  showtabs=false,                  
  stepnumber=2,                    
  stringstyle=\color{mymauve},     
  tabsize=2,                       
  title=\lstname                   
}





	\begin{lstlisting}[language=Python, caption=测量圆环半径所用的代码]
		import cv2
		import numpy as np
		from scipy.optimize import least_squares
		
		# 用于存储选定点的列表
		points = []
		zoom_factor = 2  # 放大倍数
		
		def calculate_circle(points):
			""" 通过选定的三个点计算圆心和半径 """
			def calc_R(c):
				""" 计算点到圆心 (c[0], c[1]) 的距离 """
				return np.sqrt((points[:, 0] - c[0])**2 + (points[:, 1] - c[1])**2)
		
			def f_2(c):
				""" 计算到圆心的距离差的平方和 """
				Ri = calc_R(c)
				return Ri - Ri.mean()
		
			# 初始估计
			x_m = np.mean(points[:, 0])
			y_m = np.mean(points[:, 1])
			center_estimate = (x_m, y_m)
			result = least_squares(f_2, center_estimate)
			center_2 = result.x
		
			# 计算半径
			Ri = calc_R(center_2)
			R = Ri.mean()
			return center_2, R
		
		def on_mouse(event, x, y, flags, param):
			global points
			global zoom_factor
			global zoom_window_open
		
			if event == cv2.EVENT_MOUSEMOVE:
				if zoom_window_open:
					# 放大区域并显示
					zoom_img = img[max(0, y-50):min(img.shape[0], y+50), max(0, x-50):min(img.shape[1], x+50)]
					zoom_img = cv2.resize(zoom_img, (0, 0), fx=zoom_factor, fy=zoom_factor)
					# 绘制光标
					cv2.line(zoom_img, (zoom_img.shape[1]//2 - 10, zoom_img.shape[0]//2),
							(zoom_img.shape[1]//2 + 10, zoom_img.shape[0]//2), (255, 0, 0), 1)
					cv2.line(zoom_img, (zoom_img.shape[1]//2, zoom_img.shape[0]//2 - 10),
							(zoom_img.shape[1]//2, zoom_img.shape[0]//2 + 10), (255, 0, 0), 1)
					cv2.imshow("Zoom", zoom_img)
		
			elif event == cv2.EVENT_LBUTTONDOWN:
				zoom_window_open = True
				# 显示放大镜窗口
				cv2.namedWindow("Zoom")
				cv2.moveWindow("Zoom", x + 50, y + 50)
		
			elif event == cv2.EVENT_LBUTTONUP:
				zoom_window_open = False
				cv2.destroyWindow("Zoom")
				# 添加点
				points.append((x, y))
				print(f"Point selected: {(x, y)}")
				cv2.circle(img, (x, y), 5, (0, 255, 0), -1)
				cv2.imshow("Image", img)
		
				# 如果已经选择了三个点,则计算圆
				if len(points) == 3:
					points_np = np.array(points)
					center, radius = calculate_circle(points_np)
					print(f"Circle center: {center}, radius: {radius}")
					cv2.circle(img, (int(center[0]), int(center[1])), int(radius), (255, 0, 0), 2)
					cv2.imshow("Image", img)
					points.clear()  # 清空点列表以便继续选择新的点
		
		# 读取图像
		img = cv2.imread("6.jpg")
		
		# 初始放大镜窗口状态
		zoom_window_open = False
		
		# 显示图像并绑定鼠标事件
		cv2.imshow("Image", img)
		cv2.setMouseCallback("Image", on_mouse)
		
		cv2.waitKey(0)
		cv2.destroyAllWindows()
		

	\end{lstlisting}
		
		
\end{document}
